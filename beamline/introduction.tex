\section{Introduction}
\label{introduction}

The physics program for CLAS12 in Hall~B at Jefferson Lab requires the use of electron beams of various
energies and currents that impinge upon targets ranging from liquid hydrogen to lead. A significant part of
the physics program includes running with polarized targets that require a rastered beam on the target. In
order to extract experimental observables, accurate measurements of the beam charge and polarization are
required. Also, for safe and efficient operation of a large, open acceptance spectrometer, proper shielding
and a stable beam with a small lateral size and minimal beam halo are necessary. 

The Hall~B beamline is designed to satisfy the experimental requirements and to provide the necessary controls
and monitoring of the electron beam properties for safe and efficient operation of CLAS12. The key set of
parameters required by experiments with CLAS12 is listed in Table~\ref{tab:beam_par}. The main challenges
for the beamline setup are the open acceptance of CLAS12 and the close proximity of various sensitive detectors
to the target and beam. Such challenges were successfully overcome in Hall~B in the past for CLAS
experiments~\cite{CLAS} and the Heavy Photon Search (HPS) experiment~\cite{HPS}.

 \begin{table}[htb]
 \centering
 \begin{tabular}{|c|c|c|}
\hline
Parameter & Requirement &Unit \\ \hline 
Beam energies &  $\le 11$& GeV \\ \hline
%$\delta p/p$ & $< 10^{-4}$ & \\ \hline 
Beam currents & $<~500$ & nA \\ \hline
Current instability & $\sim 10$ &\% \\ \hline 
Accuracy of current& $\sim 1$ &\% \\ 
measurement & &\\ \hline 
Beam widths ($\sigma_x $, $\sigma_y$)&$< 300$& $\mu$m \\ \hline 
Position stability &$< 200$ &$\mu$m \\ \hline
Divergence& $< 100$& $\mu$rad \\ \hline 
Beam halo ($> 5\sigma$) &$< 10^{-4}$& \\ \hline
Beam polarization &$>$80 & \%\\ \hline
Accuracy of polarization&$<3$&\% \\ 
measurement&& \\ \hline
\end{tabular}
\caption{Nominal required Hall B beam parameters.} 
\label{tab:beam_par}
\end{table}

A few key modifications to the beamline used during the lower-energy run of the HPS experiment
\cite{HPSBeamline} have been introduced in order to establish high-quality physics beams in Hall~B and run
CLAS12 at the design luminosity of $10^{35}$ cm$^{-2}$s$^{-1}$. Additions to the beamline for high-energy
running of CLAS12 include a new intermediate beam dump upstream of the hall, a cryogenic target system,
shielding downstream of the target to protect the CLAS12 detectors from electromagnetic backgrounds, and
the M{\o}ller polarimeter for beam polarization measurements.   

This paper will discuss the design of the Hall~B beamline for CLAS12 and its performance during the 2018
experimental run. It will review the beamline instrumentation used to measure and monitor the beam parameters
and to protect the CLAS12 detectors against errant beam motion. As will be demonstrated, excellent quality and
stability of the CEBAF beams, coupled with the Hall~B beamline protection systems, allowed operation of the
CLAS12 detector at the design luminosity.
