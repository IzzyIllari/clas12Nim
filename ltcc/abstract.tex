\begin{abstract}

The CLAS Cherenkov threshold gas detector at Jefferson Lab was instrumental in the electron identification
in the 6 GeV era at Jefferson Lab.
The detector's scope has been modified to identify π+ and π− for momenta
greater than 3.5 GeV/c, thus becoming a Low Threshold Cherenkov Counter (LTCC).
This was accomplished with a refurbishment of the gas container, its windows,
the mirrors, the winston light collecting cones and pmt system.

The design, construction, and performance of the LTCC is described.
The detector consists of 180 optical modules.
Each module consists of three adjustable mirrors of lightweight composite construction and Winston cones,
all re-surfaced with a high reflective coating.
The 5-in. photomultiplier tubes glasses have been treated with p-terphenyl to enhance the ultra-violet response.
Efficiencies and response functions have been measured.

\end{abstract}
