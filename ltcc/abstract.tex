\begin{abstract}

The CLAS Cherenkov threshold gas detector at Jefferson Lab was instrumental for electron identification
in the 6 GeV era at Jefferson Lab.
The detector's scope has been modified to identify $\pi^+$ and $\pi^-$ for momenta
greater than 3.5 GeV, thus becoming a Low Threshold Cherenkov Counter (LTCC).
This was accomplished with a refurbishment of the gas container, its windows,
mirrors, Winston light collecting cones, the photomultipliers.

The design, construction, and performance of the refurbished LTCC is described.
The lightweight mirrors and Winston cones have been re-surfaced with a high reflective coating.
The 5-in photomultiplier tube glasses have been treated with p-terphenyl to enhance the ultra-violet response.
The gas volume has been expanded to increase the number of photo-electrons.
The calibration is performed on the single photo-electron (SPE) signal.
Inefficiencies and response functions have been measured.

\end{abstract}
