\section{Requirements}


The LTCC requirements to satisfy an adequate pion/kaon discrimination include:


\begin{itemize}
	\item Maximizing the coverage in each of the six sectors up to an angle of $45^0$.
	\item Minimize the radiation length in the active area of CLAS12.
	\item Fit the LTCC Box in the reduced space between the Drift Chambers and the Time-Of-Flight.
	\item Provide a sensibile response to pions.
	\item Address the gas leaks.
\end{itemize}

The azimuthal and polar coverage was already achieved in the CLAS6 era and was not modified by the refurbish.

The radiation length of the detector was minimized by placing the light collecting cones and photomultiplier tubes
in the regions obscured by the magnet coils. In the active area the window radiation length is $0.02\%$.

The distance between the target and the LTCC is increased by about $2$ m in CLAS12. This bring some of the passive
element in the active area of the detectos behind the LTCC, namely the support structure of the mirrors and the
dectector walls.

The remaining requirements are addressed by the refurbishment and addressed in this paper.
