\section{Requirements}


The LTCC requirements to allow for an adequate pion/kaon discrimination include:


\begin{itemize}
	\item Maximizing the coverage in each of the six sectors up to an angle of 30\mdeg;
	\item Minimizing the radiation length in the active area of CLAS12;
	\item Fitting the LTCC box in the available space between the Drift Chambers \cite{dc2019} and the Forward Time-Of-Flight \cite{ftof2019};
	\item Produce a signal for pions in the momentum range $4-8$ GeV.
\end{itemize}

The azimuthal and polar coverage achieved in the CLAS 6 GeV era and was modified by the refurbishment:
the distance between the target and the LTCC was increased by about 2 m in CLAS12. This brought some of the passive
elements into the active area of the detectors behind the LTCC, namely the support structure of the mirrors, the Winston
cones, the PMT magnet shields, and the detector walls.

The radiation length of the detector was minimized by placing the light collecting cones and photomultiplier tubes
in the regions obscured by the torus magnet coils. In the active area the window radiation length is 0.02$\%$.

The remaining requirements are addressed by the refurbishment and addressed in this paper.
