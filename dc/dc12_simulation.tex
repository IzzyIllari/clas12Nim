\section{Chamber Construction}

\subsection{Construction Overview}

\hskip 0.15 in
The three chamber Regions share the same basic design elements simply
scaled up in size by a factor of 1.5 between R2 and R1 and a factor
of 2 between R3 and R1.  Thus the R3 chambers have a volume which is
about 8 times that of R1 and about 2.5 times that of R2.  
Each chamber is a solid trapezoid 
a pair of wire-supporting endplates that bear both the load of the 
wire tensions and the weight of all associated hardware. A representative 
chamber is shown in Fig.~\ref{dcsector}.  The wire tension load in
R1 and R2 was borne by the stiffness of the endplates only, while in
R3 the wire tension was offset 
by compression in a carbon-fiber outer shell and in a set of carbon-fiber 
posts that were part of the chamber construction.  At the raially outward
end of each chamber, a thick backplate was employed to maintain the relative 
alignment of the endplates, to stiffen the chamber against bending moments, 
and to provide a place to attach gas seals and fittings. At the radially inward 
end of each chamber, the endplates were connected together via a small joining 
piece called the noseplate.  The hardware fabrication and placement 
was of critical importance to the dimensional accuracy of the chambers.

%%%%%%%%%%%%%%%%%%%%%% Figure : DC Sector Schematic %%%%%%%%%%%%%%%%%%%%%%
\begin{figure}[htpb]   
\vspace{4.5cm}
\special{psfile=r3dc.eps hscale=55 vscale=55 hoffset=-30 voffset=-150} 
\caption{\small{Schematic representation of a typical drift-chamber sector 
(here a R3 sector) highlighting some of the common hardware pieces used by each.}}
\label{dcsector}
\end{figure}   
%%%%%%%%%%%%%%%%%%%%%%%%%%%%%%%%%%%%%%%%%%%%%%%%%%%%%%%%%%%%%

\subsection{Endplate Design and Construction}
\hskip 0.15 in

Each endplate contains thousands of accurately positioned holes into which the 
wire-fixture assemblies were placed.  The sense, field, and guard wires in each 
sector were strung between pairs of wire ``feedthroughs''.  As the endplates of 
each chamber faced each other at a 60$^{\circ}$ angle, the wires had to bend 
30$^{\circ}$ at each endplate.  This was accomplished by means of a large radius 
stainless-steel insert at the tip of each feedthrough known as a ``trumpet''.
This trumpet was fitted into an injection-molded Noryl plastic feedthrough.  
All wires are held in place using gold-plated copper crimp pins.  
Low-outgassing epoxy was employed to 
ensure a gas seal around the feedthroughs and crimp pins.  The feedthrough assemblies
are shown in Figs.~\ref{xsect}, \ref{r2_inserts}, and \ref{r3_cut} for R1, R2, 
and R3, respectively.

\subsection{Wire Choice}
\hskip 0.15in

Our ``thin endplate'' design required minimizing wire tensions and
thus diameter or the wire.  In addition, smaller diameter sense wires
require lower operating voltates.
The sense wire for all chambers, supplied by the Luma
Sweden Company, consists of 30-$\mu$m diameter gold-plated tungsten.  
The previous CLAS drift chambers used 20-$\mu$m diameter wire for the
sense wires.  We decided to use the thicker 30-$\mu$m wire for two 
reasons: first, it is significantly tougher making it easier to handle without
accidental kinking and less likely to break, and second, to operate at the same 
gas gain as chambers with the thinner wire requires a high voltage setting 
some 300 V higher and this results in a higher field throughout the cell
and a more linear drift velocity function.  As will be discussed later,
we determined that our on-chamber HVTBs could
handle the slightly higher operating voltages.

Tungsten was chosen because of its durability, 
and the gold-plating of the wires, amounting to a thickness of 0.127~$\mu$m, 
ensures chemical inertness as well as a smooth surface finish.  The expected 
electronics gain and thresholds dictated that the gas gain be a few times 
10$^4$.  Under this condition, the electric field at the surface of the sense 
wires is $\approx$200~kV/cm.  The field-wire diameter was chosen to ensure 
that the electric field at their surface remained below 50~kV/cm to minimize 
conditions which could cause unwanted cathode emissions.  We note that
our choice violated the ``20~kV/cm rule'' proposed by Fabio Sauli and others
to reduce cathode emission.  Our own studies showed that there was no cathode
emission below 50~kV/cm from any wire with good surface finish.  Each batch
of wire was tested with our test device to ensure that at operating field 
values there was no emission.  

We chose 80-$\mu$m gold-plated Cu-Be wire for our field wire.
It is very tough, and is easily plated and resists ``flaking'' of the gold
plating.  The 80-$\mu$m diameter choice was made to obtain the minimum
radius that would satisfy our surface electric field limit.  This small
diameter is important because it means that the field wires could be strung 
at lower tension than a more dense wire, with minimal gravitational sag.  
This minimizes the forces on the endplates which we wanted to keep as 
thin as possible to maximize the solid-angle of the sensitive area of
the chambers.

\subsection{Gas Choice}
\hskip 0.15in

The main requirements for the chamber gas were that it have reasonably low 
multiple scattering, allow for reasonable gas gains, have short collection 
times in order to reduce the random background expected from M{\o}ller 
electrons and target-generated X-rays, and be inexpensive because of the 
large volume of the chambers.  The initial choice of chamber gas was a
50$\%$ argon - 50$\%$ ethane mixture.  However, safety concerns prompted a 
search for a non-flammable gas mixture.  Additional concerns about small gas 
leaks and the proximity of many photomultiplier tubes argued against helium 
mixtures.  Ultimately a 90$\%$ argon - 10$\%$ CO$_2$ mixture was employed 
for several reasons: the gas has a fairly high saturated drift velocity 
($>$ 5~cm/$\mu$s), and it has an operating voltage plateau of several hundred 
volts before breakdown occurs.  The 90$\%$/10$\%$ mixture 
provides good efficiency, adequate resolution, and reasonable collection times.

\subsection{Construction Materials}

\hskip 0.15 in
Due to the complex geometry of CLAS and the difficulty of removing and repairing 
drift chambers, it was crucial to minimize chamber aging.  Care was taken to 
ensure that all materials in contact with the gas volume were clean and ``chamber 
safe'' as defined in Ref~\cite{kadyk}.  All construction was carried out in 
Class-10000 or better clean rooms.

The drift-chamber frames are made primarily of aluminum (R1), fiberglass (R2),
or steel-clad structural foam (R3).  The aluminum and steel endplates were 
manufactured with machine oils and were subsequently cleaned with  
Micro-laboratory detergent from the Cole-Parmer Instrument Company.  The 
fiberglass endplates were machined without any lubricating oils.  Immediately 
prior to chamber assembly all endplates were sequentially cleaned with detergent,
 deionized water, and alcohol, and then blown dry with pure nitrogen gas.  The 
wire feedthroughs, trumpets, and crimp pins were cleaned in an ultrasonic bath 
with detergent and then rinsed in a second ultrasonic bath with deionized water.

During construction three different types of epoxy were used in areas exposed 
to the chamber gas.  Shell Epon resin 826 mixed with Versamid 140, and 
Scotchweld varieties 210 and 2216 were employed.  These mixtures have been 
studied extensively and found not to outgas significantly~\cite{nasa}.

The on-chamber gas tubing employed is mainly stainless steel, with some
nylon tubing included for gas manifolds.  Special care was taken during
all steps of construction and testing to ensure that no oils or
silicones contacted any of the chamber materials.

\subsection{Chamber Wire Stringing}

\hskip 0.15 in
With nearly 130,000 wires to be threaded through the chambers and only 2 years 
for construction, it was important to minimize the time required to string 
each wire.  All chambers were strung with 
the wires running vertically.  The stringing technique involved attaching a 
small steel needle to the wire before threading it through the feedthrough in 
the upper endplate.  The wire was then despooled and gravity acted to bring the 
wire close to the feedthrough in the lower endplate.  A small magnet was then 
used to pull the needle and wire through the lower feedthrough.  After the upper 
crimp pin was attached and remaining slack in the wire was removed, the lower 
crimp pin was slid over the wire and stringing weights were attached to set the 
proper tension.  The lower crimp pin was then attached, completing the process.
  
Wires that wrapped around each other while being threaded through the chamber 
were a major contribution to stringing inefficiency. To avoid the wrapping
problem, a machine was built to spool the wire through the chambers at a fast 
and smooth rate.  Another important development was the design of a crimp pin 
that accepted both the tungsten and aluminum wire types.  Using a thick-walled 
copper pin ensured a good crimp through a range of gap settings~\cite{sbc}.  This 
eliminated the need to use separate crimping tools, each requiring frequent 
calibrations.  This was possible because the wire position was determined by the 
radius of the trumpet at the end of the feedthrough inside of the chamber, and 
not by the concentricity of the pin, feedthrough, and endplate-hole diameters. 
As a result, the average time to string a wire was less than 4 minutes.

\input{region1}

\input{region2}

\input{region3}




