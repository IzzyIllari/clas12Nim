
\subsection{Region Three Construction}

\hskip 0.15 in
The R3 chambers were designed and constructed at Jefferson Laboratory.  The
shape of each 7-m long sector follows the outer contour of the CLAS torus.
The sectors are located outside of the toroid in a field-free region at a 
distance from the target of between 3.0 and 3.5~m.  Each chamber has 2304 
hexagonal drift cells ranging in diameter from 40 to 45~mm, with cell size 
increasing uniformly with layer number.  The wires are laid out with the 
axial superlayer closer to the target.  Each of the 12 wire layers contains 
192 drift cells.

%%%%%%%%%%%%%%%%%%% Figure : Region Three Cross Section %%%%%%%%%%%%%%%%%%
\begin{figure}[htpb]
\vspace{7.9cm}
\special{psfile=r3_cut.eps hscale=70 vscale=70 hoffset=-80 voffset=-165}
\caption{\small{Schematic cross-sectional view of an R3 endplate highlighting
the wire-positioning and chamber-frame hardware.}}
\label{r3_cut}
\end{figure}
%%%%%%%%%%%%%%%%%%%%%%%%%%%%%%%%%%%%%%%%%%%%%%%%%%%%%%%%%%%%%%%%%%%%%%%%%%%

The endplates are a composite assembly consisting of 1-mm thick, pre-drilled 
stainless-steel skins sandwiching 5-cm thick pre-drilled structural foam enclosed 
with aluminum close-out pieces about the perimeter of the foam.  The two endplates 
of each chamber are held apart at their outer radius by a curving 3-cm thick 
composite shell made from 300-$\mu$m skins of carbon fiber, sandwiching a
Nomex-paper honeycomb.  Fig.~\ref{r3_cut} shows a cross section of the 
skin-to-endplate junction. Along the inner radius are five equally spaced, 
thin-walled carbon-fiber cylinders keeping the endplates apart.  Choices for 
these materials were driven by multiple-scattering considerations and the desire 
to combine strength with light weight. Because the R3 chambers are located farthest
from the target where more space is available, it was possible to include 
extra material to build a very rigid, self-supporting structure, thus avoiding 
some of the special challenges for tension transfer peculiar to R1 and R2.

The endplate hole-position accuracy for the wire feedthrough penetrations was 
maintained to 50~$\mu$m.  Before stringing began, the feedthrough assemblies were
glued into the endplates and the sectors were pre-tensioned.  This was 
accomplished with large springs to simulate the load of the tensioned wires.  
The goal during stringing was to keep all wires to within $\pm$10$\%$ of nominal 
tension.  The few percent of wires that were outside of this tolerance were replaced.
The guard wires served the dual function of shaping the electric field around the 
perimeter of the superlayers and pre-stressing the endplates prior to 
stringing. The sense wires were tensioned at 18~g and field and guard wires were 
tensioned at 134~g.  Their relative values were chosen to ensure matching local 
gravitational sag for all the wires~\cite{chew89}.

A 1-cm diameter gas input line running longitudinally over the endplates 
distributed the gas to the downstream end of the chamber. Gas was exhausted 
through a low-impedance 2.5-cm diameter manifold at the upstream end.  The gas 
window on the inner radius was made from 15-$\mu$m aluminized nylon.  All of 
the chamber utilities remained within the shadow of the torus cryostats or beyond 
the 142$^{\circ}$ angle of acceptance.
