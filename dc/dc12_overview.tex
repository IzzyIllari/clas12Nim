\section{Forward Tracking System}
\label{overview}

\hskip 0.15 in
Jefferson Lab recently upgraded its electron accelerator to provide beams with 
energies greater than 11 GeV/c.  To take advantage of new physics that would 
be possible at these higher energies, the CLAS detector was significantly
upgraded, and is now named the ``CLAS12'' detector.  This section describes
the tracking of charged particles in the forward part of the detector.

The CLAS12 forward detector is constructed around a toroidal magnet consisting of six 
iron-free superconducting coils.  The particle detection system consists of drift 
chambers to determine charged-particle trajectories, {\v C}erenkov detectors 
for electron/pion separation, scintillation counters for flight-time 
measurements, and calorimeters to identify electrons and high-energy neutral 
particles.  An overview of the CLAS12 subsystems and geometry may be found in the 
CLAS12 Overview section.  A schematic view of the torus magnet with drift chambers
attached is shown in Fig.~\ref{chambers-and-torus}.  

The forward tracker was designed to track charged particles emerging from the target with
momenta greater than 200~MeV/c over the polar angle range from 7$^{\circ}$ to 
40$^{\circ}$.  Because the coils of the torus magnet represent a 'dead area'
in which we cannot detect charged particles, we designed the chamber endplates
and attached electronics to be as thin as possible.  The resulting azimuthal
coverage varied from 50\% of 2\pi at 5$^{\circ}$ to 80\% of 2\pi at 40$^{\circ}$.

The forward tracking system consists of three regions divided into six
sectors as shown in Fig.~\ref{chambers-and-torus}; located just before, inside, 
and just outside the torus field volume, and referred to as Regions~1, 2, 
and 3, respectively.  Thus in each sector we have a 'Region1' chamber located
in front of the torus, followed by a 'Region2' chamber situated in the magnetic
field volume, between the torus coils, and then a 'Region3' chamber located just
downstream of the magnetic field.

All chambers have the same, roughly equilateral triangular, shape.
The regions are isomorphic with Region2 approx. 1.5 times larger
in all dimensions than Region1, and Region3 is approximately 2 times larger than Region1
in all dimensions.
The chambers are planar, with the wires strung in layer with the first ``superlayer'' (of 6 layers)
tilted at a 6$^\circ$ stereo angle, and the second superlayer at -6$^\circ$ stereo
angle.  The design is discussed in more detail in the next section.

%%%%%%%%%%%%%%%%%%%%%% Figure : CLAS 3D Picture %%%%%%%%%%%%%%%%%%%%%%%%%%%%%%%
\begin{figure}[htpb]
\vspace{6.7cm} 
\caption{\small{A sketch of the torus magnet with drift chambers attached.
Note that cable runs and gas lines have been removed for clarity.  The largest
(R3) chambers are approximately equilateral triangular solids with 4 m long sides
and 0.8 m depth.}}
\label{clasview_3d}
\end{figure}
%%%%%%%%%%%%%%%%%%%%%%%%%%%%%%%%%%%%%%%%%%%%%%%%%%%%%%%%%%%%%%%%%%%%%%%%%%%%%%



\subsection{Physics Requirements for {\tt CLAS12} Forward Tracking}

There are several broad areas of physics research that drive 
the design of the forward tracking system: 
spectroscopic studies of excited baryons, investigations of 
the influence of nuclear matter on propagating quarks, studies of polarized 
and unpolarized quark distributions, and a comprehensive measurement of 
generalized parton distributions (GPDs).  Many of the reactions of interest 
are electroproduction of exclusive and semi-inclusive final states.  The 
cross sections for these processes are small, necessitating high-luminosity 
experiments.  A variety of simulated experiments rely on luminosities of 
10$^{35}$~cm$^{-2}$s$^{-1}$ to achieve the desired statistical accuracy in 
runs of a few months duration.  This is an order of magnitude increase
in beam flux compared to the previous CLAS detector.  
The new kinematic range to be explored is 
characterized not only by smaller cross sections, but also by more outgoing 
particles per event, with those particles being emitted with higher values 
of momentum and at smaller laboratory angles.  These basic physics criteria 
drive the design. 

\begin{itemize}
\item Physics constraints
\begin{itemize}
\item electron beam
\item exclusive reactions
\item wide kinematic coverage
\end{itemize}
\item Event types
\begin{itemize}
\item small cross-sections
\item multi-particle final states
\item large background rates
\end{itemize}
\end{itemize}


Exclusive reactions, in which an electron scattering event 
results in a meson-baryon final state, provide the most 
stringent requirements for the {\tt CLAS12} tracking system.  A final state 
of a few high-momentum, forward-going particles (the electron as well as one 
or more mesons), combined with a moderate-momentum baryon emitted at large 
angles, is the typical event type that determines the specifications of the 
tracking system.

In broad strokes, the forward detector must provide tracking for laboratory angles as 
small as 7$^\circ$ and as large as 40$^\circ$ in order to cover as much of 
the hadronic center-of-mass region as possible.  We require very good momentum 
and angular resolution for the scattered electron in order to determine the 
virtual photon flux factor, $\Gamma_v$, to an accuracy of a few percent.
Because the average particle momenta will be higher, the resolution of the 
tracking system must be better than the previous {\tt CLAS} values; the goal 
for the fractional momentum resolution is 0.3\% to  1\% at a track momentum 
of 10~GeV and an angle of 7$^\circ$.  Angular resolutions of about 1~mrad are required for the electron 
in order to know the virtual photon flux factor, and hence, the cross section, 
to a few percent. Finally, good vertex resolution will allow detection of 
secondary decay vertices and serve as a good marker for strangeness production.


















































