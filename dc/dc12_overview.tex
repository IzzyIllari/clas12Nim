\section{Forward Tracking System}
\label{overview}

Jefferson Lab recently upgraded its electron accelerator to provide beams with 
energies greater than 11 GeV/c.  To take advantage of new physics that would 
be possible at these higher energies, the CLAS detector was significantly
upgraded, and is now named the ``CLAS12'' detector.  This section describes
the tracking of charged particles in the forward part of the detector.
See Fig.~\ref{clas12-cut-view} in the CLAS12 overview paper for the relative placement
of the forward tracking system with respect to the whole CLAS12 detector.


The CLAS12 forward detector is constructed around a toroidal magnet consisting of six 
iron-free superconducting coils.  The particle detection system consists of drift 
chambers to determine charged-particle trajectories, {\v C}erenkov detectors 
for electron/pion separation, scintillation counters for flight-time 
measurements, and calorimeters to identify electrons and high-energy neutral 
particles.  An overview of the CLAS12 subsystems and geometry may be found in the 
CLAS12 Overview section.  A schematic view of the torus magnet with drift chambers
attached is shown in Fig.~\ref{chambers-and-torus}.   This assembly is referred to
as the ``forward tracker''. 

The forward tracking system consists of six sectors with three drift
chambers; located just before, inside, 
and just outside the torus field volume, and referred to as Regions~1, 2, 
and 3, (R1, R2 and R3) respectively.  Thus in each sector we have an R1 chamber located
in front of the torus, followed by an R2 chamber situated in the magnetic
field volume, between the torus coils, and then an R3 chamber located just
downstream of the magnetic field.

All chambers have the same, roughly equilateral triangular, shape, with
side lengths of approximately 2, 3 and 4 m for R1, R2 and R3, respectively.
The chambers are planar, with the wires strung in layer with the first 
``superlayer'' (of 6 layers)
tilted at a 6$^\circ$ stereo angle, and the second superlayer at -6$^\circ$ stereo
angle.  

The forward tracker can detect charged particles emerging from the target with
momenta greater than 200~MeV/c over a polar angular range from roughly 5$^{\circ}$ to 
40$^{\circ}$.  Because the coils of the torus magnet represent a 'dead area'
in which we cannot detect charged particles, we designed the chamber endplates
and attached electronics to be as thin as possible.  The resulting azimuthal
coverage varied from 50\% of 2$\pi$ at 5$^{\circ}$ to 80\% of 2$\pi$ at 40$^{\circ}$.
The design is discussed in more detail in the next section.

%%%%%%%%%%%%%%%%%%%%%% Figure : CLAS 3D Picture %%%%%%%%%%%%%%%%%%%%%%%%%%%%%%%
\begin{figure}[htpb]
\vspace{6.7cm} 
\caption{\small{A sketch of the torus magnet with drift chambers attached.
Note that cable runs and gas lines have been removed for clarity.  The largest
(R3) chambers are approximately equilateral triangular solids with 4 m long sides
and 0.8 m depth.}}
\label{chambers-and-torus}
\end{figure}
%%%%%%%%%%%%%%%%%%%%%%%%%%%%%%%%%%%%%%%%%%%%%%%%%%%%%%%%%%%%%%%%%%%%%%%%%%%%%%
















































