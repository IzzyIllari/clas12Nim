\section{Forward Tracking System}
\label{overview}

\hskip 0.15 in
Jefferson Lab recently upgraded its electron accelerator to provide beams with 
energies greater than 11 GeV/c.  To take advantage of new physics that would 
be possible at these higher energies, the CLAS detector was significantly
upgraded, and is now named the ``CLAS12'' detector.  This section describes
the tracking of charged particles in the forward part of the detector.

CLAS12 was designed to track charged particles emerging from the target with
momenta greater than 200~MeV/c over the polar angle range from 5$^{\circ}$ to 
140$^{\circ}$, while covering up to 80$\%$ of the azimuth.  Charged particles with 
angles from 5$^{\circ}$ to 40$^{\circ}$ enter the forward detector's drift chamber
system, while particles with angles greater than 40$^{\circ}$ enter the 
central detector's tracking system comprised of Silicon vertex detectors and
Micromegas-based tracking chambers.  These are covered in the section
on central detector charged particle tracking.

The CLAS12 forward detector is constructed around a toroidal magnet consisting of six 
iron-free superconducting coils.  The particle detection system consists of drift 
chambers to determine charged-particle trajectories, {\v C}erenkov detectors 
for electron/pion separation, scintillation counters for flight-time 
measurements, and calorimeters to identify electrons and high-energy neutral 
particles.  An overview of the CLAS12 subsystems and geometry may be found in the 
CLAS12 Overview section.  A schematic view of the torus magnet with drift chambers
attached is shown in Fig.~\ref{chambers-and-torus}.  


%%%%%%%%%%%%%%%%%%%%%% Figure : CLAS 3D Picture %%%%%%%%%%%%%%%%%%%%%%%%%%%%%%%
\begin{figure}[htpb]
\vspace{6.7cm} 
\caption{\small{A sketch of the torus magnet with drift chambers attached.
Note that cable runs and gas lines have been removed for clarity.  The largest
(R3) chambers are approximately equilateral triangular solids with 4 m long sides
and 0.8 m depth.}}
\label{clasview_3d}
\end{figure}
%%%%%%%%%%%%%%%%%%%%%%%%%%%%%%%%%%%%%%%%%%%%%%%%%%%%%%%%%%%%%%%%%%%%%%%%%%%%%%



\subsection{Physics Requirements for {\tt CLAS12} Forward Tracking}

There are several broad areas of physics research that drive 
the design of the forward tracking system: 
spectroscopic studies of excited baryons, investigations of 
the influence of nuclear matter on propagating quarks, studies of polarized 
and unpolarized quark distributions, and a comprehensive measurement of 
generalized parton distributions (GPDs).  Many of the reactions of interest 
are electroproduction of exclusive and semi-inclusive final states.  The 
cross sections for these processes are small, necessitating high-luminosity 
experiments.  A variety of simulated experiments rely on luminosities of 
10$^{35}$~cm$^{-2}$s$^{-1}$ to achieve the desired statistical accuracy in 
runs of a few months duration.  This is an order of magnitude increase
in beam flux compared to the previous CLAS detector.  
The new kinematic range to be explored is 
characterized not only by smaller cross sections, but also by more outgoing 
particles per event, with those particles being emitted with higher values 
of momentum and at smaller laboratory angles.  These basic physics criteria 
drive the design. 

\begin{itemize}
\item Physics constraints
\begin{itemize}
\item electron beam
\item exclusive reactions
\item wide kinematic coverage
\end{itemize}
\item Event types
\begin{itemize}
\item small cross-sections
\item multi-particle final states
\item large background rates
\end{itemize}
\end{itemize}


Exclusive reactions, in which an electron scattering event 
results in a meson-baryon final state, provide the most 
stringent requirements for the {\tt CLAS12} tracking system.  A final state 
of a few high-momentum, forward-going particles (the electron as well as one 
or more mesons), combined with a moderate-momentum baryon emitted at large 
angles, is the typical event type that determines the specifications of the 
tracking system.

The higher momentum and more forward angles of most hadrons leads us to 
split the design into a ``forward'' detector, which covers lab angles between 
5$^\circ$ and 40$^\circ$, and a ``central'' detector for hadrons with angles 
greater than 35$^\circ$.  The higher luminosity goal necessitates the use of 
a solenoidal magnet to shield the detector from M{\o}ller electrons.  To 
reduce interactions between this solenoidal field and the main {\tt CLAS12} 
toroidal field, and to facilitate construction and installation of new 
detector elements, the torus has been re-designed.  It is more compact than 
the previous {\tt CLAS} torus, while providing equivalent bending power.  This 
smaller torus design provides several advantages in the overall detector 
design: it decouples the design of the central solenoid and detector from 
that of the torus and forward tracking system, and it makes detector 
installation and removal easier.  

In broad strokes, the detector must provide tracking for laboratory angles as 
small as 5$^\circ$ and as large as 135$^\circ$ in order to cover as much of 
the hadronic center-of-mass region as possible.  We require very good momentum 
and angular resolution for the scattered electron in order to determine the 
virtual photon flux factor, $\Gamma_v$, to an accuracy of a few percent.
Because the average particle momenta will be higher, the resolution of the 
tracking system must be better than the previous {\tt CLAS} values; the goal 
for the fractional momentum resolution is 0.3\% to  1\% at a track momentum 
of 10~GeV and small angles of 5-7$^\circ$.  Angular resolutions of about 1~mrad are required for the electron 
in order to know the virtual photon flux factor, and hence, the cross section, 
to a few percent. Finally, good vertex resolution will allow detection of 
secondary decay vertices and serve as a good marker for strangeness production.


















































