\section{Drift Chamber System Conceptual Design}

The overall tracking requirements (0.5 -  1\% fractional momentum resolution 
at 5~GeV and efficient tracking at a luminosity of 
10$^{35}$~cm$^{-2}$s$^{-1}$) are the main drivers for drift chamber design.  

Because the {\tt CLAS} drift chamber system~\cite{dcnim} operated 
successfully for 8~years, we re-used many of the design concepts and 
most of the utility infrastructure.  For example, we re-used some
parts of the gas mixing and handling system, and also the high-voltage 
and low-voltage systems and many of the high voltage and signal cables. 

The forward tracking system consists of three regions divided into six
sectors as shown in Fig.~\ref{chambers-and-torus}; located just before, inside, 
and just outside the torus field volume, and referred to as Regions~1, 2, 
and 3, respectively.  

Each chamber has its wires arranged in two superlayers of
six layers each, with the wires in the two superlayers strung with 
$\pm$6$^\circ$ stereo angles, respectively.  The cell structure is 
hexagonal, that is, each sense wire is surrounded by six field wires.  This 
arrangement is similar to the previous {\tt CLAS} design and offers good 
resolution with very good pattern recognition properties.  Refer to our 
article on the previous {\tt CLAS} detector~\cite{clasnim} and our article 
on the previous drift chambers themselves~\cite{dcnim} for details of the 
previous detector and chambers.  

The groups responsible for the {\tt CLAS12} drift
chamber design and construction were Old Dominion University, Idaho State
University, and Jefferson Laboratory.

The major difference in the drift chambers for {\tt CLAS12} is that the 
cells cover a smaller solid angle than those in the previous {\tt CLAS} 
chambers, allowing efficient tracking at higher luminosities because the 
accidental occupancy from particles not associated with the event is smaller.  
Table~\ref{fwd-dc-design-parms} lists the main design parameters for each 
region of the {\tt CLAS12} drift chambers.  For the purposes of simulating 
track resolutions, we assumed that the position resolution of the individual 
drift cells would be 250~$\mu$m.  

%%%%%%%%%%%%%%%%%%%%%%%%%%%%%%%%%%%%%%%%%%%%%%%%%%%%%%%%%%%%%%%%%%%%%%%%%
\small{
\begin{table}[ht]
\begin{center}
\begin{tabular}{||c|c|c|c||} \hline \hline
            &{\bf Region 1}&{\bf Region 2}&{\bf Region 3}\\ \hline
Distance from target & 2.3 m    & 3.5 m        & 4.7 m    \\ \hline
Num. of superlayers  & 2        & 2            & 2        \\ \hline
Layers/superlayer    & 6        & 6            & 6        \\ \hline
Wires/layer          & 112      & 112          & 112      \\ \hline
Cell size            & 0.75 cm  & 1.18 cm      & 2.07 cm  \\ \hline
Active time window   & 150 ns   & 325 - 750 ns & 700 ns   \\ \hline
Drift distance resolution   & 250~$\mu$m   & 250~$\mu$m &250~$\mu$m  \\ \hline
Layer efficiency   & \> 95\%   & \> 95\% \> 95\%   \\ \hline
\end{tabular}
\caption{\small{Design parameters for the {\tt CLAS12} drift chambers.}}
\label{fwd-dc-design-parms}
\end{center}
\end{table}
}
%%%%%%%%%%%%%%%%%%%%%%%%%%%%%%%%%%%%%%%%%%%%%%%%%%%%%%%%%%%%%%%%%%%%%%%%%

\vskip 10 pt

%%%%%%%%%%%%%%%%%%%%%%%%%%%%%%%%%%%%%%%%%%%%%%%%%%%%%%%%%%%%%%%%%%%%%%%%%
\begin{table}[ht]
\begin{center}
\begin{tabular} {||c|c|c||} \hline \hline
{\bf Design Feature  }       &{\bf Advantages} &{\bf Disadvantages}\\ \hline
``All-wire design'' & Little cathode emission & \\ \hline
Small Cells & Robust track-finding  & Many wires to string \\ \hline
Hexagonal Cells & Minimum number of wires  & Angle-dependence of time to distance  \\ \hline
30-Micron diameter sense wire & Resistant to wire breakage & Higher operating voltage \\ \hline
80-Micron diameter field wire & Lower total wire tension & Higher fields on cathode wires \\ \hline
Shared, opposite voltage  & Identical fields & More HV \\
for sense and field & for all layers & channels required \\ \hline
Self-supporting design & Easier maintenance & 1 - 2 mm bowing of endplates \\ \hline
\end{tabular}
\caption{\small{Design features of the {\tt CLAS12} drift chambers.}}
\label{fwd-dc-design-features}
\end{center}
\end{table}
%%%%%%%%%%%%%%%%%%%%%%%%%%%%%%%%%%%%%%%%%%%%%%%%%%%%%%%%%%%%%%%%%%%%%%%%%

\subsection{Design Similarities and Differences from the previous CLAS Chambers}
The CLAS12 drift chambers share some design characteristics with the
previous CLAS chambers:
\begin{itemize}
\item ``brick-wall'' wire layout resulting in individual hexagonally-shaped
drift cells
\item sense wire layers are grouped into two ``superlayers'' of 6 layers each
\item the ``endplates'' on the two sides of the chamber are tilted 
at approximately 60 deg. with respect to each other
\item to maximize the active volume of the chamber, the ``dead areas'', e.g.
the endplates and electronics boards are kept as thin as possible
\item because of the possibility of large eddy currents and resultant
force on the endplates in case of the quench of our torus magnet, we
use non-conducting G10 endplates for our R2 chambers
\item The choice of gas; a 90:10 Argon:CO$_2$ mixture is the same
as for the previous chambers.  We plan to run with a gas gain of 
$5 \times 10^4$.
\end{itemize}

To improve the chambers' peformance we made some important changes to the design:
\begin{itemize}
\item all chambers have the same, roughly equilateral triangular, shape.
The regions are isomorphic with R2 approx. 1.5 times larger
in all dimensions than R1, and R3 is approx. 2 times larger than R1
\item all chambers are independent and self-supporting, allowing easy
maintenance and repair
\item all chambers are attached to the torus cryostat using 6 independent
rods with ``ball and socket'' ends, meaning that the chambers can be
moved out to maintenance position and moved back to the operating 
postion by turning one precision turn-buckle assembly
\item The previous CLAS chambers were interconnected to each other (R1) or 
connected directly to the endplates (R2).  In these chambes all of the wire tension
is borne by the endplates and thus they can be independently mounted.
The key to this improvement is the use of  
ultra-stiff endplate assemblies that obtain their stiffness 
by a flanged design.  
\item For the previous CLAS detector, the $\phi$ resolution times $\sin \theta$ is about two 
times larger than the $\theta$ resolution.  To have more equal resolution in 
the two angles, we doubled our stereo angle from 0 and 6$^\circ$ to 
$\pm$6$^\circ$.
\item all chambers are planar, with the first ``superlayer'' (of 6 layers)
tilted at 6$^\circ$ stereo angle, and the second superlayer at -6$^\circ$ stereo
angle
\item all wires within one superlayer are parallel to each other, thus
every cell in the superlayer is identical, making it easier to model
and fit the distance to time response
\item all chambers are strung with 30 micron gold-plated Tungsten wire,
considerably tougher and easier to handle than our previous choice 
of 20 micron wire
\item the choice of the cathode (``field'') wire is 80 micron gold-plated
Cu-Be wire, tougher and with better surface properties than our previous
choice of 140 micron gold-plated Aluminum wire
\item  Our choice of guard wire is 140-$\mu$m diameter, gold-plated
copper-beryllium.  These wires were strong enough that we pre-tensioned 
the chambers using only the guard wires; a simplification in the process.

\end{itemize}

One of the most significant design changes is the decision 
to use 30-$\mu$m diameter sense wire rather than the previously-used 
20~$\mu$m wire. 
This should make the chambers more resistant to wire 
breakage.  The larger radius of the sense wires means that higher 
voltages will be required to achieve the same gas gain, 
and the resulting higher electric field in the drift cells will result in 
a more nearly constant drift velocity that should be easier to calibrate.
Prototypes were built to study possible negative side-effects of the 
higher voltage operation, such as leakage currents on the circuit boards 
and/or higher rates of cathode emission from the field wire surfaces.

Fig.~\ref{garfield} shows GARFIELD calculations for a Region~3 drift cell
with both a 20-$\mu$m and a 30-$\mu$m diameter sense wire.  Here the
cells with the thicker sense wire will have a significantly higher drift 
velocity, which is desirable to reduce the time window, and hence the 
chamber occupancy.

%%%%%%%%%%%%%%%%%%%%%%%%%%%%%%%%%%%%%%%%%%%%%%%%%%%%%%%%%%%%%%%%%%%%%%%%%%%
\begin{figure}[ht]
\vspace{12.0cm}
\special{psfile=garfield1.eps hscale=30 vscale=27 hoffset=40 voffset=165}
\special{psfile=garfield2.eps hscale=30 vscale=27 hoffset=40 voffset=-5}
\special{psfile=garfield3.eps hscale=30 vscale=27 hoffset=230 voffset=165}
\special{psfile=garfield4.eps hscale=30 vscale=27 hoffset=230 voffset=-5}
\caption{\small{GARFIELD calculations of the electric field lines (top)
and drift time vs. drift distance (bottom) for a Region~3 drift cell.  The 
left plots show the configuration with a 20-$\mu$m diameter sense wire and 
the right plots show the configuration with a 30-$\mu$m diameter sense wire.
The high voltages were set to provide the same gas gain for each
configuration.}}
\label{garfield}
\end{figure}
%%%%%%%%%%%%%%%%%%%%%%%%%%%%%%%%%%%%%%%%%%%%%%%%%%%%%%%%%%%%%%%%%%%%%%%%%%%
 
