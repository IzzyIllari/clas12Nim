\subsection{Region One Construction (Special Considerations)}

\hskip 0.15 in 
The R1 chambers were designed and constructed through a collaboration 
of Idaho State University and Jefferson Lab.  These 
chambers are located about 2 m from the target, 
before particles enter the magnetic field of the torus,
and are thus key to good angular resolution. 

As we have seen from the generic assembly sketch of a chamber, the R1
chambers have a similar shape to the R2 and R3 chambers, differing in
scale and in some material choices.
Most notably, the endplates were constructed of aluminum with stainless
steel stiffener bars.

The main challenges in the R1 construction and design came about because
of the small wire spacing (8 mm between sense and field wires).  This
increased the electrostatic attraction of neighboring wires if they were
not perfectly and symmetrically placed, and it also made the physical act
of stringing the wires more difficult.

Wires with opposite voltage are electrostatically attracted.  If perfectly
placed in a symmetric array the forces would cancel each other. 
However, the sense wires might be slightly misplaced and so they would feel
a force which, if the tension were below a critical value, would increase
and pull them further out, further increasing the force, and so on until 
the wire begins to oscillate and then spark.  For our electric field configuration
this critical tension was about 2 g, a factor of 9 below our nominal
tension of 18 g.

