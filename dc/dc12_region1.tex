\subsection{Region One Construction}

\hskip 0.15 in 
The R1 chambers were designed and constructed through a collaboration 
of Idaho State University and Jefferson Lab.  These 
chambers are located before particles enter the magnetic field of the torus
and are thus key to good angular resolution.  They also are subject to the
highest level of background radiation. 

As we have seen from the generic assembly sketch of a chamber, the R1
chambers have a similar shape to the R2 and R3 chambers, differing in
scale and in some material choices.
Most notably, the endplates were constructed of aluminum with stainless
steel stiffener bars.
Fig.~\ref{region1-endplate} shows and assembly drawing of an R1 endplate.


%%%%%%%%%%%%%%%%%%%%%% Figure : Region One %%%%%%%%%%%%%%%%%%%%%%%%%%%%%%%
\begin{figure}[htpb]   
\vspace{7.0cm}

\caption{\small{Assembly drawing of a R1 endplate.}}
\label{region1}
\end{figure}   
%%%%%%%%%%%%%%%%%%%%%%%%%%%%%%%%%%%%%%%%%%%%%%%%%%%%%%%%%%%%%%%%%%%%%%%%%%

The main challenges in the R1 construction and design came about because
of the small wire spacing (~ 8mm between sense and field wires).  This
increased the electrostatic attraction of neighboring wires if they are
not perfectly and symmetrically placed, and it also made the physical act
of stringing the wires more difficult.

Wires with opposite voltage are electrostatically attracted.  If perfectly
placed in a symmetric array the forces would cancel out, since each sense
wire is surrounded by six field wires.  This is true for the sense wires
even at the boundary layers because we grade the guard wire voltage so
that the electric field are that same as they would be for an infinite grid.

However, the sense wires might be slightly misplaced and so they would feel
a force which, if the tension were below a critical value, would increase
and pull them further out, further increasing the force, and so on until 
the wire begins to oscillate and then spark.  For our electric field configuration
this critical tension was about 2 grams, a factor of 9 below our nominal
tension of 18 grams.

The challenges to stringing the chambers was overcome by careful attention to
detail ...


