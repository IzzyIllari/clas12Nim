\section{Chamber Operation and Performance Monitoring}

\subsection{Choice of Gas}
\hskip 0.15in

The main requirements for the chamber gas were that it have reasonably low 
multiple scattering, allow for reasonable gas gains, have short collection 
times in order to reduce the random background expected from M{\o}ller 
electrons and target-generated X-rays, and be inexpensive because of the 
large volume of the chambers. Also, safety considerations motivate the use of
a non-flammable gas mixture.  Additional concerns about small gas 
leaks and the proximity of many photomultiplier tubes argued against helium 
mixtures.  Ultimately a 90$\%$ argon - 10$\%$ CO$_2$ mixture was employed 
for several reasons: the gas has a fairly high saturated drift velocity 
($>$ 5~cm/$\mu$s), and it has an operating voltage plateau of several hundred 
volts before breakdown occurs.  The 90$\%$/10$\%$ mixture 
provides good efficiency, adequate resolution, and reasonable collection times.

\subsection{Wire and Gas Choice}

\hskip 0.15in
To limit wire tensions and operating voltages meant that the wire diameters 
had to be minimized.  The sense wire for all chambers, supplied by the Luma
Sweden Company, consists of 20-$\mu$m diameter gold-plated tungsten, the 
smallest practical choice.  Tungsten was chosen because of its durability, 
and the gold-plating of the wires, amounting to a thickness of 0.127~$\mu$m, 
ensures chemical inertness as well as a smooth surface finish.  The expected 
electronics gain and thresholds dictated that the gas gain be a few times 
10$^4$.  Under this condition, the electric field at the surface of the sense 
wires is $\approx$280~kV/cm.  The field-wire diameter was chosen to ensure 
that the electric field at their surface remained below 20~kV/cm to minimize 
conditions causing cathode deposits~\cite{kadyk,cathode}.  Since there are 
roughly twice as many field as sense wires, this required that the diameter 
of the field wires be about seven times the diameter of the sense wire.  The 
field wire for all chambers consists of 140-$\mu$m diameter gold-plated 
aluminum 5056 alloy from the California Fine Wire Company.  Aluminum was 
chosen because it has the longest radiation length of any practical wire 
material and thus minimizes multiple scattering.  Additionally, the low density 
of aluminum means that the field wires could be strung at lower tension than a 
more dense wire, with minimal gravitational sag.  This minimizes the forces on 
the endplates.

The main requirements for the chamber gas were that it have reasonably low 
multiple scattering, allow for reasonable gas gains, have short collection 
times in order to reduce the random background expected from M{\o}ller 
electrons and target-generated X-rays, and be inexpensive because of the 
large volume of the chambers.  The initial choice of chamber gas was a
50$\%$ argon - 50$\%$ ethane mixture.  However, safety concerns prompted a 
search for a non-flammable gas mixture.  Additional concerns about small gas 
leaks and the proximity of many photomultiplier tubes argued against helium 
mixtures.  Ultimately a 90$\%$ argon - 10$\%$ CO$_2$ mixture was employed 
for several reasons: the gas has a fairly high saturated drift velocity 
($>$ 4~cm/$\mu$s), and it has an operating voltage plateau of several hundred 
volts before breakdown occurs.  A 95$\%$/5$\%$ mixture achieves velocity 
saturation at even lower values of the electric field but has an excessively 
short voltage plateau due to lack of quenching.  The 90$\%$/10$\%$ mixture 
provides good efficiency, adequate resolution, and reasonable collection times.



\subsection{Selecting the Proper Operating Voltage}
\hskip 0.15in



References: 'A study of electron drift velocity in Ar-CO2 and AR-CO2-CF4 gas
mixtures', NIM A340 (1994) p.485-490

CLAS-Note 2009-27 'Calculation of Drift Chamber Gas Gain' Mestayer
