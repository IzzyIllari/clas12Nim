\section{Chamber Operation and Performance Monitoring}

\subsection{Choice of Gas}

The main requirements for the chamber gas were that it have reasonably low 
multiple scattering, allow for reasonable gas gains, have short collection 
times in order to reduce the random background expected from M{\o}ller 
electrons and target-generated X-rays, and be inexpensive because of the 
large volume of the chambers. Also, safety considerations motivate the use of
a non-flammable gas mixture.  Additional concerns about small gas 
leaks and the proximity of many photomultiplier tubes argued against helium 
mixtures.  Ultimately a 90$\%$ argon - 10$\%$ CO$_2$ mixture was employed 
for several reasons: the gas has a fairly high saturated drift velocity 
($>$ 5~cm/$\mu$s), and it has an operating voltage plateau of several hundred 
volts before breakdown occurs.  The 90$\%$/10$\%$ mixture 
provides good efficiency, adequate resolution, and reasonable collection times.




\subsection{Selecting the Proper Operating Voltage}

In this section we discuss our operating voltages and how it was determined.
First, we discuss how we divide the total voltage between our sense, field 
and guard wires in order to mimic a cell layout with an infinite number of
layers, to achieve a situation in which all wires, regardless of layer, have
the same gain.  Then we discuss our choice of the total sense to field
wire difference in voltage; including the resulting gas gain and efficiency.


\subsubsection{Dividing the Total Voltage between Sense, Field and Guard Wires}
We ran our chambers with a mixed voltage scheme:
positive high voltage on the sense wires, negative voltage on the
field wires and positive voltage on the guard wires.
This mixed-voltage scheme has a couple of advantgages over a scheme in
which the field wires, for example, are held at ground potential:
\begin{itemize}
\item fewer field lines run from the sense wire to the endplate which
is grounded.  This reduces the likelihood of producing a `Malter effect'
(Ref.~\cite{malter}) in which an accidental source of cathode emission
(due to an insulating contaminant on the endplate, for example) causes
a self-sustaining discharge
\item the sense to ground potential and the field to ground potentials 
are smaller; decreasing surface electric fields on the on-chamber
circuit boards
\end{itemize}

In addition, by carefully selecting the values of the sense, field and
guard wire voltages we can create potential distributions which mimic
an infinite grid of cells, where the gain on any wire is the same as
any other, regardless of whether is the first, last or middle layer.

This optimum condition is reached when the sense voltage is twice the
field voltage (and opposite sign).  This is because we have twice as many
field wires as sense wires and all field lines which originate on a
field wire land on a sense wire.  

The guard wire voltage was then chosen so that the total charge on all wires is zero.  
If we have a nearby ground plane due to the metallized gas bag then it will
have no net effect on the nearby wire planes if there is no net flux of
electric field through the bag which is the case if the enclosed net charge
is zero.
So, the ratio of voltages from Sense to Field to Guard wires is 1 : -1/2 : 5/14.




\subsection{Determining the Operating Values of the Discriminator Thresholds and High Voltage Settings}

We set the discriminator levels in the DCRB's to reduce the accidental hit rate (with no beam) due to electronic
noise to be less than about 1\%.  Since the electronic noise was generally proportional to wire length, we had less
electronic noise on the smaller R1 chambers.  Using this criteria, we set the thresholds to 30, 45 and 45 mV, respectively,
for R1, R2 and R3.  
Once we set the discriminator thresholds, we performed a High Voltage efficiency scan.  We raised the high voltage in
steps of 75 Volts and analysed the data.  We set the operating value for the high voltage at the point at which
the layer efficiency (the probability that a track passing through a layer will fire at least one wire) equaled or exceeded 97\%.

We determined the layer efficiency using the `excluded layer' method.  In one superlayer (of six layers) we found track
segments by our usual fitting method, but ignoring the data from a pre-selected layer (layer 3, for example).  We then
projected the track segment through that layer and determined whether or not the indicated wire (or an adjacent one) had a good hit.
Fig.~\ref{effcy-vs-voltage} shows the `plateau curve' of efficiency plotted versus voltage for one typical chamber 
and superlayer.  The layer efficiency at the chosen operating high voltage poiont was between 97\% and 98\%.

%%%%%%%%%%%%%%%%%%%%%%%%%%%%%%%%%%%%%%%%%%%%%%%%%%%%%%%%%%%%%%%%%%%%%%%%%%%
\begin{figure}[htbp]
\vspace{5cm}
\begin{picture}(50,50)
\put(-10,10)
{\hbox{\includegraphics[width=0.35\textwidth,natwidth=610,natheight=642]{img/trace-routing-schematic.jpg}}}
\end{picture}
\caption{\small{Layer efficiency plotted versus high voltage for a typical superlayer.}}
\label{effcy-vs-voltage}
\end{figure}
%%%%%%%%%%%%%%%%%%%%%%%%%%%%%%%%%%%%%%%%%%%%%%%%%%%%%%%%%%%%%%%%%%%%%%%%%%%



\subsubsection{Operating Voltage, Gas Gain and Layer Efficiency}
The gas gain varies exponentially with the total sense to field wire voltage
difference, with a doubling voltage of about 100, 110 or 120V respectively, for
R1, R2 and R3.  During our Fall 2019 run, we ran with sense - field wire voltage
differences of 2100, 2325 and 2475 V, respectively for R1, R2 and R3
We calculate that our total gas gain is approximately 2.7~$10^4$, 3.7~$10^4$, and 4.4~$10^4$,  
respectively, for R1, R2 and R3.

The single layer inefficiency is not uniform across the drift cell.  It is higher near the sense wire and also near the outer
edge of the cell.  A track passing close to a sense wire leave many ions in the cell, but the ion arrival times are stretched
out from near-zero to the maximum drift time, Tmax.  The result is that the pre-amplifier's output signal has a low voltage
amplitude but persists for a long time.  So, even though the collected charge is large the voltage put out by our trans-impedance
pre-amplifiers may not be large enough to exceed the voltage discriminator threshold of the DCRB,
For the case of tracks near the outer edge of the cell (so-called `corner-clippers') they simply leave a very small number
of ions in the cell and thus have a small signal.

In Fig.~\ref{effcy-vs-doca} we plot the layer efficiency as a function of the distance-of-closest-approach (DOCA) and
you can see the characteristic rise of the ineffiency near to and far from the sense wire.
%%%%%%%%%%%%%%%%%%%%%%%%%%%%%%%%%%%%%%%%%%%%%%%%%%%%%%%%%%%%%%%%%%%%%%%%%%%
\begin{figure}[htbp]
\vspace{5cm}
\begin{picture}(50,50)
\put(-10,10)
{\hbox{\includegraphics[width=0.35\textwidth,natwidth=610,natheight=642]{img/trace-routing-schematic.jpg}}}
\end{picture}
\caption{\small{Layer efficiency plotted versus DOCA.}}
\label{effcy-vs-doca}
\end{figure}
%%%%%%%%%%%%%%%%%%%%%%%%%%%%%%%%%%%%%%%%%%%%%%%%%%%%%%%%%%%%%%%%%%%%%%%%%%%

References: `A study of electron drift velocity in Ar-CO2 and AR-CO2-CF4 gas
mixtures', NIM A340 (1994) p.485-490

CLAS-Note 2009-27 1Calculation of Drift Chamber Gas Gain' Mestayer
