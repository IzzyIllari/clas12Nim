\begin{abstract}

Experimental Hall B at Jefferson Laboratory houses the CEBAF Large
Acceptance Spectrometer, the magnetic field of which is produced by a
superconducting toroid.  The six coils of this toroid divide the detector
azimuthally into six sectors.  Each sector contains three multi-layer 
drift chambers for reconstrucing the trajectories of charged particles
emanating from a fixed target.  Each of the 18 chambers has a total of 1344
individually instrumented drift cells.  The wires are arranged in planar
layers, and each chamber has two `superlayers' of six layers each. 
The six-layer structure allows for redundancy in track segment finding
and facilitates good tracking efficiency even in the presence of some
individual wire inefficiency.  The design, construction, operation and
calibration methods are described and estimates of efficiency and resolution
are presented from in-beam measurements.

\end{abstract}




