\begin{abstract}

The CEBAF Large Acceptance Spectrometer (CLAS12) is a magnetic
spectrometer located in Hall B at Jefferson Lab.
It is built around a
superconducting toroidal magnet.
The six coils of this toroid divide the detector
azimuthally into six sectors.  Each sector contains three multi-layer 
drift chambers for reconstrucing the trajectories of charged particles
originating from a fixed target.  

Each of the 18 chambers contains 1344
individually instrumented drift cells arranged in planar
layers, and each chamber has two `superlayers' of six layers each. 
The six-layer structure allows for redundancy in track segment finding
and facilitates good tracking efficiency even in the presence of some
individual wire inefficiency.  The design, construction, operation and
calibration methods are described, and estimates of efficiency and resolution
are presented from in-beam measurements.
\end{abstract}




