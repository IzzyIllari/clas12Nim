\begin{abstract}

Experimental Hall B at Jefferson Laboratory houses the CEBAF Large
Acceptance Spectrometer, the magnetic field of which is produced by a
superconducting toroid.  The six coils of this toroid divide the detector
azimuthally into six sectors, each of which contains three large multi-layer 
drift chambers for tracking charged particles produced from a fixed target on 
the toroidal axis.  Within the 18 drift chambers are a total of 24,192 
individually instrumented hexagonal drift cells.  The simple planar 
geometry of these chambers provides for good tracking resolution and 
efficiency, along with large acceptance. The design and construction 
challenges posed by these large-scale detectors are described, and detailed 
results are presented from in-beam measurements.

\end{abstract}




