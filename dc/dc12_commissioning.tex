\section{Pre-Commissioning and Installation}

In this section we describe our procedures which took
place after chamber stringing was complete to get the
chambers ready for installation and to install them 
in the experiment.

\begin{enumerate}
\item install electronics boards and ``on-chamber'' cables
and gradually bring the chambers to full operating voltage 
\item install and survey the chambers
\end{enumerate}

\subsection{Electronics Installation and Turn-on}
After the chambers were strung and went through a mechanical quality
check to insure that all wires are intact and properly tensioned, we
installed the on-chamber electronics boards.


After stringing was complete we did the following:
\begin{enumerate}
\item ``daisy-chained'' the field wire crimp pins so that a single
HV cable could power two rows of field wires (32 wires)
\item physically positioned the boards so that their plated through
holes aligned directly above the sense wire crimp pins, and attached
the boards to the chamber with screws, and
\item electrically connected each sense wire crimp pin to each
plated through hole using a conductive rubber 'sleeve' which fit
over the crimp pin and also contacted the plated through hole on
its outer radius.
\end{enumerate}

Now the chamber was ready for ``burn-in'' and ``pre-testing''.

\subsection{``Burn-in and Pre-testing''}
When drift chambers are first turned on, they typically draw fairly high
`dark' currents, even at low voltages.  The standard procedure is to
slowly raise the high voltage, wait for a certain time period during
which the current subsides and raise the voltage again, and so on.
For our chambers, the typical time period was an hour and the typical
voltage step was 75 V which is approximately the `doubling voltage' of
our chambers (the voltage step which increases the gain by a factor
of two).

\subsection{Installation and Survey}

The chambers are attached by ball and socket joints to rods which are attached
on the other end by ball and socket to the toroidal magnet frame
After an initial installation the chambers were moved to an approximate
working location.  Then, with the survey crew's information, the chamber location
was fine tuned by lengthening or shortening the rods with fine-pitch screw adjustments.
In this way the initial installation was performed with sub-millimeter accuracies as
determined by the survey group's laser positioning system.

The installation of 18 chambers took months to accomplish and the survey crew's work
was hindered at times by obscured views of some of the fiducial marks on the 
chambers.  In fact, we checked and updated the survey information with a later 
``straight-track'' alignment run and analysis procedure (see the Calibration section).
Although most of alignment numbers were verified to sub-millimeter accuracy, there
were a few parameters which were off by as much as 2 millimeters.

Of particular note regarding the ``rod and ball and socket'' mounting scheme is that
we showed that it is reproducible to sub-millimeter accuracy, thus reducing the time
and manpower required for maintenace and repair.  In a matter of hours, a chamber 
can be moved to ``maintenance position'', repaired, and moved back to installation
position without the need for a survey.
