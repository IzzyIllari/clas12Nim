\section{Pre-Installation Commissioning}

In this section we describe our procedures which took
place after chamber stringing was complete to get the
chambers ready for installation in the experiment.

\begin{enumerate}
\item install electronics boards and ``on-chamber'' cables
and gradually bring the chambers to full operating voltage 
\item install and survey the chambers
\item perform efficiency scans to determine the HV and discriminator operational settings
\end{enumerate}

\subsection{Electronics Installation and Turn-on}
After the chambers were strung and went through a mechanical quality
check to insure that all wires are intact and properly tensioned, we
installed the on-chamber electronics boards.


After stringing was complete we did the following:
\begin{enumerate}
\item ``daisy-chained'' the field wire crimp pins so that a single
HV cable could power two rows of field wires (32 wires)
\item physically positioned the boards so that their plated through
holes aligned directly above the sense wire crimp pins, and attached
the boards to the chamber with screws, and
\item electrically connected each sense wire crimp pin to each
plated through hole using a conductive rubber 'sleeve' which fit
over the crimp pin and also contacted the plated through hole on
its outer radius.
\end{enumerate}

Now the chamber was ready for ``burn-in'' and ``pre-testing''.

\subsection{``Burn-in and Pre-testing''}
When drift chambers are first turned on, they typically draw fairly high
`dark' currents, even at low voltages.  The standard procedure is to
slowly raise the high voltage, wait for a certain time period during
which the current subsides and raise the voltage again, and so on.
For our chambers, the typical time period was an hour and the typical
voltage step was 75 V which is approximately the `doubling voltage' of
our chambers (the voltage step which increases the gain by a factor
of two).

\subsection{Installation and Survey}
- chambers attached by ball and socket joints to rods which are attached
on the other end by ball and socket to the toroidal magnet frame

