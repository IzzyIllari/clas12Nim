\section{Commissioning}

In this section we describe our procedures to do the following:

\begin{enumerate}
\item install electronics boards and ``on-chamber'' cables
\item gradually bring the chambers to full operating voltage (``burn-in'') 
and to test for any other problems
\item install and survey the chambers
\item diagnose and fix any early failures, and
\item perform efficiency scans over a range of HV and discriminator settings
\end{enumerate}

\subsection{Electronics Installation and Turn-on}
After the chambers were strung and went through a mechanical quality
check to insure that all wires are intact and properly tensioned, we
installed the on-chamber electronics boards.

There are two types of board: one is a High-Voltage Distribution Board (HVTB)
which brings RC-filtered high voltage to the wires.  Because there are three
types of wire: 'sense', 'field' and 'guard' we supply three different voltages
to each board.  As discussed in the design section, we have a mixed high
voltage system: the sense (or anode) wires are operated at positive voltage,
typically about 2000 Volts; the field (or cathode) wires are operated at
about -1000 V and the guard wires at about +700 V.  This creates an
electric field which converges radially onto the sense wire with surface
field values of about 200 kV/cm.  Similarly each field wire has an
electric field directed radially outward from the surface with a strength
of about 50 kV/cm.  The surface field for the field wires is weaker 
because there are twice as many field wires as sense wires and their
diameter is larger (80 microns compared to 30).  The guard wires
close out the structure and their voltage is chosen to approximate an
infinite grid of wires; thus every sense and field wire in a superlayer
has the same field configuration unless there is an irregularity (like
a missing, broken wire) nearby.

On the other side of the chamber are our Signal Translator Boards (STB) 
which support an individual Single Inline Package (SIP) transimpedance
pre-amplifier for each sense wire.  This pre-amplifier takes the
small current pulse (microAmps) and translates it into a voltage 
pulse with a transimpedance of 2 mV/microAmp.  The signals (typically
10's to 100's of mV and 10's to 100's of nanosecond duration) are
transmitted down twisted-pair conducting wire to our downstream
Drift Chamber Readout Boards (DCRB) which further amplifies and
discriminates the voltage pulse and then converts the leading edge
to a digital time signal.
See figures (STB, HVTB) in the electroncis section for more details.

At this stage, our technicians did three important tasks:
\begin{enumerate}
\item ``daisy-chained'' the field wire crimp pins so that a single
HV cable could power two rows of field wires (32 wires)
\item physically positioned the boards so that their plated through
holes aligned directly above the sense wire crimp pins, and attached
the boards to the chamber with screws, and
\item electrically connected each sense wire crimp pin to each
plated through hole using a conductive rubber 'sleeve' which fit
over the crimp pin and also contacted the plated through hole on
its outer radius.
\end{enumerate}

Now the chamber was ready for ``burn-in'' and ``pre-testing''.

\subsection{``Burn-in and Pre-testing''}
- work flow: raise by a step, wait until current subsides, raise by
a step ... and repeat


\subsection{Installation and Survey}
- chambers attached by ball and socket joints to rods which are attached
on the other end by ball and socket to the toroidal magnet frame

\subsection{Early Failures and Repair}

\subsection{Efficiency Scans of HV and Discriminator Thresholds}
