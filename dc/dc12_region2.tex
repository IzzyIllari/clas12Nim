
\subsection{Region Two Construction}

\hskip 0.15 in 
The R2 chambers, which were designed and constructed by Old Dominion University 
with assistance from Jefferson Laboratory, are the middle of the three  
drift-chamber packages.  They track all charged particles in the magnetic field 
of the torus near the point of maximum sagitta.  The six identical R2 sectors 
are approximately equilateral triangular boxes with 3 m sides. 
They are located at a radius of $\approx$3~m from the nominal target location.  
  
The R2 chambers were designed with ultra-thin endplates which were thin enough
to be wholly within the ``shadow'' cast by the torus cryostat; in other words,
the full length of the wires is in the active fiducial volume of CLAS12. 
All chamber support hardware and electronics had to fit 
entirely within this shadow region.

Each R2 sector contains a total of 1344 hexagonal drift cells.  The drift cell
diameter of the first six layers (first ``superlayer'') was 1.8 cm while that
of the second was 1.9 cm.   The 
wires are arranged in two superlayers, one rotated by +6$^/circ$ in the wire plane 
(at smaller radius) and the second superlaye by -6$^/circ$ (at larger radius).  


Several design constraints were peculiar to R2.  These chambers have to operate 
in a magnetic field up to 2~T, and the chambers have to withstand any rapid 
changes in magnetic field, such as what might occur due to a magnet quench.
The R2 endplates are constructed from 2-cm thick Stesalit 4411W, a disordered 
epoxy-fiberglass composite commonly used in wire-chamber construction
\cite{stesalit}, and known not to cause aging problems~\cite{stesalitaging}.  
Using a nonconducting material eliminates any possible forces on the endplate 
due to eddy currents produced during a magnetic-field quench.  It also allows 
the trumpets that position the wires to be essentially flush with the endplates, 
rather than having to insulate the trumpets from the conducting endplates as in 
the other two Regions (see Fig.~\ref{r2_inserts}).  This reduced the thickness of 
the inactive region by 1 to 2~cm.

%%%%%%%%%%%%%%%%%%%%%% Figure : Region 2 inserts %%%%%%%%%%%%%%%%%%%%%%%%%%
\begin{figure}[htpb]   
\vspace{6.0cm}
\special{psfile=r2_inserts.eps hscale=70 vscale=70 hoffset=-93 voffset=-200}  
\caption{\small{Schematic cross-sectional view of the R2 endplate 
wire-positioning hardware.}} 
\label{r2_inserts}
\end{figure}   
%%%%%%%%%%%%%%%%%%%%%%%%%%%%%%%%%%%%%%%%%%%%%%%%%%%%%%%%%%%%%%%%%%%%%%%%%%%




 
