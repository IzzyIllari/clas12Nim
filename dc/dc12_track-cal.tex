\section{Drift Chamber Calibration Procedures}
In this section, we discuss the calibration procedures to obtain the 
best possible spatial resolution for reconstructing the trajectory of
a forward-going charged particle.  We need to know three things to 
accurately reconstruct a trajectory: the location of the wires, the
distance of closest approach (DOCA) of the track to the wire and the
value of the magnetic field traversed.  These were determined by
\begin{itemize}
\item alignment procedures
\item time to distance calibration
\item magnetic field mapping and modelling
\end{itemize}.  We discuss all three after a brief overview of the track reconstruction.

\subsection{Track Reconstruction Overview}

\hskip 0.15 in
The reconstruction of charged-particle tracks is performed in several stages.  In 
the first stage, individual tracks are fit only to hit-wire positions in a 
procedure known as ``hit-based'' tracking.  Hit-based tracking proceeds
in the following steps:
\begin{itemize}
\item in each superlayer, ``clusters'' of hits which are consistent with
being part of a track are identified
\item a ``noise rejection algorithm'' is then applied to the clusters, 
with one of the more efficient algorithms being the removal of the
interior hits from horizontal ``strings'' of hits along a layer.
\item the resulting trimmed clusters are then fit to a straight-line hypothesis,
and those hits with acceptable residuals are kept and identified collectively
as a ``track segment''.  This fit uses the wire position as the assumed hit
position and is called ``hit-based tracking''.
\item because the ``hits'' in a track segment are 2-dimensional objects (we
do not know their location along the wire direction) a track segment is not
a line but a plane.  Thus pairs of segments in neighboring superlayers within
one chamber (with superlayers of +/- $6^\circ$ stereo angle) represent the
intersection of two planes; that is a line.  This line's coordinates 
are evaluated midway between the two superlayers, and is a 6-dimensional
object (x,y,z and 3 angles) which we call a ``cross''.
\item in the first pattern-recognition step to find a track candidate,
the positions of 3 crosses (1 each in R1, R2 and R3) are fit to a
parabolic functional form to give us a ``track candidate''.
\item the track candidate is the initial ``state vector'' of our
Kalman filter tracking program
\end{itemize} 


Due to the comparatively small size of the drift cells and the large 
number of wire layers, the track momenta can already be reconstructed 
at the ``hit-based'' level with a 
resolution of 3$\%$ to 5$\%$.  Additional information on these tracks, derived
from the {\v C}erenkov, time-of-flight, and electromagnetic calorimeter 
detectors, allows for determination of the identities and speeds of the 
charged particles.  In the second stage of the analysis, flight-time 
information of the particles from the target to the outer scintillators is 
used to correct the measured drift times.  A pre-determined table is then used
to convert the corrected drift times to drift distances (see section~\ref{tdistcal} 
for details of the function). These corrected 
track positions in each drift cell are input data to our Kalman filter
fit to determine the final track parameters.  This is colloquially referred
to as ``time-based tracking''.

\subsection{Alignment Procedures}
\label{align}

\hskip 0.15in
As described in section~\ref{survey}, each of the 18 drift chambers was 
surveyed after installation into CLAS.  The survey procedures had sub-millimeter
accuracy, but we wanted an independent check of the chambers' positions.  For 
this reason, the survey values for the chamber geometry were viewed only as 
a reasonable starting point to be refined by comparisons with data.

To adjust the chamber geometry parameters to improve the tracking resolution,
``straight-track'' data with the torus magnetic field off were analyzed.  
Tracks were found and fitted with our standard track reconstruction package.
For various bins in the angle of the track, we measured the shifts of the
track residual means as a function of layer number. 
Before alignment, the data showed significant displacements of the means
from zero.  

Our procedure for aligment was straight-forward.  On a first pass through
the data we used misaligment parameters (shifts and rotations of individual
chambers) set to zero.  On subsequent passes, we deliberately misaligned
a particular chamber by a particular offset in position or angle and 
produced a second set of plots of residual mean vs. layer.  We ran 18 passes
through the data, adjusting all combinations of region (1, 2, 3) and
of offset type $\delta$x, $\delta$ y, $\delta$ z, $\theta$ x, 
$\theta$ y, $\theta$ z, one at
a time.  The offsets in $x, y$ and $z$ were 2 mm, and the angular rotations
were 0.2 degrees.  We called these ``unit distortions''.

We then subtracted the pass1 residual distribution from a pass``i'' distribution
to give a ``change of residual'' distribution caused by a given ``unit distortion''.
We then fit the observed residual distribution from the data to a weighted
sum of the 18 ``change of residual'' distributions.  In principle, we
could have had 18 free parameters, but in practice we had 12 free parameters:
 $\delta$ x, $\delta$ y, $\delta$ z, and $\theta$ y for each of the 3 chambers: R1, R2 and R3,
where $\theta y$ is a tilt of a chamber.  The yaw ($\theta x$) and roll ($\theta z$)
were not varied because they did not improve the fits.


The 
results of this procedure indicated that the best-fit position of the chambers 
along the three coordinate axes varied by up to several millimeters relative 
to the surveyed positions.  

\subsection{Time to Distance Calibration}
The drift chambers' TDC's measure time.  This time is corrected for a number
of effects, and this corrected time is converted to a distance, DOCA, by 
a pre-calculated time to distance function.  In this subsection we 
explain the time corrections, the function used to calculate time as a 
function of DOCA and how we calibrate the parameters of this funciton.

\subsubsection{Time Corrections}
The drift time is the elapsed time between the time that the particle 
traversed the wire cell and the time that the released gas ions (electrons)
reached the sense wire.

The drift time is given by the following expression:
\begin{equation} 
\label{drift}
t_{drift} = t_{tdc} - t_{start} - t_{0} - t_{flight} - t_{prop} - t_{walk},
\end{equation}

\noindent
where $t_{TDC}$ is the raw time measured by the TDC, $t_{start}$ is the event start time, 
$t_0$ is the fixed-time (cable) delay for the wire, $t_{flight}$ is the 
flight time of the particle from the interaction vertex to the wire, $t_{prop}$ 
is the signal propagation time along the wire, and $t_{walk}$ is a time-walk 
correction made for short drift times due to different ionizations for slow 
and fast particles.  

With a trigger based on detecting an electron in the CLAS12 detector, the event start time is 
given by the time-of-flight counter time for the primary scattered electron 
corrected for the calculated flight time of this electron from the beam-target vertex.

Through the use of an appropriate function, the drift time determines the 
distance-of-closest-approach (DOCA) of the charged-particle track to the sense 
wire.  However, there remains an ambiguity regarding which side of the sense 
wire the track passed by.  This ``left-right ambiguity'' is resolved within 
the individual superlayers by comparing the $\chi^2$ values for the track fit 
for all different combinations of drift-distance signs.  After selection of 
the full set of drift-distance signs within each superlayer, a final fit 
results in improved track parameters.

\subsubsection{T0 Determination}
As indicated in eq(\ref{drift}), the fixed-time delays (mainly fixed cable delays) 
for each wire must be known in order to determine the drift times.   To determine
this t$_0$ value, we produced a histogram of the following quantity for all hits
used on tracks:$ ( t_{tdc} - t_{start} - t_{flight} - t_{prop} - t_{walk} )$.
This produced a characteristic plot of a drift chamber signal on a flat
background from out-of-time tracks.  Our 24,192 drift chamber signals are carried
on individually made multi-conductor cables which each carred 16 signals.  We thus
produced and analysed 24,192/16 = 1512 histograms to determine that many values of
t$_0$. We occasionally re-do this analysis whenever we have a new trigger condition
or new configuration of readout boards.


\subsubsection{Time-to-Distance Functional Parameterization}
\label{tdistcal}

\hskip 0.15in
Each hit on a track is characterized by two variables, the measured drift 
time from the sense wire and the distance-of-closest-approach (DOCA) to the 
sense wire resulting from the track fit.  
A best fit to the dependence of DOCA on time defines the 
drift-velocity function of the drift cells. However, several factors 
complicate this analysis. For example, the DOCAs obtained from the fitted 
tracks are biased quantities since an initial estimate of the drift-velocity 
function is used in the track determination.  Moreover, the drift cells are 
not circular, as the analysis implicitly assumes, but are hexagonal, leading 
to angle-dependent corrections.   The R2 chambers, in particular, are in a 
region of high and spatially varying magnetic field.  Finally, the different 
ionization densities of the tracks from particles with different velocities 
leads to substantial time-walk corrections for tracks near the wire.  Each of 
these points is briefly discussed in this section.


%%%%%%%%%%%%%%%%%%%%%% Figure : Garfield Picture %%%%%%%%%%%%%%%%%%%%%%%%%%
\begin{figure}[htpb]
\vspace{4.5cm} 
\special{psfile=garfield.eps hscale=49 vscale=49 hoffset=-15 voffset=-130}
\caption{\small{Plot of electric-field lines and equal-time isochrone contours
(100 ns interval) for a 90$\%$ argon - 10$\%$ CO$_2$ gas mixture for (a) an R3
drift cell where two rays are drawn highlighting two different track entrance 
angles of $\alpha$ = 0$^{\circ}$ and 30$^{\circ}$, and (b) an R2 cell that 
was assumed to be located within a uniform 1~T magnetic field along the z 
direction.}}
\label{garfield}
\end{figure}
%%%%%%%%%%%%%%%%%%%%%%%%%%%%%%%%%%%%%%%%%%%%%%%%%%%%%%%%%%%%%%%%%%%%%%%%%%%

Fig.~\ref{garfield} shows the isochrone contours and electric-field lines for 
a representative R3 and R2 cell.  Note that the contours are circular close 
to the wire but become hexagonal near the outer boundaries of the cell.  This 
illustrates the necessity of knowing the entry angle of the track in order to 
determine the drift distance to the sense wire from the measured drift time.

\subsubsection{Function Parameterization}
\label{funcpar} 

In the CLAS detector, the drift distance was parameterized and fit as a function
of drift time.~\cite{mdm95}.
For CLAS12, we have instead chosen to parameterize the time as a function of
distance.  This is a more natural description of the drift chamber signal
for several reasons:
\begin{itemize}
\item the maximum drift distance is given by geometry (the distance from
a sense wire to the nearest field wire) and so it is fixed
\item the drift velocity is a function of electric field strength, so the
point of minimum field is the point of minimum velocity (and thus the inflection point on the T vs X curve). 
This inflection point of the curve occurs at a
definite value of distance within the cell and not at a definite value of time.
\item the time walk due to finite ionization is
naturally parameterized as a function of distance and not as a function of time.
\item two of the major time corrections (time walk which is a function of the
particle $\beta$ and a time correction for wires in a magnetic field, $B$ which
scales like $B^2$) can simply be added to the nominal functional form.
\end{itemize}

A {\bf single functional form} is used to fill two tables: one of time indexed by discrete
values of distance for use in the simulation by GEMC and one of 
distance indexed by time for use by the track reconstruction code.


\subsection{Choice of Mathematical Forms for the Distance to Time Function}
We use a 4th order polynomial to model the distance to time relationship.

\begin{equation}
t(x) =  a x^4 + b x^3 + c x^2 + {1 \over v_0} x,
\end{equation}
\item An exponential form:

By the use of simple calculus we convert the parameters a, b, c and d to equivalent parameters which have
a physically intuitive meaning (see next section).

\subsection{Physical Constraints on the Drift Velocity Function}

Inspection of  Fig.~\ref{garfield}a reveals that for tracks near the outer
edge of the cell, the first arriving ions follow the electric-field line from 
the field wire to the sense wire, independent of track entrance angle.  Their 
corresponding drift time is referred to as $t_{max}$ and is one of the fundamental
parameters of the function. 

A second constraint is that the velocity near the wire is the ``saturated drift
velocity'' for our gas mixture, 90$\%$ argon - 10$\%$ CO$_2$.  We call this parameter $V_0$.

Another constraint is imposed by the fact that there is a definite point in the
cell at which the electric field is a minimum.  This implies that this is the point
of minimum velocity and is thus an inflection point.  This occurs at a value
$r = (x/x_{max}) = 0.615$ and the drift velocity at this point is termed $V_{mid}$.

\subsubsection{Constraints on the Parameters for the Polynomial Form}
In this subsection, we present the algebra for the constraints on the parameters
(a, b, c and d) of the polynomial form.  Because there are four free parameters, we
have imposed four constraints.

Constraints on the function coefficients:
\begin{itemize}
\item  $t(x)$ must equal $t_{max}$ when $x = x_{max} $.
\item  the drift velocity near the sense wire ($x = 0$)
must equal the saturated value, $V_0$
\item the function has an inflection point (a
mininum in velocity) at the point in the cell with the lowest electric field
strength.  From the geometry of our cells, this occurs at a distance
of $ 0.615 \times x_{max}$.  Finally,
\item the velocity equals $V_{mid}$ at the inflection point.
\end{itemize}

To summarize, these are the four constraints on the distance to time functions:
\begin{enumerate}
\item $t(x = x_{max}) = t_{max}$
\item $dt / dx (x = 0) = 1 / V_0$
\item $d^2 t / dx^2 (\hat{x} = 0.615 ) = 0$   
\item $dt / dx (\hat{x} = 0.615 ) = 1/V_{mid}$  
\end{enumerate}

In this way we convert our original parameters, a, b, c and d to the physically meaningful
parameters $t_{max}, V_0, r, and V_{mid}$ where $r$ is the value 0.615 (the fractional distance
at which the inflection point occurs) which can in principle also be varied.


\subsection{Dependence of Distance to Time Function on Local Angle}
\noindent
The preceding was the derivation for the function of time as a function
of drift distance for tracks with a local angle, $\alpha = 30^0$.  
We now discuss the
functional dependence on varying local angle and on non-zero and varying
values of the B-field.

Please refer back to Fig.~\ref{garfield} which shows a 0 degree track and a 30 degree
track, both at maximum distance from the sense wire.  Note that they will give the
SAME TIME, Tmax, even though their distance-of-closest-approach differs by a factor
of cos(30deg).  If Dmax is the distance from sense to field wire (and the maximum
doca possible for a 30 deg. track), then Dmax times $\cos(30^\circ-\alpha)$ is the maximum
doca for a track with local angle, $\alpha$.  Call this distance, $Dmax_{\alpha}$.

\subsubsection{Local Angle Dependence of Polynomial Form}
We derived the function for time versus distance for a particular local angle, $\alpha$, by
assuming the same functional form as for $\alpha = 30$ but with a {\bf different coefficient, a}, which 
satisfies the constraint that  $F(dmax_{\alpha},\alpha)$ = $t_{max}$.

Using this constraint, we can solve for $a_{\alpha}$ in terms of the known coefficients $V_0, ~a, ~n$, and $m$,
yielding the following:
\begin{equation}
\label{aalphaequation}
a_{\alpha} = {{t_{max} - b dmax_{alpha}^3 - c dmax_{alpha}^2 - d dmax_{alpha}}\over{dmax_{alpha}^4}}
\end{equation}

Using this formula for $a_{\alpha}$ we can derive the time as a function of distance and local
angle, $\alpha$ as shown in Fig.~\ref{xvst}.  See, for instance, the upper-left sub-figure 
which shows the time as function of distance for 5 different angles between $0^{\circ}$ and 
$30^{\circ}$, equally spaced in $\cos \left(30^\circ-\alpha\right)$.  Note two things:
\begin{enumerate}
\item for each angle, $\alpha$, the time is $t_{max}$ at $dmax_{\alpha}$, and
\item the distances for a given time are linear in $\cos \left(30^\circ-\alpha\right)$.
\end{enumerate}

Thus, the general functional form for time as a function of distance and local angle, $\alpha$
is given by
\begin{equation}
\label{tfunctionofxandlocalangle}
t(x,\alpha) = a_{\alpha} x^4 + b x^3 + c x^2 + d x
\end{equation}




\subsection{Dependence of Distance to Time Function on Magnetic Field Strength}
Since the R2 chambers are located within the field region of the CLAS torus, the 
magnetic field affects the drift velocity as shown in 
Fig.~\ref{xvst}b.  In particular, the field rotates and shrinks the isochrones
as shown in Fig.~\ref{garfield}b.  These effects can be modeled by a 
modification to the effective entrance angle of the track and by an increase 
in the time at a particular DOCA.  Both of these corrections are assumed to depend only on the 
magnitude of the magnetic field, and not its direction, following a study 
described in Ref~\cite{MM-IEEE}.  

The rotation of the isochrones is parameterized as a shift in the effective
entrance angle.  
\begin{equation} 
\label{eq-bcorrn-to-ang}
\alpha_b = \alpha_0 + \alpha_c \cos^{-1}(1 - a B), 
\end{equation}

The correction term $\alpha_c$ is determined from a 
GARFIELD simulation to be:

\begin{equation} 
\label{eq-bang}
\alpha_c = \cos^{-1}(1 - a B), 
\end{equation}

\noindent
where $a$ is a constant equal to $0.02$ and $B$ is the magnetic field strength in Tesla and angular
units are degrees.

%%%%%%%%% Figure : TRKDOCA vs. Drift Time -- Angle and Field Dependence %%%%%%%%%%
\begin{figure}[htb]
\vspace{15.cm} 
\special{psfile=images/tvsx.eps hscale=80 vscale=80 hoffset=-20 voffset=-10}
\caption{\small{Scatterplot of the corrected drift time versus TRKDOCA for 
(upper-left) R1, showing curves for various local angles from 30$^{\circ}$
(righmost curve) to 0$^{\circ}$ (leftmost curve).  (Upper-right) for R2; 
additionally showing 3 bands for B-field magnitudes of 0, 1, 1.5 Tesla.
(Lower-left) for R3 with the inflection point identified.}}
\label{xvst}
\end{figure}
%%%%%%%%%%%%%%%%%%%%%%%%%%%%%%%%%%%%%%%%%%%%%%%%%%%%%%%%%%%%%%%%%%%%%%%%%%%%%%%


The maximum drift time used in the time-to-distance function was extracted 
directly from the data.  For R2 the maximum drift time was parameterized as:

\begin{equation} 
\label{eq-bmax}
t_{max}(B) = t_{max}(0) + b B^2,
\end{equation}

\noindent
where $b$ is a constant and $B$ is the magnetic field strength.

At any given local magnetic field point, the distance-to-time function 
includes an additional correction term $\delta t_B$ to describe 
the magnetic field dependence.  See this reference~\cite{qin96} for a related
parameterization of the change in the distance at a particular time due to a
B-field. 

\begin{equation}
\label{XTB}
t(\hat{x},\alpha,B) = t(\hat{x},\alpha-\alpha_c, B=0) +  \beta(\hat{x})*B^2.
\end{equation}

\noindent
In this expression, the first term is the time calculated assuming B=0, and the
second term is the time increase due to the B field.  For the R1 and R3 functions, no magnetic field 
dependence is included, as the chambers are located outside the torus 
cryostats in regions that are relatively field-free.


\section{Determining the Distance to Time Function Parameters}
We determine the experimental values of the function parameters by fitting
a histogram of $TRKDOCA$ vs. time.

\subsection{Method of Calibrating the Time-to-Distance Function}
\label{tdistcal}

Each hit on a track is characterized by two parameters, the measured drift 
time from the sense wire and the distance-of-closest-approach (TRKDOCA) to the 
sense wire.  A best fit to the dependence of time on TRKDOCA determines the
values of the parameters of the drift-velocity function. 

However, several factors 
complicate this analysis. For example, the TRKDOCAs obtained from the fitted 
tracks are biased quantities since an initial estimate of the drift-velocity 
function is used in the track determination.  Moreover, the drift cells and
the resulting isochrones are 
not circular, as the analysis implicitly assumes, but are hexagonal, leading 
to angle-dependent corrections.   The R2 chambers, in particular, are in a 
region of high and spatially varying magnetic field which affects the distance
versus time function.  Finally, the different 
ionization densities of the tracks from particles with different velocities 
leads to substantial time-walk corrections.  Each of 
these points is briefly discussed in the next section.
For these reasons, our fitting is an iterative process in which we calibrate,
re-do track fitting, re-calibrate, etc.  We usually converge in 1 or 2 iterations.

\section{Using the Distance to Time Function in Reconstruction}
The track reconstruction program needs to know the expected distance as a function
of time.  However, as explained in the previous paragraph, we will have calibrated and fitted
the observed time as a function of distance.  So, we need to NUMERICALLY INVERT the t=f(x)
function in order to fill a table of X (real number) as a function of the time index (integer).

\subsection{Filling the Time to Distance and Distance to Time Tables}
We calibrate the distance to time function by fitting the drift time versus
TRKDOCA at a particular local angle, alpha.  We write the independent parameters,
$V_0, V_{mid} amd T_{max}$ to the data-base.  At the beginning of a reconstruction program,
an $inversion program$ is run to numerical invert the function to produce a table
of distance indexed by time.


\subsection{How to Interpolate and Extrapolate in Local Angle}
Please refer back to Fig.~\ref{xvst} in order to understand the local-angle dependence
of distance versus time.  When the time is equal to $t_{max}$ the distance is equal to
the largest value for the local angle; that is, $dmax_{\alpha}$.  Also note that by
simple geometrical reasoning, $dmax_{\alpha} = dmax$  $cos(30-\alpha)$.
We assume that at times less than tmax and distances less than dmax, the calculated
distances still vary linearly as $cos(30-\alpha)$.  This angle dependence is built into
our functional form.

This means that we
\begin{itemize}
\item {\bf Fill} our time to distance tables for different local angles using the function, and
\item {\bf Interpolate} between time to distance tables for different local angles to obtain
the calculated distance at a particular local angle
\end{itemize}
For example if ``$X_0$'' is the distance (at a particular time) for a table filled for tracks with local angle of 0 degrees
and ``$X_{30}$'' is the corresponding quantity for a table of 30 degree tracks, then
\begin{equation} 
\label{eq-extrap30}
X(t,\alpha) = X_0 + (X_{30}-X_0) (cos(30-\alpha) - cos(30)) / (1. - cos(30))
\end{equation}



