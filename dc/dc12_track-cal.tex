\section{Calibration Procedures}

\subsection{Track Reconstruction Overview}

\hskip 0.15 in
The reconstruction of charged-particle tracks is performed in several stages.  In 
the first stage, individual tracks are fit only to hit-wire positions in a 
procedure known as ``hit-based'' tracking.  Hit-based tracking proceeds
in the following steps:
\begin{itemize}
\item in each superlayer, ``clusters'' of hits which are consistent with
being part of a track are identified
\item a ``noise rejection algorithm'' is then applied to the clusters, 
with one of the more efficient algorithms being the removal of the
interior hits from horizontal ``strings'' of hits along a layer.
The effectiveness of this cut follows from the observation that
high-momentum tracks from hadrons typically cross the superlayers
at a large angle, while ``curlers'' from low-momentum background
follow curling trajectories with a significant part of the pattern
being along layers
\item the resulting trimmed clusters are then fit to a straight-line hypothesis,
and those hits with acceptable residuals are kept and identified collectively
as a ``track segment''
\item because the ``hits'' in a track segment are 2-dimensional objects (we
do not know their location along the wire direction) a track segment is not
a line but a plane.  Thus pairs of segments in neighboring superlayers within
one chamber (with superlayers of +/- 6deg stereo angle) represent the
intersection of two planes; that is a line.  This line's coordinates 
are evaluated midway between the two superlayers, and is a 6-dimensional
object (x,y,z and 3 angles) which we call a ``cross''.
\item in the first pattern-recognition step to find a track candidate,
the positions of 3 crosses (1 each in R1, R2 and R3) are fit to a
parabolic functional form to give us a ``track candidate''.
\item the track candidate is actually the ``state vector'' of our
Kalman filter tracking program
\item the state vector's ``member coordinates'' are the following
\begin{itemize}
\item x and y at a specified z-location, and
\item the two angles, thetax and thetay and
\item the inverse momentum of the track candidate
\end{itemize} 

\end{itemize} 

Due to the comparatively small size of the drift cells and the large 
number of wire layers, the track momenta can already be reconstructed 
at the ``hit-based'' level with a 
resolution of 3$\%$ to 5$\%$.  Additional information on these tracks, derived
from the {\v C}erenkov, time-of-flight, and electromagnetic calorimeter 
detectors, allows for determination of the identities and speeds of the 
charged particles.  In the second stage of the analysis, flight-time 
information of the particles from the target to the outer scintillators is 
used to correct the measured drift times.  A pre-determined table is then used
to convert the corrected drift times to drift distances (see section~\ref{tdistcal} 
for details of the function). These corrected 
track positions in each drift cell are input data to our Kalman filter
fit to determine the final track parameters.  This is colloquially referred
to as ``time-based tracking''.

The drift time is given by the following expression:

\begin{equation} 
\label{drift}
t_{drift} = t_{tdc} - t_{start} - t_{0} - t_{flight} - t_{prop} - t_{walk},
\end{equation}

\noindent
where $t_{TDC}$ is the raw time measured by the TDC, $t_{start}$ is the event start time, 
$t_0$ is the fixed-time (cable) delay for the wire, $t_{flight}$ is the 
flight time of the particle from the interaction vertex to the wire, $t_{prop}$ 
is the signal propagation time along the wire, and $t_{walk}$ is a time-walk 
correction made for short drift times due to different ionizations for slow 
and fast particles.  

With a trigger based on detecting an electron in the CLAS12 detector, the event start time is 
given by the time-of-flight counter time for the primary scattered electron 
corrected for the calculated flight time of this electron from the beam-target vertex.

Through the use of an appropriate function, the drift time determines the 
distance-of-closest-approach (DOCA) of the charged-particle track to the sense 
wire.  However, there remains an ambiguity regarding which side of the sense 
wire the track passed by.  This ``left-right ambiguity'' is resolved within 
the individual superlayers by comparing the $\chi^2$ values for the track fit 
for all different combinations of drift-distance signs.  After selection of 
the full set of drift-distance signs within each superlayer, a final fit 
results in improved track parameters.

\subsection{Alignment Procedures}
\label{align}

\hskip 0.15in
As described in section~\ref{survey}, each of the 18 drift chambers was 
surveyed after installation into CLAS.  The survey procedures had sub-millimeter
accuracy, but we wanted an independent check of the chambers' positions.  For 
this reason, the survey values for the chamber geometry were viewed only as 
a reasonable starting point to be refined by comparisons with data.

---------(more needed here)----------------
To adjust the chamber geometry parameters to improve the tracking resolution,
``straight-track'' data with the torus magnetic field off were analyzed.  The 
results of this procedure indicated that the best-fit position of the chambers 
along the three coordinate axes varied by up to several millimeters relative 
to the surveyed positions.  These studies of the absolute and relative chamber 
positioning are still ongoing, and should ultimately allow for the best-fit 
determination of more complex degrees of freedom such as chamber pitch, yaw, 
and roll.  In addition we have functions to correct for  wire sag due to 
gravitational forces and chamber distortions, which can cause 1 - 2 millimeter
changes in hit position.

\subsection{Time-Delay Calibration}
\label{tdlycal}

\hskip 0.15 in
As indicated in eq(\ref{drift}), the fixed-time delays (mainly fixed cable delays) 
for each wire must be known in order to determine the drift times.   To determine
this t$_0$ value, we produced a histogram of the following quantity for all hits
used on tracks:$ ( t_{tdc} - t_{start} - t_{flight} - t_{prop} - t_{walk} )$.
This produced a characteristic plot of a drift chamber signal on a flat
background from out-of-time tracks.  Our 24,192 drift chamber signals are carried
on individually made multi-conductor cables which each carred 16 signals.  We thus
produced and analysed 24,192/16 = 1512 histograms to determine that many values of
t$_0$. We occasionally re-do this analysis whenever we have a new trigger condition
or new configuration of readout boards.


\subsection{Time-to-Distance Calibration}
\label{tdistcal}

\hskip 0.15in
Each hit on a track is characterized by two variables, the measured drift 
time from the sense wire and the distance-of-closest-approach (DOCA) to the 
sense wire resulting from the track fit.  
A best fit to the dependence of DOCA on time defines the 
drift-velocity function of the drift cells. However, several factors 
complicate this analysis. For example, the DOCAs obtained from the fitted 
tracks are biased quantities since an initial estimate of the drift-velocity 
function is used in the track determination.  Moreover, the drift cells are 
not circular, as the analysis implicitly assumes, but are hexagonal, leading 
to angle-dependent corrections.   The R2 chambers, in particular, are in a 
region of high and spatially varying magnetic field.  Finally, the different 
ionization densities of the tracks from particles with different velocities 
leads to substantial time-walk corrections for tracks near the wire.  Each of 
these points is briefly discussed in this section.

The effect of the DOCA bias was reduced by fitting data from a selected single
layer in each superlayer that was excluded from the original track fit.  
However, excluding a layer from the track fit is not enough to entirely 
eliminate all bias.  Because of the regular brick-wall cell layout, tracks at 
entrance angles of 0$^{\circ}$ or 30$^{\circ}$ yield biased values for DOCA, 
even from the excluded layer, because the times in the included layers are 
strongly correlated with the value of the time in the excluded layer.

%%%%%%%%%%%%%%%%%%%%%% Figure : Garfield Picture %%%%%%%%%%%%%%%%%%%%%%%%%%
\begin{figure}[htpb]
\vspace{4.5cm} 
\special{psfile=garfield.eps hscale=49 vscale=49 hoffset=-15 voffset=-130}
\caption{\small{Plot of electric-field lines and equal-time isochrone contours
(100 ns interval) for a 90$\%$ argon - 10$\%$ CO$_2$ gas mixture for (a) an R3
drift cell where two rays are drawn highlighting two different track entrance 
angles of $\alpha$ = 0$^{\circ}$ and 30$^{\circ}$, and (b) an R2 cell that 
was assumed to be located within a uniform 1~T magnetic field along the z 
direction.}}
\label{garfield}
\end{figure}
%%%%%%%%%%%%%%%%%%%%%%%%%%%%%%%%%%%%%%%%%%%%%%%%%%%%%%%%%%%%%%%%%%%%%%%%%%%

Fig.~\ref{garfield} shows the isochrone contours and electric-field lines for 
a representative R3 and R2 cell.  Note that the contours are circular close 
to the wire but become hexagonal near the outer boundaries of the cell.  This 
illustrates the necessity of knowing the entry angle of the track in order to 
determine the drift distance to the sense wire from the measured drift time.

\subsubsection{Function Parameterization}
\label{funcpar} 

\hskip 0.15 in
The drift time to drift distance function for R3 at a given track entrance 
angle is parameterized in terms of a power law~\cite{mdm95} as:

\begin{equation}
x(t) = v_0 t + \eta \left( {t \over t_{max}} \right)^p
+ \kappa \left( { t \over t_{max}} \right)^q,
\end{equation}

\noindent
where $v_0$ is the value of the saturated drift velocity near $t$=0, and the 
coefficients $\eta$ and $\kappa$, and the exponents $p$ and $q$, are 
determined by fitting the time-to-distance correlation.  Polynomial forms give
comparable results and, in fact, are used for R1 and R2.

Inspection of  Fig.~\ref{garfield}a reveals that for tracks near the outer
edge of the cell, the first arriving ions follow the electric-field line from 
the field wire to the sense wire, independent of track entrance angle.  Their 
corresponding drift time is referred to as $t_{max}$.  A reduced or normalized 
drift time $\hat{t}=t/t_{max}$ is used as an argument of the time-to-distance
function that satisfies the cell boundary constraint,

\begin{equation} 
\label{eq-xtmax}
x(\hat{t}=1,\alpha)=C \cdot \cos(30^\circ-\alpha).
\end{equation}

\noindent
The angle $\alpha$ is the track entrance angle and $C$ represents the cell 
size.  At any given entrance angle, as shown in Fig.~\ref{xvst}a, the 
time-to-distance function can be deduced using a correction function 
$f(\hat{t})$:

\begin{equation} 
\label{eq-xtang}
x(\hat{t},\alpha) = x_0(\hat{t},\alpha_0) +C \left (\cos \left(30^\circ-\alpha\right)
-\cos\left(30^\circ-\alpha_0\right)\right) f(\hat{t}).
\end{equation}

\noindent
In this expression, $x_0$ represents the drift distance expected for a given
normalized drift time assuming an entrance angle $\alpha_0$.  This entrance
angle represents the average entrance angle for the full fitted data sample.
The function $f(\hat{t})$ is used to correct the extracted drift distance for 
the true entrance angle of the track.

Since the R2 chambers are located between the torus cryostats, the 
inhomogeneous magnetic field affects the drift velocity as shown in 
Fig.~\ref{xvst}b.  In particular, the field rotates and shrinks the isochrones
as shown in Fig.~\ref{garfield}b.  These effects can be modeled by a 
modification to the effective entrance angle of the track and by an increase 
in $t_{max}$.  Both of these corrections are assumed to depend only on the 
magnitude of the magnetic field, and not its direction, following a study 
described in Ref~\cite{MM-IEEE}.  

The rotation of the isochrones is parameterized as a shift in the effective
entrance angle.  This correction term $\alpha_c$ is determined from a 
GARFIELD simulation to be:

\begin{equation} 
\label{eq-bang}
\alpha_c = \cos^{-1}(1 - a B), 
\end{equation}

\noindent
where $a$ is a constant and $B$ is the magnetic field strength.

%%%%%%%%% Figure : DOCA vs. Drift Time -- Angle and Field Dependence %%%%%%%%%%
\begin{figure}[htpb]
\vspace{9.8cm} 
\special{psfile=xvst.eps hscale=60 vscale=60 hoffset=-70 voffset=-80}
\caption{\small{Scatterplot of DOCA versus the corrected drift time for (a) R3
axial wires showing the track local-angle dependence, and (b) R2 axial wires
showing the magnetic-field dependence where the local angle range is between 
23$^{\circ}$ and 25$^{\circ}$.  The overlaid curves represent the fitted 
time-to-distance function.}}
\label{xvst}
\end{figure}
%%%%%%%%%%%%%%%%%%%%%%%%%%%%%%%%%%%%%%%%%%%%%%%%%%%%%%%%%%%%%%%%%%%%%%%%%%%%%%%


The maximum drift time used in the time-to-distance function was extracted 
directly from the data.  For R2 the maximum drift time was parameterized as:

\begin{equation} 
\label{eq-bmax}
t_{max}(B) = t_{max}(0) + b B^2,
\end{equation}

\noindent
where $b$ is a constant and $B$ is the magnetic field strength.

At any given local magnetic field point, the time-to-distance function 
included an additional linear correction term $\beta(\hat{t})$ to describe 
the magnetic field dependence~\cite{qin96}, 

\begin{equation}
\label{XTB}
x(\hat{t},\alpha,B) = x(\hat{t},\alpha-\alpha_c,B_0) + (B-B_0) \beta(\hat{t}).
\end{equation}

\noindent
In this expression, $B_0$ represents the average magnetic field value for the 
full fitted data sample.  For the R1 and R3 functions, no magnetic field 
dependence is included, as the chambers are located outside the torus 
cryostats in regions that are relatively field-free.


