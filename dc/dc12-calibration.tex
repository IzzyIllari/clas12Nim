\documentclass{article}
\usepackage{graphicx}
\usepackage{epsfig}
\setlength{\textheight}{8.5in}
\setlength{\textwidth}{6.in}
\setlength{\oddsidemargin}{0.in}
\setlength{\evensidemargin}{0.in}


\begin{document}

\title{CLAS12 Drift Chamber Calibration:\\
Distance vs. Time Functional Form}
\author{
\vspace{5cm}\\
Mac Mestayer\\
Jefferson Lab\\
\vspace{0.20cm}\\
Krishna Adhikari\\
Mississippi State University\\
\vspace{0.3cm}\\
}
\date{\today}
\maketitle


\begin{abstract}

The drift time to distance functions are described including a listing of
the calibration parameters, their meaning, and where they are
kept in the calibration database.  

Our emphasis throughout this note is on how to best model
the functional dependence of the drift time on the drift
distance for different values of the local angle, $\alpha$,
particle velocity, $\beta$ and magnitude of the magnetic field,
$B$. Questions, comments, criticisms are welcome!

\end{abstract}
\pagebreak

\tableofcontents



%%%%%%%%%%%%%%%% Drift Chamber Nomenclature %%%%%%%%%%%%%%%%%%%%%

\section{Drift Chamber Nomenclature}
\label{sec:nomenclature}
The CLAS drift chambers are mounted directly onto the torus magnet. 
The ``Region 1'' chambers are mounted on the upstream face of the
torus, ``Region 2'' chambers are situated between adjacent coils and
``Region 3'' chambers are located on the downstream face of the torus.
Because the chambers are approximately triangular solids, the active
response area of the chamber matches the triangular wedge between the
torus coils.

\begin{figure}[hbt]
\centering
\includegraphics[width=0.8\textwidth]{images/dcgeom.eps}
\caption{Schematic of CLAS Drift chambers showing how regions and 
superlayers are named}
\label{fig:dcgeom}
\end{figure}

Figure \ref{fig:dcgeom} gives a basic idea of
how the chambers are arranged. 
It shows a slice through the chamber system at the z-position of the
target.  The ``sectors'' are apparent as the six azimuthal segments.  Within
each sector the chambers are grouped radially into three ``regions'' of two
``superlayers'' each.


	
Each region is a separate physical chamber volume containing
two superlayers. Each superlayer contains six layers of sense wires. 
Each superlayer of each sector is calibrated separately for a
total of 36 sets of parameters.

When a charged particle goes through the drift chambers, each
of the 36 layers is traversed
\footnote{Because of inefficiencies and/or the rejection of hits with large
residuals, we find an average of 30 hits per time based
track.}.
Each hit detected in the chamber is used to determine the particle's
track via a least squares fit done inside the CLAS reconstruction 
program.  Two terms are used to describe
the distance a charged particle track is from a sense wire:
\begin{itemize}
\item $TRKDOCA$ - (Distance Of Closest Approach) The distance from the fitted
track to the sense wire.
\item $DOCA$ - The calculated distance from the sense wire to the track. This
is calculated from the drift time as well as some other parameters.
\end{itemize}
These two values are signed quantities: being positive if the hit is to the
``right'' of the wire (closer to the beamline) and being negative if the
hit is to the ``left'' of the wire (further from the beamline.

\noindent
Additionally, the $residual$ has two defintions:
\begin{itemize}
\item $timeResidual = abs(TRKDOCA)-abs(CALCDOCA)$.  Note that $timeResidual$ is
positive if the track is outside the calculated distance of closest approach
and negative if inside.  If a histogram of $timeResidual$ is offset from 0, it
means that there is a systematic offset in the time vs. distance function.
\item $fitResidual = TRKDOCA-CALCDOCA$  Note that $fitResidual$ is positive
if the track is to the ``right'' of the wire and negative if to the ``left'',
so an offset of the distribution of $fitResidual$ indicates a misalignment.
\end{itemize}


Both $TRKDOCA$ and $CALCDOCA$ are signed quantities.
$TRKDOCA$'s sign is determined by whether the track passed
to the right or to the left of the wire.  $CALCDOCA$'s sign is similarly 
determined by the ``left-right ambiguity resolution'' subroutine.



\section{Track Reconstruction Overview}

The reconstruction of charged-particle tracks is performed in two stages.  In 
the first stage, individual tracks are fit only to hit-wire positions in a 
procedure known as ``hit-based'' tracking.  In hit-based tracking, clusters
of hits are identified by a pattern-recognition program and are fit to a straight line
to idenify track``segments'' within individual superlayers. Segments in adjacent
superlayers within a chamber (1 and 2, 3 and 4, or 5 and 6)
are combined to form a ``cross'' which is a six-vector 
(position and angle) located at the middle plane between the two superlayers
of one chamber.  A polynomial form is then used to link the found ``crosses'' in
each of the three regions within a sector. 

Because only the wire positions of
the ``hit'' wires are used, this is called ``Hit-Based Tracking''.
These hit-based tracks form the seed state vector for our Kalman filter track
algorithm.

Due to the comparatively small size of the drift cells and the large 
number of wire layers, the ``hit-based'' track momenta can be reconstructed with a 
resolution of 3$\%$ to 5$\%$.  Additional information on these tracks, derived
from the {\v C}erenkov, time-of-flight, and electromagnetic calorimeter 
detectors, allows for determination of the identities and velocities of the 
charged particles.  In the second stage of the analysis, flight-time 
information of the particles from the target to the outer scintillators is 
used to correct the measured drift times.  A pre-determined table is then used
to convert the corrected drift times to drift distances.  These corrected 
track positions in each drift cell are fit in a procedure known as 
``time-based'' tracking in order to determine the final track parameters.

Through the use of an appropriate look-up table, the drift time is used to 
calculate the most likely 
distance (CALCDOCA) from the sense wire to the charged-particle track.  
However, there remains an ambiguity regarding which side of the sense 
wire the track passed by.  This ``left-right ambiguity'' is resolved within 
the individual superlayers by comparing the $\chi^2$ values for the track fit 
for different combinations of drift-distance signs.  After selection of 
the full set of drift-distance signs within each superlayer, a final fit 
results in improved track parameters.

\subsection{Drift Chamber Online (``Raw'') Signals}
\label{sec:rawsignals}

Drift chamber signals are preamplified by an on-chamber board (the ``Signal
Translation Board'') ({\bf STB}) by an amount of $\approx 2 mV/\mu A$.
These pre-amplified signals are routed to the ``Drift Chamber Readout Board'' 
({\bf DCRB}) where the signals are further amplfied, discriminated and digitized into
a time signal (ns).  Thus the raw signal from the detector consists of
a detector identification number $(crate,slot,channel)$ and a time in nanoseconds.

These raw signals are transformed into drift times by subtracting a ``fixed time delay", 
$t_0$(iwire), which accounts for the different fixed delays (cables, etc.)
to each wire in the system, and subtracting an event-dependent time
delay to account for flight times and signal propogation times relative
to an event start time.  Once the drift time is determined, it is used
to calculate the drift distance using a look-up table of distance
versus time.  Separate distance versus time tables are kept for different
values of track entrance angle (``local angle") as well as for different
values of the magnetic field.  The distance at a specific value of entrance
angle or at a specific B field is determined by interpolating between
these table values.

\subsection{Use of Calibration Constants in Track Reconstruction}
\label{sec:trkrecon}

Calibration constants are stored as run-indexed calibration data-base (CCDB)
values.  In the case of
the fixed time delays, $t_0$, a value is stored for each of the 
1512 sense wire cables (each of which contains 16 twisted-cable pairs of the
same length).  

In the case of the distance versus time function, the
CCDB values are function coefficients used to fill the distance versus
time tables for different values of track entrance angle or B field, and,
of course, for different values of time.   
There is an independent set of time-to-distance coefficients for each
sector and superlayer combination (6 x 6 = 36 in all).

The reconstruction basically works as follows:
\begin{itemize}
\item at {\bf begin run time}, arrays are filled with wire position 
information, fixed delays, $t_0$, and distance versus time information.
\item at {\bf event time}, a hit wire is turned into
a space point by getting the wire position from the geometry service, 
calculating the drift time by subtracting fixed and event dependent time
delays from the measured time, and then determining the 
distance between the wire and
the point of closest approach of the track by looking it up in the pre-filled 
time to distance table.
\end{itemize}


\subsection{Time-Delay Calibration}
\label{tdlycal}

As indicated in Equation(\ref{drift}), the fixed-time delays for each wire must be 
known in order to determine the drift times. These delays, which derive mainly 
from cable lengths, are determined by fits to time distributions accumulated
for each of the 1512 cable bundles (containing 16 wire signals each).

The drift time corrections for various fixed and event-dependent
time delays are given here:

\begin{equation} 
\label{drift}
t_{drift} = t_{raw} - t_{0} - t_{start} - t_{flight} - t_{prop} - t_{beta},
\end{equation}

\noindent
where
\begin{itemize} 
\item $t_{raw}$ is the raw time measured by the DCRB, 
\item $t_0$ is the fixed cable delays for the wire, 
\item $t_{start}$ is the event start time, 
\item $t_{flight}$ is the flight time of the particle from the reaction vertex to the wire, 
\item $t_{prop}$ is the signal propagation time along the wire, and 
\item $t_{beta}$ is a time-walk correction whose beta-dependence is due to different 
ionization for slow and fast particles.  
\end{itemize} 

With an electron beam, the event start time is 
given by the time-of-flight counter time for the primary scattered electron 
corrected for the calculated flight time of this electron from the target
and other scintillator counter-specific corrections.



\section{Distance to Time Function}
\label{funcpar} 
When a charged particle passes near a sense wire, it leaves a 
trail of ionized atoms.  The freed electrons drift to the sense wire.  In general,
the further a track is from the wire the longer the so-called "drift time".

%%%%%%%%%%%%%%%%%%%%%% Figure : Garfield Picture %%%%%%%%%%%%%%%%%%%%%%%%%%
\begin{figure}[htpb]
\vspace{7.5cm} 
\special{psfile=images/garfield.eps hscale=75 vscale=75 hoffset=0 voffset=-200}
\caption{\small{Plot of electric-field lines and equal-time isochrone contours
(100 ns interval) for a 90$\%$ argon - 10$\%$ CO$_2$ gas mixture for (a) an R3
drift cell where two rays are drawn highlighting two different track entrance 
angles of $\alpha$ = 0$^{\circ}$ and 30$^{\circ}$, and (b) an R2 cell that 
was assumed to be located within a uniform 1~T magnetic field along the z 
direction.}}
\label{garfield}
\end{figure}
%%%%%%%%%%%%%%%%%%%%%%%%%%%%%%%%%%%%%%%%%%%%%%%%%%%%%%%%%%%%%%%%%%%%%%%%%%%

Fig.~\ref{garfield}, drawn using the GARFIELD~\cite{garfield} program 
shows the isochrone contours and electric-field lines for 
a representative R3 and R2 cell.  Note that the contours are circular close 
to the wire but become hexagonal near the outer boundaries of the cell.  This 
illustrates the necessity of knowing the entry angle of the track in order to 
determine the drift distance to the sense wire from the measured drift time.

The drift time for a particular distance of closest approach of a track 
depends on a number of factors:
\begin{enumerate}
\item the electric field vector's magnitude and direction.
The field near the sense wire is radial and falls off like 1/r.
The field at the edge of the drift cell and near a field wire points
   radially outward from the field wire and falls off like 1/r where in this case
   "r" is the distance to the field wire.  The resulting isochrones thus look like
   circles close to the sense wire and morph into an hexagonal shape at the outer
   edges of the drift cell.
\item the magnetic field strength.  A higher field causes the drift electrons to follow
   a circular arc between molecular collisions which doesn't cover as much radial distance
   as would a straight-line path.  In addition, the magnetic field effectively rotates the
   hexagonal isochrones.
\item the ``local angle" of incidence.  There is no local angle dependence close to the wire
   when the isochrones are circular, but there is a large effect near the outer edge
   of the cell.
\item the beta of the particle.  The ionization density and thus the size of the signal
   voltage depends on the velocity of the particle.  This in turn directly affects the
   ``time walk" of the signal, that is, the time it takes the signal to cross discriminator
   threshold.
\end{enumerate}

In the CLAS detector, the drift distance was parameterized and fit as a function
of drift time.~\cite{mdm95}.
For CLAS12, we have instead chosen to parameterize the time as a function of
distance.  This is a more natural description of the drift chamber signal
for several reasons:
\begin{itemize}
\item the maximum drift distance is given by geometry (the distance from
a sense wire to the nearest field wire) and so it is fixed and can be used
to scale the drift distance
\item the drift velocity is a function of electric field strength, so the
point of minimum field is the point of minimum velocity (and thus the inflection point on the T vs X curve). 
This inflection point of the curve occurs at a
definite value of distance within the cell and NOT at a definite value of time.
\item the time walk due to finite ionization is
naturally parameterized as a function of distance and NOT as a function of time.
\item two of the major time corrections (time walk which is a function of the
particle $\beta$ and a time correction for wires in a magnetic field, $B$ which
scales like $B^2$) can simply be added to the nominal functional form.
\end{itemize}

A {\bf single functional form} is used to fill two tables: one of time indexed by discrete
values of distance for use in the simulation by GEMC and one of 
distance indexed by time for use by the track reconstruction code.


\subsection{Choice of Mathematical Forms for the Distance to Time Function}
We are investigating two functional forms: one is an exponential form and the other, polynomial.

The two basic functional forms (for local angle $\alpha = 30^0$, small values
of particle velocity $\beta$ and no magnetic field) are the following:
\begin{enumerate}
\item A polynomial form:
\begin{equation}
t(x) =  a x^4 + b x^3 + c x^2 + {1 \over v_0} x,
\end{equation}
\item An exponential form:
\begin{equation}
t(x) = {1 \over v_0} x + a \left( {x \over x_{max}} \right)^n
+ b \left( { x \over x_{max}} \right)^m,
\end{equation}
\end{enumerate}

\noindent
where $v_0$ is the value of the saturated drift velocity near $x$=0.

These functional forms were chosen because they can be made to satisfy several physical
constraints on the distance to time function.

\subsection{Physical Constraints on the Drift Velocity Function}

Inspection of  Fig.~\ref{garfield}a reveals that for tracks near the outer
edge of the cell, the first arriving ions follow the electric-field line from 
the field wire to the sense wire, independent of track entrance angle.  Their 
corresponding drift time is referred to as $t_{max}$ and is one of the fundamental
parameters of the function. 

A second constraint is that the velocity near the wire is the ``saturated drift
velocity'' for our gas mixture, 90$\%$ argon - 10$\%$ CO$_2$.  We call this parameter $V_0$.

Another constraint is imposed by the fact that there is a definite point in the
cell at which the electric field is a minimum.  This implies that this is the point
of minimum velocity and is thus an inflection point.  This occurs at a value
$r = (x/x_{max}) = 0.615$ and the drift velocity at this point is termed $V_{mid}$.

\subsubsection{Constraints on the Parameters for the Polynomial Form}
In this subsection, we present the algebra for the constraints on the parameters
(a, b, c and d) of the polynomial form.  Because there are four free parameters, we
have imposed four constraints.

Constraints on the function coefficients:
\begin{itemize}
\item The {\bf first constraint} is that $t(x)$ must equal $t_{max}$ when $x = x_{max} $.
\item The {\bf second constraint} is that the drift velocity near the sense wire ($x = 0$)
must equal the saturated value, $V_0$
\item The ({\bf third constraint}) is that the function has an inflection point (a
mininum in velocity) at the point in the cell with the lowest electric field
strength.  From the geometry of our cells, this occurs at a distance
of $ 0.615 \times x_{max}$.  Finally,
\item a ({\bf fourth constraint}) is that the velocity equals $V_{mid}$ at the inflection point.
\end{itemize}

To summarize, these are the four constraints on the distance to time functions:
\begin{enumerate}
\item $t(x = x_{max}) = t_{max}$
\item $dt / dx (x = 0) = 1 / V_0$
\item $d^2 t / dx^2 (\hat{x} = 0.615 ) = 0$   
\item $dt / dx (\hat{x} = 0.615 ) = 1/V_{mid}$  
\end{enumerate}

In this way we convert our original parameters, a, b, c and d to the physically meaningful
parameters $t_{max}, V_0, r, and V_{mid}$ where $r$ is the value 0.615 (the fractional distance
at which the inflection point occurs) which can in principle be varied.

\subsubsection{Constraints on the Parameters for the Exponential Form}
In this subsection, we present the algebra for the constraints on the parameters
of the exponential form:

Evaluating these equations (and assuming that $m-n = \delta mn$)
yields the following constraints among the parameters:
\begin{enumerate}
\item $x_{max}/V_0 + a + b = t_{max}$
\item $1/V_0 + (na+mb)/x_{max} = 1/V_0$
\item $a n (n-1) .615^{n-2} + b m (m-1) .615^{m-2} = 0$
\end{enumerate}

Rearranging yields the following equations:
\begin{enumerate}
\item $a + b = t_{max} - x_{max}/V_0$
\item $a + (m/n) b = 0$
\item $a + [(m(m-1))/(n(n-1))] .615^{\delta nm} b = 0$
\end{enumerate}
where $\delta nm$ is the difference between exponent values.
	
Solving equations 1 and 2 for a and b yields the following:
\begin{equation}
\label{bequation}
b = (t_{max} - x_{max}/V_0) / (1 - m/n)
\end{equation}
\begin{equation}
\label{aequation}
a = (t_{max} - x_{max}/V_0) - b)
\end{equation}
while solving equations 2 and 3 yields:
\begin{equation}
n - 1 = (m - 1) .615^{\delta nm}
\end{equation}
which simplifies to 
\begin{equation}
\label{nequation}
n = (1 + (\delta nm - 1) .615^{\delta nm}) / (1 - .615^{\delta nm})
\end{equation}
\begin{equation}
\label{mequation}
m = n + \delta nm
\end{equation}
where the {\bf three independent parameters} are the following:
\begin{itemize}
\item {\bf $V_0$} (the saturated drift velocity in the high-field region), 
\item {\bf $\delta nm$} (the difference between the exponent values) and 
\item {\bf $t_{max}$} the maximum drift time.
\end{itemize}


\subsection{Dependence of Distance to Time Function on Local Angle}
\noindent
The preceding was the derivation for the function of time as a function
of drift distance for tracks with a local angle, $\alpha = 30^0$.  
We now discuss the
functional dependence on varying local angle and on non-zero and varying
values of the B-field.

Please refer back to Fig.~\ref{garfield} which shows a 0 degree track and a 30 degree
track, both at maximum distance from the sense wire.  Note that they will give the
SAME TIME, Tmax, even though their distance-of-closest-approach differs by a factor
of cos(30deg).  If Dmax is the distance from sense to field wire (and the maximum
doca possible for a 30 deg. track), then Dmax times $\cos(30^\circ-\alpha)$ is the maximum
doca for a track with local angle, $\alpha$.  Call this distance, $Dmax_{\alpha}$.

\subsubsection{Local Angle Dependence of Polynomial Form}
We derived the function for time versus distance for a particular local angle, $\alpha$, by
assuming the same functional form as for $\alpha = 30$ but with a {\bf different coefficient, a}, which 
satisfies the constraint that  $F(dmax_{\alpha},\alpha)$ = $t_{max}$.

Using this constraint, we can solve for $a_{\alpha}$ in terms of the known coefficients $V_0, ~a, ~n$, and $m$,
yielding the following:
\begin{equation}
\label{aalphaequation}
_{\alpha}={t_{max} - b dmax_{alpha}^3 - c dmax_{alpha}^2 - d dmax_{alpha}}\over{dmax_{alpha}^4}
\end{equation}

Using this formula for $a_{\alpha}$ we can derive the time as a function of distance and local
angle, $\alpha$ as shown in Fig.~\ref{xvst}.  See, for instance, the upper-left sub-figure 
which shows the time as function of distance for 5 different angles between $0^{\circ}$ and 
$30^{\circ}$, equally spaced in $\cos \left(30^\circ-\alpha\right)$.  Note two things:
\begin{enumerate}
\item for each angle, $\alpha$, the time is $t_{max}$ at $dmax_{\alpha}$, and
\item the distances for a given time are linear in $\cos \left(30^\circ-\alpha\right)$.
\end{enumerate}

Thus, the general functional form for time as a function of distance and local angle, $\alpha$
is given by
\begin{equation}
\label{tfunctionofxandlocalangle}
t(x,\alpha) = a_{\alpha} x^4 + b x^3 + c x^2 + d x
\end{equation}.


\subsubsection{Local Angle Dependence of Exponential Form}
We derived the function for time versus distance for a particular local angle, $\alpha$, by
assuming the same functional form as for $\alpha = 30$ but with a {\bf different coefficient, b}, which 
satisfies the constraint that $t_{max}$ = $F(dmax_{\alpha},\alpha)$.

Using this constraint, we can solve for $b_{\alpha}$ in terms of the known coefficients $V_0, ~a, ~n$, and $m$,
yielding the following:
\begin{equation}
\label{balphaequation}
b_{\alpha}={{(t_{max}-dmax_{\alpha}/V_0)-a(\cos \left(30^\circ-\alpha\right)^n)}\over{\cos \left(30^\circ-\alpha\right)^m}}
\end{equation}

Using this formula for $b_{\alpha}$ we can derive the time as a function of distance and local
angle, $\alpha$ as shown in Fig.~\ref{xvst}.  See, for instance, the upper-left sub-figure 
which shows the time as function of distance for 5 different angles between $0^{\circ}$ and 
$30^{\circ}$, equally spaced in $\cos \left(30^\circ-\alpha\right)$.  Note two things:
\begin{enumerate}
\item for each angle, $\alpha$, the time is $t_{max}$ at $dmax_{\alpha}$, and
\item the distances for a given time are linear in $\cos \left(30^\circ-\alpha\right)$.
\end{enumerate}

Thus, the general functional form for time as a function of distance and local angle, $\alpha$
is given by
\begin{equation}
\label{tfunctionofxandlocalangle}
t(x,\alpha) = {1 \over v_0} t + a \left( {x \over x_{max}} \right)^n
+ b_\alpha \left( { x \over x_{max}} \right)^m,
\end{equation}
where the {\bf independent parameters} are {\bf $V_0, \delta_{nm}$} and {\bf $t_{max}$} and the intermediate
parameters a, $b_\alpha$, n and m are given by equations (\ref{aequation}), (\ref{balphaequation}), 
(\ref{nequation}) and (\ref{mequation}), respectively.

This function parameters {\bf $V_0, \delta_{nm}$} and {\bf $t_{max}$} are what is determined in a particular calibration.
The average local angle, $\alpha_0$, is also written to the CALDB.  
This entrance
angle represents the average entrance angle for the full fitted data sample.
The distance versus time look-up tables are constructed at begin run
time by extrapolating the function $x_0$ from its
nominal local angle, $\alpha_0$, to the local angles of the tables, 
nominally $0^\circ$ and $30^\circ$.  This extrapolation procedure and the related
interpolation procedure are discussed in the next section.

\subsection{Dependence of Distance to Time Function on Magnetic Field Strength}
Since the R2 chambers are located within the field region of the CLAS torus, the 
magnetic field affects the drift velocity as shown in 
Fig.~\ref{xvst}b.  In particular, the field rotates and shrinks the isochrones
as shown in Fig.~\ref{garfield}b.  These effects can be modeled by a 
modification to the effective entrance angle of the track and by an increase 
in the time at a particular DOCA.  Both of these corrections are assumed to depend only on the 
magnitude of the magnetic field, and not its direction, following a study 
described in Ref~\cite{MM-IEEE}.  

The rotation of the isochrones is parameterized as a shift in the effective
entrance angle.  This correction term $\alpha_c$ is determined from a 
GARFIELD simulation to be:

\begin{equation} 
\label{eq-bang}
\alpha_c = \cos^{-1}(1 - a B), 
\end{equation}

\noindent
where $a$ is a constant and $B$ is the magnetic field strength.

%%%%%%%%% Figure : TRKDOCA vs. Drift Time -- Angle and Field Dependence %%%%%%%%%%
\begin{figure}[htb]
\vspace{15.cm} 
\special{psfile=images/tvsx.eps hscale=80 vscale=80 hoffset=-20 voffset=-10}
\caption{\small{Scatterplot of the corrected drift time versus TRKDOCA for 
(upper-left) R1, showing curves for various local angles from 30$^{\circ}$
(righmost curve) to 0$^{\circ}$ (leftmost curve).  (Upper-right) for R2; 
additionally showing 3 bands for B-field magnitudes of 0, 1, 1.5 Tesla.
(Lower-left) for R3 with the inflection point identified.}}
\label{xvst}
\end{figure}
%%%%%%%%%%%%%%%%%%%%%%%%%%%%%%%%%%%%%%%%%%%%%%%%%%%%%%%%%%%%%%%%%%%%%%%%%%%%%%%


The maximum drift time used in the time-to-distance function was extracted 
directly from the data.  For R2 the maximum drift time was parameterized as:

\begin{equation} 
\label{eq-bmax}
t_{max}(B) = t_{max}(0) + b B^2,
\end{equation}

\noindent
where $b$ is a constant and $B$ is the magnetic field strength.

At any given local magnetic field point, the distance-to-time function 
includes an additional correction term $\delta t_B$ to describe 
the magnetic field dependence.  See this reference~\cite{qin96} for a related
parameterization of the change in the distance at a particular time due to a
B-field. 

\begin{equation}
\label{XTB}
t(\hat{x},\alpha,B) = t(\hat{x},\alpha-\alpha_c, B=0) +  \beta(\hat{x})*B^2.
\end{equation}

\noindent
In this expression, the first term is the time calculated assuming B=0, and the
second term is the time increase due to the B field.  For the R1 and R3 functions, no magnetic field 
dependence is included, as the chambers are located outside the torus 
cryostats in regions that are relatively field-free.


\section{Determining the Distance to Time Function Parameters}
We determine the experimental values of the function parameters by fitting
a histogram of $TRKDOCA$ vs. time.

\subsection{Method of Calibrating the Time-to-Distance Function}
\label{tdistcal}

Each hit on a track is characterized by two parameters, the measured drift 
time from the sense wire and the distance-of-closest-approach (TRKDOCA) to the 
sense wire.  A best fit to the dependence of time on TRKDOCA determines the
values of the parameters of the drift-velocity function. 

However, several factors 
complicate this analysis. For example, the TRKDOCAs obtained from the fitted 
tracks are biased quantities since an initial estimate of the drift-velocity 
function is used in the track determination.  Moreover, the drift cells and
the resulting isochrones are 
not circular, as the analysis implicitly assumes, but are hexagonal, leading 
to angle-dependent corrections.   The R2 chambers, in particular, are in a 
region of high and spatially varying magnetic field which affects the distance
versus time function.  Finally, the different 
ionization densities of the tracks from particles with different velocities 
leads to substantial time-walk corrections.  Each of 
these points is briefly discussed in the next section.
For these reasons, our fitting is an iterative process in which we calibrate,
re-do track fitting, re-calibrate, etc.  We usually converge in 1 or 2 iterations.

\section{Using the Distance to Time Function in Reconstruction}
The track reconstruction program needs to know the expected distance as a function
of time.  However, as explained in the previous paragraph, we will have calibrated and fitted
the observed time as a function of distance.  So, we need to NUMERICALLY INVERT the t=f(x)
function in order to fill a table of X (real number) as a function of the time index (integer).

\subsection{Filling the Time to Distance and Distance to Time Tables}
We calibrate the distance to time function by fitting the drift time versus
TRKDOCA at a particular local angle, alpha.  We write the independent parameters,
($V_0, deltanm, and T_{max}$ for the exponential function or $V_0, V_{mid} amd T_{max}$ for
the polynomial form) to the data-base.  At the beginning of a reconstruction program,
an $inversion program$ is run to numerical invert the function to produce a table
of distance indexed by time.


\subsection{How to Interpolate and Extrapolate in Local Angle}
Please refer back to Fig.~\ref{xvst} in order to understand the local-angle dependence
of distance versus time.  When the time is equal to $t_{max}$ the distance is equal to
the largest value for the local angle; that is, $dmax_{\alpha}$.  Also note that by
simple geometrical reasoning, $dmax_{\alpha} = dmax  cos(30-alpha)$.
We assume that at times less than tmax and distances less than dmax, the calculated
distances still vary linearly as $cos(30-alpha)$.  This angle dependence is built into
our functional form.

This means that we
\begin{itemize}
\item {\bf Fill} our time to distance tables for different local angles using the function, and
\item {\bf Interpolate} between time to distance tables for different local angles to obtain
the calculated distance at a particular local angle
\end{itemize}
For example if ``X_0'' is the distance (at a particular time) for a table filled for tracks with local angle of 0 degrees
and ``X_{30}'' is the corresponding quantity for a table of 30 degree tracks, then
\begin{equation} 
\label{eq-extrap30}
X(t,\alpha) = X_0 + (X_{30}-X_0) (cos(30-\alpha) - cos(30)) / (1. - cos(30))
\end{equation}


\subsection{How to Interpolate and Extrapolate in B-Field Value}

% -------------------------------------------------

\input{references}

% -------------------------------------------------


\end{document}
