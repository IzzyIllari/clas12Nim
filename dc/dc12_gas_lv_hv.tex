\section{Drift Chamber Utilities: Gas, Low Voltage and High Voltage Systems}

\subsection{Gas System: Mixing, Monitoring and Pressure Control}

The chambers operate on a gas mixture consisting of 90\% Argon and 10\% CO$_2$.
Using precision Mass-Flow Controllers (MFC), the gas is mixed and temporarily
stored in large-volume buffer tanks.  From these tanks it is delivered
to the experimental hall.  

In the hall, pressure and flow control is achieved by a pair of MFC's with
fail-safe bubblers to prevent any over- or under-pressure condition in
the chambers.  
Fig.~\ref{dc-gas-system} is a schematic of the gas delivery system 
and a snapshot 
of the control panel for monitoring the state of the system.

%%%%%%%%%%%%%%%%%%%%%% Figure : DC Sector Schematic %%%%%%%%%%%%%%%%%%%%%%
\begin{figure}[htpb]   
\vspace{4.5cm}
\special{psfile=r3dc.eps hscale=55 vscale=55 hoffset=-30 voffset=-150} 
\caption{\small{A schematic of the gas system and smapshot of the monitoring
and control panel screen.}}
\label{dc-gas-system}
\end{figure}   
%%%%%%%%%%%%%%%%%%%%%%%%%%%%%%%%%%%%%%%%%%%%%%%%%%%%%%%%%%%%%%%%%%%%%%%%%%

 
\subsection{Low Voltage System}
TWe reusded tha low voltage power supplies that were used for {\tt CLAS}.  
These units are robust and manufactured by Hewlett Packard.  
The supplies are remotely programmable and monitored.   We 
isolated the low voltage from 
ground loops by using local voltage regulators on the pre-amplifier interface 
boards (STBs).  The segmentation of the low voltage distribution cables is 
based on 32 preamplifier channels per supply cable.  Each of these supply 
cables is fused with an appropriate type of fuse rated for over-current 
protection based on 32 pre-amplifier loads.  

We designed our low voltage system: supplies, fusing, cables and control
system to be as robust and maintenance-free as possible.  To minimize
the damage to the tracking system in the event of a failure such as
a shorted pre-amplifier, we built in fine segmentation with only
32 preamplifier channels per supply cable.  Each of these supply 
cables is fused with an appropriate type of fuse rated for over-current 
protection based on this load.
In the event of a short circuit which causes a fuse to blow,
a simple, external cable disconnect will reduce the size of the affected
area to 16 signal wires without the need to access the chambers.

The on-chamber pre-amplifiers require 6V and approximately 18A per chamber
(a total of 1344 pre-amps per chamber).
The low voltage power supplies are re-used from the original CLAS detector.  
These units are robust and manufactured by Hewlett Packard.  
The supplies are remotely programmable and monitored.   We 
isolated the low voltage from 
ground loops, using local voltage regulators on the pre-amplifier interface 
boards (STBs).  

Fig.~\ref{dc-low-voltage-system} is a schematic of the low voltage
supply system and a snapshot 
of the control panel for monitoring the state of the system.

\subsection{High Voltage System}

As in the case of the low-voltage system, we designed our high voltage system: 
supplies, distribution boxes, cables and control system to be as robust and 
maintenance-free as possible.  
We re-used our CAEN 
system 527 high voltage supplies with somewhat finer segmentation than our 
previous system, consistent with our total channel count dropping from 34000 
to about 24000.
To minimize the damage to the tracking system in the event of a failure such as
a broken wire, we built in very fine segmentation.
Each individual high-voltage channel powers a variable-sized group of 
wires: a 48-wire group for wires in the small-angle region, a 96-wire group
in the middle-angle region and a 192-wire group at large angles.

In the event of a failure (e.g. a broken wire) which results in a trip
of a single HV channel, we can further reduce the size of the affected
area from the whole group (48, 96 or 192 wires) to a much smaller grouping
of 8, 16 or 32 wires by an external cable disconnect without the need to 
physically access the chambers themselves.

The high-voltage supply and distribution system consists of the following:
\begin{itemize}
\item a crate-based high voltage power supplies with 36 independent
high voltage channels for each drift chamber (1344 signal wires each).
Of these 36 channels, 16 supply positive high voltage to the sense
wires, 16 supply negative voltage to the field wires, and 4 supply
positive voltage to the guard wires
\item a series of two distribution boxes which distribute the high
voltage from the supplly channels on individual cables to groups
of wires, with the group size being 8 wires (for small angle wires)
to 16 (intermediate angles) to 32 (large angles). 
\item  on-chamber printed circuit boards which distribute high voltage
to all of the wires; these are the HVTB's
\end{itemize}



Fig.~\ref{dc-high-voltage-system} is a schematic of the high voltage
supply system and a snapshot 
of the control panel for monitoring the state of the system.

%%%%%%%%%%%%%%%%%%%%%% Figure : DC Sector Schematic %%%%%%%%%%%%%%%%%%%%%%
\begin{figure}
\vspace{4.5cm}
%\includegraphics[width=0.5\columnwidth]{dc-view.png}
\caption{\small{A schematic of the high voltage supply system and smapshot of 
the monitoring and control panel screen.}}
\label{dc-high-voltage-system}
\end{figure}   
%%%%%%%%%%%%%%%%%%%%%%%%%%%%%%%%%%%%%%%%%%%%%%%%%%%%%%%%%%%%%%%%%%%%%%%%%%


