\begin{thebibliography}{99}

\bibitem{clas12-nim}
V.D. Burkert {\it et al.}, {\it ``The CLAS12 Spectrometer at Jefferson Laboratory''}, to be published in
Nucl. Inst. and Meth. A, (2020). (see this issue)

\bibitem{clasnim}
B.A. Mecking {\it et al.}, Nucl. Inst. and Methods A {\bf 503}, (2003) 513-553.

\bibitem{dcnim}
M.D. Mestayer {\it et al.}, Nucl. Inst. and Methods A {\bf 449}, (2000) 81-111.

\bibitem{kadyk}
J.A. Kadyk, Nucl. Inst. and Methods A {\bf 300}, 436 (1991).

\bibitem{nasa}
W. Campbell and J. Scialdone, ``Outgassing Data for Selecting Spacecraft Materials'', NASA internal
report RP-1124 Rev. 3 (1993).

\bibitem{sbc}
S.B. Christo, ``Considerations for Crimping the CLAS Drift Chamber Wires'', CLAS-Note
89-021, (1989). \\ https://www.jlab.org/Hall-B/notes/clas\_notes89/note89-021.pdf

\bibitem{cathode-emission}
S.B. Christo and M.D. Mestayer, ``Minimizing Cathode Emission in Drift Chambers'', CLAS-Note 92-016,
(1992). https://www.jlab.org/Hall-B/notes/clas\_notes92/92-016.pdf

\bibitem{patent}
U.S. Patent 8,863,568 ``Apparatus and procedure to characterize the surface quality 
of conductors by measuring the rate of cathode emission as a function of surface electric field 
strength'', Mac Mestayer, Steve Christo, and Mark Taylor.

\bibitem{stesalit}
S. Bernreuther {\it et al.}, Nucl. Inst. and Methods A {\bf 367}, 96 (1995); W.L. Imhof {\it et al.},
Space Science Reviews {\bf 71}, 305 (1995).

\bibitem{stesalitaging}
R. Bouclier {\it et al.}, Nucl. Inst. and Methods A {\bf 350}, 464 (1994).

\bibitem{fjb92}
F.J. Barbosa, ``A Preamp for the CLAS DC'', CLAS-Note 92-003, (1992).
https://www.jlab.org/Hall-B/notes/clas\_notes92/note92-003.pdf

\bibitem{daq-nim}
S. Boyarinov {\it et al.}, {\it ``The CLAS12 Data Acquisition System''}, to be published in Nucl. Inst.
and Meth. A, (2020). (see this issue)

\bibitem{trigger-nim}
B. Raydo {\it et al.}, {\it ``The CLAS12 Trigger System''}, to be published in Nucl. Inst. and Meth. A, (2020).
(see this issue)

\bibitem{oxygen-contamination}
Y. Chiba {\it et al.}, Nucl. Inst. and Methods A {\bf 269}, 171 (1988).

\bibitem{malter}
L. Malter, Phys. Rev. 50, 48 - 58 (1936).

\bibitem{mdm92}
M.D. Mestayer, ``Choosing the Correct Combination of Sense, Field, and Guard Wire Voltage'', CLAS-Note 
92-005, (1992). https://www.jlab.org/Hall-B/notes/clas\_notes92/note92-005.pdf

\bibitem{GARFIELD}
GARFIELD has been developed at the University of Mainz by R. Veenhof and
revised by M. Guckes and K. Peters.  See HELIOS-Note 154, (1986).

\bibitem{ftof-nim}
D.S. Carman {\it et al.}, {\it ``The CLAS12 Forward Time-of-Flight System''}, to be published in Nucl. Inst.
and Meth. A, (2020). (see this issue)

\bibitem{mdm95}
M.D. Mestayer {\it et al.}, Nucl. Inst. and Methods A {\bf 367}, 316 (1995).

\bibitem{drift-velocity-results}
T. Zhao {\it et al.}, Nucl. Inst. and Methods A {\bf 340}, (1994) 485-490.

\bibitem{MM-IEEE}
M.D. Mestayer {\it et al.}, IEEE Transactions on Nuclear Science {\bf 39} No. 4, 690 (1992).

\bibitem{qin96}
L.M. Qin {\it et al.}, ``Performance of a Region II Drift Chamber Prototype and Region II Drift
Chamber Tracking'', CLAS-Note 96-018, (1996).
https://www.jlab.org/Hall-B/notes/clas\_notes96/note96-018.ps.gz

\bibitem{recon-nim}
V. Ziegler {\it et al.}, {\it ``The CLAS12 Software Framework and Event Reconstruction''}, to be published in Nucl. Inst.
and Meth. A, (2020). (see this issue)

\bibitem{torus-ieee}
Probir K. Ghoshal {\it et al.}, IEEE Transaction on Applied Superconductivity, Vol. 29, No. 4, June 2019.

\bibitem{magmapping}
J. Newton {\it et al.} ``Measuring and Modelling the CLAS12 Torus Magnetic Field'', CLAS12-Note (to be written)

\bibitem{sim-nim}
M. Ungaro {\it et al.}, {\it ``The CLAS12 Geant4 Simulation''}, to be published in Nucl. Inst.
and Meth. A, (2020). (see this issue)

\bibitem{magnets-nim}
R. Fair {\it et al.}, {\it ``The CLAS12 Superconducting Magnets''}, to be published in Nucl. Inst.
and Meth. A, (2020). (see this issue)

\end{thebibliography}

