\begin{thebibliography}{99}

\bibitem{cdr}
CEBAF Hall B Conceptual Design Report (1990).

\bibitem{mdm92} M.D. Mestayer, ``Choosing the Correct 
Combination of Sense, Field, and Guard Wire Voltage'', CLAS-Note 
92-005, (1992).

\bibitem{kadyk}
J.A. Kadyk, Nucl. Inst. and Methods A {\bf 300}, 436 (1991).

\bibitem{cathode}
S.B. Christo and M.D. Mestayer, ``Minimizing Cathode Emission in Drift 
Chambers'', CLAS-Note 92-016, (1992).

\bibitem{nasa} W. Campbell and J. Scialdone, ``Outgassing Data for 
Selecting Spacecraft Materials'', NASA internal report RP-1124 Rev. 3 (1993).

\bibitem{sbc} S.B. Christo, ``Considerations for Crimping the
CLAS Drift Chamber Wires'', CLAS-Note 89-021, (1989).

\bibitem{stesalit} S. Bernreuther {\it et al.}, Nucl. Inst. and Methods A {\bf 367}, 
96 (1995); W.L. Imhof {\it et al.}, Space Science Reviews {\bf 71}, 305 (1995).

\bibitem{stesalitaging} R. Bouclier {\it et al.}, Nucl. Inst. and Methods A {\bf 350}, 
464 (1994).

\bibitem{malter} L. Malter, Phys. Rev. 50, 48 - 58 (1936)

\bibitem{mdm95}
M.D. Mestayer {\it et al.}, Nucl. Inst. and Methods A {\bf 367}, 316 (1995).

\bibitem{MM-IEEE} M.D. Mestayer {\it et al.}, IEEE Transactions on Nuclear 
Science {\bf 39} No. 4, 690 (1992).

\bibitem{qin96} L.M. Qin {\it et al.}, ``Performance of a Region II Drift Chamber Prototype and Region II Drift Chamber Tracking'', CLAS-Note 96-018, (1996).

\bibitem{torus-ieee} Probir K. Ghoshal {\it et al.} IEEE Transaction on Applied Superconductivity, Vol. 29, No. 4, June 2019.

\bibitem{magmapping}
J. Newton {\it et al.} ``Measuring and Modelling the CLAS12 Torus Magnetic Field'', CLAS-Note (to be written)

\bibitem{dscnim}
D.S. Carman {\it et al.}, Nucl. Inst. and Methods A {\bf 419}, 315 (1998).

\bibitem{ereport}
R.A. Schumacher and R. Magahiz, ``The Region 1 Drift Chamber 
Endplates: Procurement History and Inspection Results'', CLAS-Note 
94-018, (1994).

\bibitem{ansys}
S.J. Bianculli, Master's Thesis, University of Pittsburgh (1993).

\bibitem{rat}
R.A. Thompson {\it et al.},``Issues in Stringing the CEBAF Large 
Acceptance Spectrometer Region 1 Drift Chamber", CLAS-Note 96-019, 
(1996).

\bibitem{r2nim} L.M. Qin {\it et al.}, Nucl. Inst. and Methods A {\bf 411}, 265 (1998).

\bibitem{chew89} M. Chew {\it et al.}, ``Investigations into Wire Sag in the
CLAS Drift Chambers'', CLAS-Notes 89-016, 89-017, (1989).

\bibitem{roth}
S.A. Roth and R.A. Schumacher, Nucl. Inst. and Methods A {\bf 369}, 215 (1996).

\bibitem{r1survcn}
R.A. Schumacher, ``Region One Drift Chamber Analysis of Survey Data'',
CLAS-Note 98-001, (1998).

\bibitem{fjb92}
F.J. Barbosa, ``A Preamp for the CLAS DC'', CLAS-Note 92-003, (1992).

\bibitem{mfv96}
M.F. Vineyard, T.J. Carroll, and M.N. Lack, in Proceedings of the 1995 International
Conference on Accelerator and Large Experimental Physics Control Systems, 
edited by M. Crowley-Milling, P. Lucas, and P. Schoessow (Fermilab Report 
CONF-96/069), W-PO-37 (1996).

\bibitem{GARFIELD}
GARFIELD has been developed at the University of Mainz by R. Veenhof and
revised by M. Guckes and K. Peters.  See HELIOS-note 154, (1986).

\bibitem{carman2}
D.S. Carman, ``CLAS Region I Prototype Detector", CLAS-Note 96-022,
(1996); The measured gain corresponds to that within a 200 ns time window.

\bibitem{carman}
D.S. Carman {\it et al.}, ``Hall B Test Run : Drift Chamber Studies", CLAS-Note
97-001, (1997).

\end{thebibliography}
