\section{Drift Chamber Electronics}
After the drift chambers were built and the wires strung, we mounted
on-chamber printed circuit boards onto the chamber endplates.
The signal connections to the wires are made through printed-circuit 
boards mounted along one side of each chamber, called Signal Translator 
Boards (STBs).  These boards are responsible for signal routing and 
pre-amplification.  

On the opposite side of each sector, another set of 
circuit boards called the High Voltage Translator Boards (HVTBs) is used 
to make the high-voltage connections to the wires (see the subsection
on gas, low voltage and high voltage for more details on the high
voltage system).

The drift chamber signal amplification and readout consists of the 
following subsystems:
\begin{itemize}
\item  chamber-mounted printed circuit boards with a single in-line package
(SIP) amplifier for each signal wire; these are the STB's
\item  chamber printed circuit boards which distribute high voltage
to all of the wires; these are the HVTB's
\item a single 17-pair twisted-pair readout cable for each group of 16
SIP's, and
\item a 96-channel drift chamber readout board (DCRB) for each group
of 6 cables (96 signal wires)
\end{itemize}


%%%%%%%%%%%%%%%%%%%%%%%%%%%%%%%%%%%%%%%%%%%%%%%%%%%%%%%%%%%%%%%%%%%%%%%%%
\begin{table}[htbp]
\begin{center}
\begin{tabular} {||c|c||} \hline \hline
{\bf Component}           & {\bf Number} \\ \hline
STB boards (6 types)      & 252 total \\ \hline
HVTB boards (6 types)     & 252 total \\ \hline
low-voltage cables        & 252 total  \\ \hline
high-voltage cables       & 252 total  \\ \hline
signal cables (17-pair)   & 1512 \\ \hline
total signals             & 24192 \\ \hline \hline
\end{tabular}
\caption{\small{Electronic channel counts for the readout, high voltage,
and low voltage.}}
\label{electronic-channels}
\end{center}
\end{table}
%%%%%%%%%%%%%%%%%%%%%%%%%%%%%%%%%%%%%%%%%%%%%%%%%%%%%%%%%%%%%%%%%%%%%%%%%

\subsubsection{On-Chamber PCBs and Cables}

The circuit boards that interface the sense 
wires to the pre-amplifiers (STBs) were  
designed for a 96-channel format.  This format allows for only seven 
circuit boards for each superlayer.   Fig.~\ref{r1stb} shows the traces 
routed from the signal pick-up at the plated-through holes to the signal 
inputs of the pre-amplifier.

%%%%%%%%%%%%%%%%%%%%%%%%%%%%%%%%%%%%%%%%%%%%%%%%%%%%%%%%%%%%%%%%%%%%%%%%%%%
\begin{figure}[htbp]
\vspace{10.0cm}
\special{psfile=r1stb.ps hscale=80 vscale=80 hoffset=30 voffset=-40}
\caption{\small{Trace routing shown on one of the Region~1 STBs being
designed.}}
\label{r1stb}
\end{figure}
%%%%%%%%%%%%%%%%%%%%%%%%%%%%%%%%%%%%%%%%%%%%%%%%%%%%%%%%%%%%%%%%%%%%%%%%%%%

The heart of the STB board is an individually packaged
single in-line package (SIP) preamplifer which was modified
from the design of the previous CLAS detectors and 
included an epoxy resin encapsulation.  
The encapsulation of the components was included to prevent 
component corrosion in a somewhat humid environment (relative
humidities as high as 60\%).
These ``CP01'' pre-amplifiers provide the gain, dynamic range, rise time, low 
noise, and low power needed for the performance requirements. 

Each SIP operates at 6V and draws about 13m Amps.   

Each group of 16 pre-amplifier output signals were routed to a 17 pair connector.
Sixteen of the pairs are used as differential signal paths which are routed from the STB's to the drift chamber readout boards (DCRB)
over individual cables consisting of 16 twisted pairs.  The seventeenth pair is used to send a calibration pulse from the DCRB
to the STB preamplifier group. We chose
twisted-pair readout because of its immunity to electronic noise.
The cables are round-jacketed with a 
0.025-in pitch so that the overall cable dimension is smaller than the 
standard 17-pair cable.  

\subsection{Off-Chamber Amplification, Time Digitization and Readout}

Our on-chamber 
pre-amplifiers send signals to the readout boards (DCRB) 
which provide another level of amplification, 
signal discrimination adjustable threshold setting, time digitization
and readout. 

These DCRB's are based on FPGA technology, and in addition to
their primary function of amplification, discrimination, digitization
and readout, they are used in a simple ``cluster-finding'' algorithm
to find track segment candidates with a latency of only hundreds
of nanoseconds.


More needed on DCRB's ...


\subsection{Low Voltage System}
We reused th low voltage power supplies that were used for {\tt CLAS}.  
The Hewlett Packard model HP6651A are linear DC power supplies with low ripple output, and high reliability.  
The supplies are remotely programmable through GPIB and EPICS and the output voltage and current are monitored.   We 
isolated the low voltage from 
ground loops by using local voltage regulators on the pre-amplifier interface 
boards (STBs).  The segmentation of the low voltage distribution cables is 
based on 32 preamplifier channels per supply cable.  Each of these supply 
cables is fused with an appropriate type of fuse rated for over-current 
protection based on 32 pre-amplifier loads.  

We designed our low voltage system: supplies, fusing, cables and control
system to be as robust and maintenance-free as possible.  To minimize
the damage to the tracking system in the event of a failure such as
a shorted pre-amplifier, we built in fine segmentation with only
32 preamplifier channels per supply cable.  Each of these supply 
cables is fused with an appropriate type of fuse rated for over-current 
protection based on this load.
In the event of a short circuit which causes a fuse to blow,
a simple, external cable disconnect will reduce the size of the affected
area to 16 signal wires without the need to access the chambers.

The on-chamber pre-amplifiers require 6V and approximately 18A per chamber
(a total of 1344 pre-amps per chamber).
The low voltage power supplies are re-used from the original CLAS detector.  
These units are robust and manufactured by Hewlett Packard.  
The supplies are remotely programmable and monitored.   We 
isolated the low voltage from 
ground loops, using local voltage regulators on the pre-amplifier interface 
boards (STBs).  

Fig.~\fig{dc-low-voltage-system} is a schematic of the low voltage
supply system and a snapshot 
of the control panel for monitoring the state of the system.

\subsection{High Voltage System}

As in the case of the low-voltage system, we designed our high voltage system: 
supplies, distribution boxes, cables and control system to be as robust and 
maintenance-free as possible.  
We re-used our CAEN 
system 527 high voltage supplies with somewhat finer segmentation than our 
previous system, consistent with our total channel count dropping from 34000 
to about 24000.
To minimize the damage to the tracking system in the event of a failure such as
a broken wire, we built in very fine segmentation.
Each individual high-voltage channel powers a variable-sized group of 
wires: a 48-wire group for wires in the small-angle region, a 96-wire group
in the middle-angle region and a 192-wire group at large angles.

In the event of a failure (e.g. a broken wire) which results in a trip
of a single HV channel, we can further reduce the size of the affected
area from the whole group (48, 96 or 192 wires) to a much smaller grouping
of 8, 16 or 32 wires by an external cable disconnect without the need to 
physically access the chambers themselves.

The high-voltage supply and distribution system consists of the following:
\begin{itemize}
\item a crate-based high voltage power supplies with 36 independent
high voltage channels for each drift chamber (1344 signal wires each).
Of these 36 channels, 16 supply positive high voltage to the sense
wires, 16 supply negative voltage to the field wires, and 4 supply
positive voltage to the guard wires
\item a series of two distribution boxes which distribute the high
voltage from the supplly channels on individual cables to groups
of wires, with the group size being 8 wires (for small angle wires)
to 16 (intermediate angles) to 32 (large angles). 
\item  on-chamber printed circuit boards which distribute high voltage
to all of the wires; these are the HVTB's
\end{itemize}



Fig.~\fig{dc-high-voltage-system} is a schematic of the high voltage
supply system and a snapshot 
of the control panel for monitoring the state of the system.

%%%%%%%%%%%%%%%%%%%%%% Figure : DC Sector Schematic %%%%%%%%%%%%%%%%%%%%%%
\begin{figure}
\vspace{4.5cm}
%\includegraphics[width=0.5\columnwidth]{dc-view.png}
\caption{\small{A schematic of the high voltage supply system and smapshot of 
the monitoring and control panel screen.}}
\label{dc-high-voltage-system}
\end{figure}   
%%%%%%%%%%%%%%%%%%%%%%%%%%%%%%%%%%%%%%%%%%%%%%%%%%%%%%%%%%%%%%%%%%%%%%%%%%




