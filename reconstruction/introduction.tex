\section{Introduction}

The paper describe the software framework, tools and algorithms that were developed in support of event reconstruction and analysis of the CLAS12 (CEBAF Large Acceptance Spectrometer at 12 GeV) experiment at Jefferson Lab (VA), USA~\cite{clas12-nim}. Installed in the experimental Hall B of the laboratory, CLAS12 is a large acceptance spectrometer based on two superconducting magnets and multiple detector subsystems to provide large coverage for the detection of charged and neutral particle produced by the interaction of the Jefferson Lab electron beam with a target located at the center of the spectrometer. A six-coil toroidal magnet defines the structure of the so-called Forward Detector, including drift chambers for charged particle tracking, threshold and Ring-Imaging Cerenkov Counters for particle identification, scintillator-based time of flights and electromagnetic calorimeters based on a lead-scintillator sandwich layout. In the target region, a 5-Tesla superconducting solenoid surrounds the central tracker based on silicon and MicroMegas detectors, a central time-of-flight system and a neutron detector, forming the so-called Central Detector. In between the central and forward region, the CLAS12 Forward Tagger, including a lead-tungstate calorimeter, a scintillation hodoscope and a tracker, extend the kinematic coverage for the detection of electrons and photons at very small angles. The total number of readout channels of CLAS12 is larger than 100000 for a typical data rate in production data taking of 400 MB/s (cross check with DAQ paper).  

The CLAS12 offline reconstruction and analysis framework was developed to cope with the complexity of the spectrometer and related data volumes. It consists of an extensive library of software tools, of detector reconstruction packages and of a framework to chain the reconstruction and analysis applications for data processing. Software tools are designed to support and standardize event reconstruction, detector calibration and monitoring, and data analysis, providing I/O functionalities, database access, detector geometry libraries, and magnetic field handling tools. These constitute the building blocks for the development of all CLAS12 detectors monitoring, calibration and reconstruction tools. Each reconstruction package is designed to extract from the raw data the relevant information for particle reconstruction such has tracks, hits or clusters. These are the input information for the CLAS12 Event Builder, which associates detector reconstructed outputs to identify particles and form the reconstructed event. The reconstruction components are deployed in a service-oriented platform, which provides the functionalities for data processing for both event reconstruction and the subsequent analysis.
While the software framework supports multiple programming languages, the CLAS12 reconstruction packages and tools currently in use are developed in JAVA.

This paper is organized as follows. The CLAS12 software framework and tools are described in Section 2. Raw and reconstructed data format are presented in Section 3. Monitoring, calibration and event display applications are described in Section 4 and 5. Section 6 provides a detailed description of the detector and event reconstruction packages. Section 7 and 8 present the data processing and code management procedures adopted for CLAS12.


