\subsection{Forward Tagger}

The Forward Tagger (FT)~\cite{ft-nim} is placed between the HTCC and the torus magnet along the beamline
and is designed to detect electrons and photons in the polar angular range from 2$^\circ$ to 5$^\circ$. The
FT is composed of an electromagnetic calorimeter based on PbWO$_4$ crystals (FT-Cal), a two-layer scintillator
hodoscope (FT-Hodo), and a Micromegas tracker (FT-Trk) similar in design to the FMT~\cite{mm-nim}. The FT
reconstruction service is designed to provide efficient algorithms to determine the energy, time, and positions of
the signals associated with the incident particle. The reconstruction matches this information to determine the type
and three-momentum of the particle. The package consists of four services, one for each of the sub-detectors and a
global service that builds the particle information from the output of the detector reconstruction. In the following,
we describe each of the FT services and their algorithms.

\subsubsection{The FT-Cal Reconstruction Service}

The calorimeter service has the role of reconstructing clusters associated with the incident particles from the
detector raw information. These include the charge and time recorded by the FADC boards that read out the
crystal signals. A cluster is defined as a contiguous ensemble of crystals within the calorimeter, in which a signals
above a minimum energy threshold (10~MeV) are found within a selected time window (10~ns) from each other.

The first step to build a cluster is to reconstruct the energy and time of the individual crystal hits from the raw
FADC information. For this purpose, the charge and raw time of the recorded pulse are converted to energy and
time using calibration constants derived from data. A linear relationship between energy and charge is assumed.
The hit time is defined from the raw time by applying an offset and a charge-dependent correction that accounts
for time-walk effects.

Reconstructed hits are then ordered by energy and, starting from the maximum energy hit, subsequent crystals
are associated with it based on their relative positions and time differences. Once all hits are associated with a
cluster, the overall cluster energy, time, and positions are computed. The cluster energy $E_{tot}$ is calculated as
the sum of the individual hit energies, $E_{rec}$, plus a global correction to account for the hit thresholds and for
shower leakages due to the finite length of the crystal and the overall calorimeter size. This correction is
parameterized as a function of the measured cluster energy based on full Geant4 simulations of the detector
response~\cite{ft-nim}. The cluster time is computed as the energy-weighted average of the individual hit times.
Finally, the cluster position in the $x-y$ plane (transverse to the beam $z$-axis) is computed as the logarithmic
energy-weighted hit coordinates $(x_i,y_i)$, i.e. the crystal position with the following functional form~\cite{ic}:
\begin{eqnarray}
x_{cluster} = \frac{\sum_{i=1}^N w_i x_i}{\sum_{i=1}^N w_i},\\
y_{cluster} = \frac{\sum_{i=1}^N w_i y_i}{\sum_{i=1}^N w_i},\nonumber
\end{eqnarray}

\noindent
where the index $i$ runs over the $N$ crystals in the cluster and the weighting factors $w_i$ are defined as:
\begin{equation}
w_i=max\left(0,w_0+ln(E_i/E_{rec}) \right).
\end{equation}
\noindent
The parameter $w_0$ was fixed to 3.45 after optimization based on Geant4 simulations. The $z$ coordinate of
the cluster is set to a constant depth from the crystal upstream face that was optimized based on Monte Carlo
studies.

The resulting clusters are finally selected by applying cuts to exclude instances where the total and seed energies
are less than a defined threshold or where the number of crystals in the cluster is below a defined
limit\footnote{Note that the seed crystal is the one with the largest signal}. All of these selection parameters, as
well as the other constants used in the cluster reconstruction, are set at run time by reading the CLAS12 calibration
constants database, CCDB.

The final list of clusters is saved to an output HIPO bank that is passed to the global FT reconstruction service for
the particle reconstruction. The intermediate hit information is also saved to a HIPO bank for debugging purposes.

\subsubsection{The FT-Hodo Reconstruction Service}

The FT-Hodo is used to discriminate photons and electrons. The system consists of two layers of plastic scintillator
tiles read out with silicon photomultipliers. The FT-Hodo reconstruction service, which is similar to that for the
FT-Cal, has the role of reconstructing hits and associating matching hits in the two layers of the detector to form
clusters. 

Hits are defined from the raw FADC information as the energy and time of the signals associated with the incident
particles. These are computed assuming a linear relation for the charge-to-energy conversion and an additive
offset between the raw and reconstructed time. The constants necessary for these conversions are derived
for each individual detector component based on beam-data calibrations as discussed in Ref.~\cite{ft-nim} and set
at run time by reading the values from CCDB. The reconstructed hits are then selected by applying a minimum energy
threshold that was optimized based on data analysis.

The selected hits are then matched to form clusters consisting of scintillator tiles in the two detector layers,
matched in space and time. The position matching distance is defined by the largest tile size, i.e. 3~cm, while the
time matching parameter was optimized based on Geant4 simulations and is set conservatively to 8~ns. The
resulting cluster parameters are the cluster size, position, total energy, and time. The cluster energy is calculated
as the sum of the individual hit energies, while both the position in the $x-y$ plane and time are calculated as the
energy-weighted average of the corresponding hit parameters. The resulting information is saved to a HIPO bank
that is passed to the global FT service. As for the calorimeter, the intermediate hit information is also saved to a
HIPO bank for debugging purposes.

\subsubsection{The FT-Trk Reconstruction Service}

The FT-Trk is used to measure the angle of the scattered electron. It consists of two double-layers of
Micromegas and is positioned upstream of the FT-Hodo. The FT-Trk reconstruction service is currently in the
development stage and will be described in detail in a future publication, while here we discuss only the general
principles. Algorithms for the conversion of the raw Micromegas detector information to hits and for matching
hits to form clusters follows those developed for the CLAS12 BMT that are discussed in Section~\ref{sec:cvt}.
All combinations of clusters identified in the $x-y$ layers of each of the two sub-detectors are then built to form
crosses. Finally, the  crosses found in the two sub-detectors are matched based on their position and saved as input
for the global FT service.

\subsection{The FT Global Service}

The final step of the FT reconstruction is the matching of the information resulting from the three sub-detectors.
Specifically, hodoscope and calorimeter clusters are matched to distinguish charged particles having a cluster in the
hodoscope from neutrals that have a low probability of creating a signal in that detector. The matching is based on
the relative position of the calorimeter and hodoscope clusters in the $x-y$ plane and on their time difference. The
position matching parameter is determined by the hodoscope component size, while the timing cut is set to 10~ns,
similar to the cut value used in the lower levels of the FT reconstruction. The output of the matching is an FT
{\it particle}, whose energy and position at the detector are determined from the calorimeter cluster parameters,
while its charge is set by the presence of a hodoscope cluster. The particle three-momentum at the target for charged
particles is then computed accounting for the bending in the solenoid field, while for neutrals it is computed assuming
a straight path from the CLAS12 target center to the FT. When available, the tracker information will be used to
refine the determination of the particle impact point on the FT front face and, therefore, to improve the
reconstruction of the angles at the vertex. The resulting particle information is saved to a HIPO bank for use in
the CLAS12 Event Builder service.
