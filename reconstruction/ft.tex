\subsection{Forward Tagger}

The reconstruction service for the Forward Tagger (FT)~\cite{ft-nim} is designed to provide efficient algorithms
to determine the energy, time, and positions of the signals associated with the incident particle. The FT consists
of three separate sub-detectors, a lead-tungstate calorimeter (FT-Cal), a two-layer scintillation hodoscope
(FT-Hodo), and a Micromegas tracker (FT-Trk). The reconstruction matches this information to determine the type
and 3-momentum of the particle. The package consists of four services, one for each of the sub-detectors and a global
service that builds the particle information from the output of the detector reconstruction. In the following, we
describe the role of each of the services and the algorithms that these implement.

\subsubsection{The FT-Cal Reconstruction Service}

The calorimeter service has the role of reconstructing clusters associated with the incident particles from the
detector raw information. These include the charge and time recorded by the FADC boards that read out the
scintillator crystal signals. A cluster is defined as a contiguous ensemble of crystals within the calorimeter, in which
a signal above a minimum energy threshold (10~MeV) is found within a selected time window (10~ns) from each other.

The first step to build a cluster is to reconstruct the energy and time of the individual crystal hits from the raw
FADC information. For this purpose, the charge and raw time of the recorded pulse are converted to energy and
time using calibration constants derived from data. As discussed in Ref.~\cite{ft-nim}, a linear relationship between
energy and charge is assumed. The hit time is defined from the raw time applying an offset and a charge-dependent
correction that accounts for the so-called time-walk effect.

Reconstructed hits are then ordered by energy and, starting from the maximum energy hit, the subsequent
crystals are associated to it based on their positions and relative time differences. Once all hits are associated
with a cluster, the overall cluster energy, time, and positions are computed as follows. The cluster energy $E_{tot}$
is calculated as the sum of the individual hit energies, $E_{rec}$, plus a global correction to account for the hit
thresholds and for shower leakages due to the finite length of the crystal and the overall calorimeter size. This
correction is parameterized as a function of the measured cluster energy based on full GEANT4 simulations of the
detector response~\cite{ft-nim}. The cluster time is computed as the energy-weighted average of the individual hit
times. Finally, the cluster position in the $x-y$ plane is computed as the logarithmic energy-weighted hit
coordinates $(x_i,y_i)$, i.e. the crystal position with the following functional form~\cite{ic}:

\begin{eqnarray*}
x_{cluster} = \frac{\sum_{i=1}^N w_i x_i}{\sum_{i=1}^N w_i},\\
y_{cluster} = \frac{\sum_{i=1}^N w_i y_i}{\sum_{i=1}^N w_i},
\end{eqnarray*}

\noindent
where the index $i$ runs over the $N$ crystals in the cluster and the weighting factors $w_i$ are defined as:
\begin{equation}
w_i=max(0,w_0+ln(E_i/E_{rec}).
\end{equation}

The parameter $w_0$ was fixed to 3.45 after optimization based on GEANT4 simulations. The $z$ coordinate of
the cluster is set to a constant depth from the crystal upstream face that was optimized based on Monte Carlo
studies.

The resulting clusters are finally selected by applying cuts to exclude instances where the total and seed energy
are less than a defined threshold or where the number of crystals in the cluster is below a defined limit. All of these
selection parameters, as well as the other constants used in the cluster reconstruction, are set at run time reading
the CLAS12 calibration constants database, CCDB.

The final list of clusters is saved to an output HIPO bank that is passed to the global FT service for the particle
reconstruction. The intermediate hit information is also saved to a HIPO bank for debugging purposes.

\subsubsection{The FT-Hodo Reconstruction Service}

Similarly to the FT-Cal, the FT-Hodo reconstruction service has the role of reconstructing hits and associating
matching hits in the two layers of the detector to form clusters. 

Hits are defined from the raw FADC information as the energy and time of the signals associated with the incident
particles. These are computed assuming a linear relation for the charge-to-energy conversion and an additive
offset between the raw and reconstructed time. The constants necessary for these conversions are derived
for each individual detector component based on beam-data calibrations as discussed in Ref.~\cite{ft-nim} and set
at run time reading the values from CCDB. The reconstructed hits are then selected applying a minimum energy
threshold that was optimized based on data analysis.

The selected hits are then matched to form clusters consisting of scintillator tiles in the two detector layers,
matched in space and time. The position matching distance is defined by the largest tile size, i.e. 3~cm, while the
time matching parameter was optimized based on GEANT4 simulations and is set conservatively to 8~ns. The
resulting cluster parameters, namely the cluster size, position, total energy, and time, are then computed as follows.
The cluster energy is calculated as the sum of the individual hit energy, while both the position in the $x-y$ plane and
time are calculated as the energy-weighted average of the corresponding hit parameters. The resulting information
is saved to a HIPO bank that is passed to the global FT service. As for the calorimeter, the intermediate hit
information is also saved to a HIPO bank for debugging purposes.

\subsubsection{The FT-Tracker Reconstruction Service}

The FT-Trk reconstruction service is currently in the development stage and will be described in detail in a
future publication, while here we discuss only the general principles. Algorithms for the conversion of the detector
raw information to hits and for matching hits to form clusters follows those developed for the CLAS12 MVT that
are discussed in Section~\ref{sec:cvt}. All combinations of clusters identified in the $x-y$ layers of each of the
two sub-detectors are then built to form crosses. Finally, the  crosses found in the two sub-detectors are matched
based on their position and saved as input for the global FT service.

\subsection{The FT Global Service}

The final step of the FT reconstruction is the matching of the information resulting from the three sub-detectors.
Specifically, hodoscope and calorimeter clusters are matched to distinguish charged particles having a cluster in the
hodoscope from neutrals that have a low probability of releasing a signal in that detector. The matching is based on
the relative position of the calorimeter and hodoscope clusters in the $x-y$ plane and on their time difference. The
position matching parameter is determined by the hodoscope component size, while the timing cut is set to 10~ns,
similar to the cut value used in the lower levels of the FT reconstruction. The output of the matching is a FT
{\it particle}, whose energy and position at the detector are determined from the calorimeter cluster parameters,
while its charge is set by the presence of a hodoscope cluster. The particle 3-momentum at the target for charged
particles is then computed accounting for the bend in the solenoid field, while for neutrals is computed assuming a
straight path from the CLAS12 target center to the FT. When available, the tracker information will be used to
refine the determination of the particle impact point on the FT front face and, therefore, to improve the
reconstruction of the angles at the vertex. The resulting particle information is saved to a HIPO bank for use of
the CLAS12 Event Builder service.
