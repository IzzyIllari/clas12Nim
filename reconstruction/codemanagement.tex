\section{Code Management}
\label{sec:manage}

\subsection{Repositories}

The software is managed in a github repository~\cite{recon-github}, and branches and forks are utilized to
accommodate parallel development by several groups.  Two main branches, {\it master} and {\it development}, are
utilized to store code ready for production and for validation, respectively. For the main branches, all modifications
are made through pull requests after passing the automated tests described in Section~\ref{sec:tests} and require
approval by a designated CLAS12 software expert.

\subsection{Releases}

There are three reconstruction code release types: test, validation, and production. A tagging scheme has been
implemented to indicate the type of change with respect to previous releases. Test releases, identified by the
letter ``c'', are tagged from branches other than the master or development branches and are intended to
validate a specific code change or algorithmic improvement. Usage of these releases is typically limited to the
developers. Validation releases, identified by the letter ``b'', are tagged from the development branch to test
code updates before merging to the master branch. Production releases are tagged from the master branch
after code updates for production data processing.

The release designator scheme uses the format $X(b/c).Y.Z$, where increments of $X$, $Y$, or $Z$ are applied
in the following cases:

\begin{itemize}
  \item $X$: introduction of new technology, major algorithmic improvements, or changes that are not backward
  compatible;
  \item $Y$: extension of interfaces, new implementations, or major bug fixes;
  \item $Z$: minor bug fixes.
\end{itemize}

\subsection{Code Tests and Validation}
\label{sec:tests}

In addition to automatic builds, the software includes both basic unit tests and advanced tests for several
packages. These are designed to verify the correctness and reproducibility of the reconstruction output for a
specific package or for the overall event, respectively. Unit tests involve, for example, reconstructing a simulated
track or particle hit in a specific detector and comparing the result to the truth information. Advanced and extended
tests are run on either a Monte Carlo or beam data sample, comparing to the Monte Carlo truth information in the first
case or to the results obtained in previous releases in the second case. A portion of the tests are run automatically
at build time, using the TravisCI system linked to the github repository.  These automatic tests take about 30
minutes to run and have proven invaluable in overseeing software development.

In addition to unit and advanced tests, every new release is subject to extensive validation on both Monte Carlo and
beam data. Samples of Monte Carlo and beam events for different beam energies and detector configurations were
chosen to test event reconstruction over the entire detector acceptance. Reconstruction of these samples is
performed and results are compared to previous code releases. The comparison focuses on several parameters, from
processing time, to momentum resolution, to particle reconstruction efficiency. A new release is accepted for
production only if it results in globally improved event reconstruction performance. 
