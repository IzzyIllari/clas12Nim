\subsection{Electromagnetic Calorimeters}

The (Forward Electromagnetic Calorimeters) FEC~\cite{ec-nim} reconstruction service provides a fast and efficient algorithm for grouping scintillator {\it strips} with
{\it hits} into multiple {\it peaks} and {\it clusters} within a single sector for each of the FEC modules, PCAL, ECIN,
and ECOU, while leaving cluster matching and particle ID to the event builder service.  Within the FEC service, these
various elements exist as objects with methods, structures and data members designed for calibration, pattern
recognition, diagnostics and serial output.  For example the service applies run dependent calibration corrections for
conversion of raw FADC and TDC digitized data to energy and time, and also provides formatted output banks used by
external services.  Also energy thresholds and cluster identification criteria can be configured to optimize reconstruction
efficiency, suppress backgrounds and avoid false or duplicate clusters arising from fluctuations at the fringes of EM
showers.

The cluster finding algorithm makes use of the unique geometry and stereo readout features of the FEC. As discussed
in REf.~\cite{ec-nim}, each triangular scintillator layer in the FEC lead:scintillator sandwich is transversely divided into strips, with the
shortest strip at the corners. The slice direction rotates by $\approx~120^{\circ}$ for each successive layer, providing
three {\it views} labeled $U$, $V$, and $W$.  For each strip within a view, layers are optically ganged together into a
stack.  Individual PMT readout of each PCAL, ECIN, and ECOU stack provides a pulse proportional to the summed energy
deposited in the stack.

The algorithm begins by finding collections of contiguous strips having signals above a user-defined threshold for each of
the three views. These groupings are called {\it peaks} and their member strips are referred to as {\it hits}.  Peak objects
may be further subdivided based on the hit energy profile of the groupings.  Each peak object is associated with the one or
more stacks of strips which belong to it, and the three-dimensional geometry of each stack is stored along with the peak
data. The service uses this geometry data to determine which collection of peaks belong to $\it{clusters}$.

\subsubsection {Cluster Position}

The criterion for defining a cluster requires the spatial intersection of three peaks, one from each of the $U$, $V$, and $W$ views.
Candidate peaks for a cluster search are based on a user-defined threshold for the summed peak raw energy.  Each peak is
represented geometrically as a directed line segment determined by the energy weighted average of the mid-lines of each
member strip.   The degree of intersection of each $U$, $V$, $W$ peak triplet is determined by calculating the line of
closest distance between a $U$ and $V$ peakline, followed by the line of closest distance between the midpoint of the $UV$
line and the $W$ peakline.  A user-defined cut on this final $UV$-$W$ distance identifies the cluster, and the midpoint of
the $UV$-$W$ line defines the $(x,y,z)$ coordinates of the cluster.

\subsubsection {Cluster Energy}

Once the cluster is localized, the path from the cluster position to the PMT readout end is calculated for each $U$, $V$,
$W$ peakline and the peak energies are corrected for scintillator light attenuation.  For isolated clusters the cluster energy
is then defined as the sum of the corrected energy from each of the $U$, $V$, and $W$ peaks which define the cluster.

More complicated scenarios arise from the triangular geometry of the FEC hodoscope, which creates the possibility of a
single peak in the $U$, $V$, or $W$ view sharing the summed energy from two or more clusters.  For these cases the energy
in each cluster which shares that peak is assumed to be proportional to the relative partial energies of the multiple clusters as
measured in the other views.  For example, if there are two clusters, both of which share the same $U$ peak, the summed
energy $V+W$ is determined for each of the clusters, and the ratio of these summed energies determines how much of the
$U$ peak energy is assigned to each of the two clusters.

Finally the clusters to be reported to external services are selected with a user-defined energy cut, and these clusters
are sorted according to energy. Typical software thresholds applied at the strip, peak and cluster level are 1, 3, and 10~MeV,
respectively.

\subsubsection {Cluster Time}

Once the cluster is localized, the path from the cluster position to the PMT readout end is calculated for each $U$, $V$,
$W$ peakline and the peak timing is corrected for propagation delay of the light, using the effective velocity of light
determined for each scintillator from the calibration procedure.  For isolated clusters the cluster timing is then taken from
the $U$, $V$, or $W$ peak with the largest uncorrected raw ADC value.  The minimizes the effect on the timing resolution
from both the time-walk correction and the photoelectron statistical fluctuations.
