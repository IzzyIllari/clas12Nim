\subsection{Central Neutron Detector}
\label{sec_rec_cnd}

The Central Neutron Detector (CND)~\cite{cnd-nim} is used to detect 0.2 to 1~GeV neutrons in the Central Detector.
The CND consists of a barrel of three layers of scintillators coupled at their downstream ends with U-turn light
guides and read out on their upstream ends with PMTs. The light readout from the scintillation bar in which a particle
interacts is called ``direct'', while the light that travels through the U-turn into the neighboring bar and read out
in the coupled counter is called ``indirect''. 

The reconstruction of the CND is done in five steps:

\begin{itemize}
\item the choice of the direct and indirect paddle, by comparing the two PMT times (referred to as the left and
  right times) of a coupled pair of counters, after correcting them for relative and absolute offsets determined in
  the calibration procedure and accounting for light propagation times \cite{cnd-nim};
\item the reconstruction of the deposited energy;
\item the reconstruction of the time and position of the hit in the paddle;
\item the matching of CND hits with CVT tracks coming from the interaction vertex;
\item the clustering of multiple hits.
\end{itemize}

\subsubsection{Energy Reconstruction}

For direct hits in the left paddle at a position $z$ along the paddle, the two associated ADCs can be written as:
\begin{equation}
\label{eq_adc}
ADC_L = \frac{E_L}{E_0} \cdot MIP_D \cdot e^{\frac{-z}{A_L}},
\end{equation}
\begin{equation}
ADC_R = \frac{E_R}{E_0} \cdot MIP_I \cdot e^{\frac{-(L-z)}{A_L}}.\nonumber
\end{equation}

\noindent
Here $MIP_D$ ($MIP_I$) is the ADC-to-energy constant for direct (indirect) minimum-ionizing particles (MIPs),
$E_{L/R}$ is half the energy deposited by the particle in the left/right paddle, $z$ is the distance along the
left counter to the left PMT, $L$ is the length of each paddle, and $A_L$ is the coupled counter pair attenuation
length. $E_0$ is given by:

\begin{equation}
\label{eq_def_e0}
E_0=\frac{h\cdot 2.001}{2}~{\rm MeV},
\end{equation}

\noindent
where $h$ is the thickness of each scintillator. In the case of direct hits in the right paddle, the applicable equations
are obtained by switching the $L/R$ indices. The energy reconstruction for each coupled paddle is obtained inverting
Eq.~\ref{eq_adc}. The total energy of the hit is then given by the sum of $E_L$ and $E_R$.

\subsubsection{Hit Position and Time Reconstruction}

The reconstruction of the time and position of a hit will be shown for the case of a hit in the left paddle. In case
of a hit in the right paddle, the applicable equations are obtained by switching the $L/R$ indices.

Starting from $t_L$ and $t_R$, defined as
\begin{equation}
\label{eq_time_hit}
t_L = t_{tof} +\frac{z}{v_{eff_L}} +t_S + t_{off} + TDC_j,
\end{equation}
\begin{equation}
t_R = t_{tof} + \frac{L-z}{v_{eff_L}} + \frac{L}{v_{eff_R}} + u_t + t_S + t_{off} + TDC_j,\nonumber
\end{equation}

\noindent
and subtracting the time offsets obtained from the calibration ($t_{off}$), the start time ($t_S$), and the time
jitter ($TDC_j$), one can define the propagation times $t_{L_{prop}}$ and $t_{R_{prop}}$ to the left and right
PMTs of the coupled pair as:

\begin{equation}
t_{L_{prop}} = t_{tof} + \frac{z}{v_{eff_L}},
\end{equation}
\begin{equation}
t_{R_{prop}} = t_{tof} - \frac{z}{v_{eff_L}} + \frac{L}{v_{eff_L}} + \frac{L}{v_{eff_R}} + u_t,\nonumber
\end{equation}

\noindent
where $v_{eff_{L/R}}$ is the effective light velocity in the left/right paddle and $u_t$ is the propagation time of
light to travel in the U-turn. Both of these quantities are obtained from CND calibration (see Ref.~\cite{cnd-nim}
for details).

The position of the hit $z$ is obtained from the difference of the left and right propagation times:
\begin{multline}
  z = \frac{v_{eff_L}}{2} \left(t_{L_{prop}} - t_{R_{prop}} \right) + \\ \frac{v_{eff_L}}{2} \left( L \cdot \left(\frac{1}{v_{eff_L}}
  + \frac{1}{v_{eff_R}}\right) + u_t \right).
\end{multline}

The $x$ and $y$ coordinates of the hit are obtained from the radius and the azimuthal angle of the hit, which are,
in turn, determined by knowing the layer, sector, and component (left or right) of the hit. Finally, the time of flight
of the particle that produced the hit is obtained from the sum of the left and right propagation times:
\begin{multline}
  t_{tof} = \frac{1}{2} \left( t_{L_{prop}} + t_{R_{prop}} \right) - \\
  \frac{1}{2} \left( L \cdot \left( \frac{1}{v_{eff_L}} + \frac{1}{v_{eff_R}} \right)
  -u_t \right).
\end{multline}

\subsubsection{Hit/Track Matching}

Tracks from charged particles crossing the CVT are associated with hits in the CND. This allows the position of
each CND hit to be computed from the track extrapolated beyond the CVT to the location of the hit counter. This
information is used in the detector calibration~\cite{cnd-nim}. CVT tracks are extrapolated to radii corresponding
to the entry point, midpoint, and exit point of the track in the paddle. These points are defined as the
intersections between the helix of the track and cylinders of radii corresponding to the distances between the
beamline and the three CND layers. A CVT track and a CND hit are matched if the hit coordinates and the
extrapolated coordinates are within a user-selected distance. The path traveled by the particle in the paddle is
approximated as the distance between the entry and exit points. The path length between the vertex and the hit is
obtained from the helix parameters.

\subsubsection{Clustering}

The clustering of CND hits is based on the geometrical space-time distance between them. The determination of
the maximal distance for clustering two hits together takes into account the measured resolutions for position and
timing of the CND counters~\cite{cnd-nim}.

The algorithm uses standard hierarchical clustering~\cite{Day1984}. A scan of all hits in an event is performed and
only hits with a deposited energy greater than 1~MeV are considered for clustering. The two closest hits are
combined into a single hit with associated energy defined as the sum of the energies of both hits. The position and
timing of the cluster hit are defined as those of the hit with the highest energy, i.e. the seed hit. The same algorithm
is recursively run on the remaining hits. Finally, the leftover hits that are relatively far from each other are called
clusters. The sector, layer, and component of each cluster are those of the seed hit.

The reconstructed scattering angles ($\theta$ and $\phi$) and momentum for neutral hits in the CND are computed
as done for the ECAL, see Section~\ref{ec-cluster} and Section~\ref{ec-time}, respectively.
