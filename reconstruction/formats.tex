\section{Data Formats}
EvIO (Event Input-Output)~\cite{evio} is a data format designed by and maintained by the
Jefferson Laboratory Data Acquisition Group, and is the data format of the raw data.
For data reconstruction and analysis, the CLAS12 data format was designed to provide
a flexible data container structure,
with features that minimize disk
access for the most common tasks performed in data analysis.
The High Performance Output (HiPO)
format developed for CLAS12 was designed to provide data compression, using LZ4
(the fastest compression algorithm currently available),
and random access.

HiPO stores data in separate records (with adjustable size), with tags associated with each record. Each record is
compressed and a pointer to the record is kept in the file's index table. This feature allows separating events during
reconstruction based on the content of the event, such as number of reconstructed particles. Users can read portions
of the file depending on the final states to be analyzed.  The meta-data of the file, describing detector and
beam conditions are common for all analysis.

A C++ interface is also provided with ROOT wrappers, allowing users to read HiPO files and analyze them in ROOT
environment.

