\section{Data Formats}
\label{sec-formats}

EVIO (Event Input-Output)~\cite{evio} is a data format designed and maintained by the JLab Data Acquisition
Group, and is the data format of the raw data. For event reconstruction and analysis, the CLAS12 data format
was designed to provide a flexible data container structure, with features that minimize disk access for the
most common tasks performed in data analysis. The High Performance Output (HIPO) format developed for
CLAS12 was designed to provide data compression, using LZ4 (the fastest compression algorithm currently
available), and random access.

HIPO stores data in separate records (with adjustable size), with tags associated with each record. Each record
is compressed and a pointer to the record is kept in the file's index table. This feature allows separating events
during reconstruction based on the content of the event, such as the number of reconstructed particles. Users can
read portions of the file depending on the final states to be analyzed.  The meta-data of the file, describing detector
and beam conditions, are common for all analyses.

The HIPO library has both Java and C++ implementations. On the basis of the C++ implementation, a library was
developed extending ROOT base classes to allow for HIPO files to be read from ROOT frameworks. Additional
tools are available to allow users to produce plots using native ROOT syntax.
