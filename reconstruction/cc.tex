\subsection{Threshold Cherenkov Counters}

The goal of the reconstruction algorithms for the High Threshold Cherenkov Counter (HTCC)~\cite{htcc-nim} and
the Low Threshold Cherenkov Counter (LTCC)~\cite{ltcc-nim} is to calculate the signal strength, time, and position
from the raw ADC signals coming from the PMTs to the corresponding FADC boards. The algorithm takes into
account the properties of the HTCC and LTCC geometries, namely, the possibility for the signal from the single
track to split into up to four mirrors. Hence, up to four separated signals (or hits) are produced. The final signal
reconstruction is done in three steps: decoding, hit reconstruction, and cluster reconstruction. For each hit,
the signal strength ($nphe_{hit}$ - the number of photoelectrons) is determined from the pedestal-subtracted
integral of the FADC pulse and the associated time ($T_{hit}$) is determined from a fit of the position of the FADC
signal threshold crossing time.

At the hit reconstruction stage, individual signals in terms of the ADC channels are converted into the number of
photoelectrons ($nphe_{hit}$) for each hit using gain constants derived from the detector calibration and stored
in CCDB:

\begin{equation}
nphe_{hit} = \frac{ADC}{gain}.
\end{equation}

\noindent
Geometry information on the PMT location is used to associate the angular coordinates ($\theta_{hit}$, $\phi_{hit}$)
to the hit.

In order to reconstruct the real signal strength ($nphe_c$), split signals (hits) have to be combined into a single
cluster. The algorithm starts by selecting the hit with the largest $nphe_{hit}$, which is used as a seed for the
cluster. Adjacent hits within a certain time window are then searched iteratively and, if found, added to the
cluster. The total signal strength is determined as the sum of the individual signals, and the signal time is
determined as the average between the individual signal times, weighted by the corresponding number of
photoelectrons. The cluster angular coordinates are determined as the average between the individual hits forming
the cluster. The following equations summarize the definitions of the cluster quantities:

\begin{eqnarray}
nphe_c &= \frac{\sum_{i=1}^N{nphe_{hit}}}{N}\\
T_c &= \frac{\sum_{i=1}^N{N*T_{hit}}}{\sum_{i=1}^N{nphe_{hit}}}\\
\theta_c &=\frac{\sum_{i=1}^N{\theta_{hit}}}{N}\\
\phi_c &= \frac{\sum_{i=1}^N{\phi_{hit}}}{N}
\end{eqnarray}

\noindent
The clustering algorithm is run iteratively until the full list of hits is exhausted.

In the HTCC, the cluster coordinates, required for the matching of the hit with the reconstructed track in the
Event Builder, are reconstructed by projecting  ($\theta_c$, $\phi_c$) of the cluster on the surface of the
ellipsoidal mirror of the detector. In the LTCC, an estimated cluster position is calculated based on a
parameterization extracted from Monte Carlo simulations. The track that passes the closest to the cluster
position is then chosen as the match for this cluster.
