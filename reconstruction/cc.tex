\subsection{Threshold Cherenkov Counters}

The CLAS12 Forward Detector includes two threshold Cherenkov detectors for particle identification. The
High Threshold Cherenkov Counter (HTCC)~\cite{htcc-nim} is located upstream of the torus and is used for
identification of the scattered electron in conjunction with the ECAL. The Low Threshold Cherenkov Counter
(LTCC)~\cite{ltcc-nim} is positioned upstream of the FTOF and is used mainly to {\color{red} identify pions}.
Both the HTCC and LTCC are large gas-filled volumes (CO$_2$ for the HTCC, C$_4$F$_{10}$ for the LTCC)
with mirrors that direct light collection to the PMTs. The goal of the HTCC and LTCC reconstruction algorithms
is to calculate the signal strength, time, and position from the raw ADC signals (read out with flash ADC boards
- FADCs). The algorithm takes into account the properties of the HTCC and LTCC geometries, namely, the
possibility for the signal from a single charged track to split into up to four mirrors. Hence, up to four separate
signals (or hits) are produced. The final signal reconstruction is done in three steps: decoding, hit reconstruction,
and cluster reconstruction. For each hit, the signal strength ($nphe_{hit}$ - the number of photoelectrons) is
determined from the pedestal-subtracted integral of the FADC pulse and the associated time ($t_{hit}$) is
determined from a fit of the position of the FADC signal threshold crossing time.

At the hit reconstruction stage, individual signals in terms of the ADC channels are converted into the number of
photoelectrons ($nphe_{hit}$) for each hit using gain constants derived from the detector calibration and stored
in CCDB:

\begin{equation}
nphe_{hit} = \frac{ADC}{gain}.
\end{equation}

\noindent
Geometry information on the PMT location is used to associate the angular coordinates ($\theta_{hit}$, $\phi_{hit}$)
to the hit.

In order to reconstruct the real signal strength ($nphe_c$), split signals (hits) have to be combined into a single
cluster. The algorithm starts by selecting the hit with the largest $nphe_{hit}$, which is used as a seed for the
cluster. Adjacent hits within a certain time window are then searched iteratively and, if found, added to the
cluster. The total signal strength is determined as the sum of the individual signals, and the signal time is
determined as the average between the individual signal times, weighted by the corresponding number of
photoelectrons. The cluster angular coordinates are determined as the average between the individual hits forming
the cluster. The cluster quantities are defined by:

\begin{eqnarray}
nphe_c &= \frac{\sum_{i=1}^N{nphe_{hit}}}{N} \nonumber \\
t_c &= \frac{\sum_{i=1}^N{N*t_{hit}}}{\sum_{i=1}^N{nphe_{hit}}} \nonumber \\
\theta_c &=\frac{\sum_{i=1}^N{\theta_{hit}}}{N} \nonumber \\
\phi_c &= \frac{\sum_{i=1}^N{\phi_{hit}}}{N}.
\end{eqnarray}

\noindent
The clustering algorithm is run iteratively until the full list of $N$ hits is exhausted.

In the HTCC, the cluster coordinates, required for the matching of the hit with the reconstructed track in the
Event Builder, are reconstructed by projecting  ($\theta_c$, $\phi_c$) of the cluster on the surface of the
ellipsoidal mirror of the detector. In the LTCC, an estimated cluster position is calculated based on a
parameterization extracted from Monte Carlo simulations. The track that passes the closest to the cluster
position is then chosen as the match for this cluster.
