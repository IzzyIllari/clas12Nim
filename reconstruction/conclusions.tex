\section{Conclusions}

We have presented the software framework and event reconstruction that are currently being utilized for the processing of data collected by the CLAS12 experiment, in Hall B at Jefferson Lab.  

The framework was developed to allow processing of CLAS12 data for reconstruction and analysis purposes with an elastic, eclectic, and expandable approach thanks to the service-oriented architecture. The specific software applications leverage on an extensive set of common libraries for handling I/O, geometry, databases and magnetic field, designed to support data monitoring, calibration, reconstruction and analysis.

Full event reconstruction is implemented in the framework as a chain of micro-services that perform reconstruction of the individual CLAS12 sub-systems and whose output information is collected by the Event Builder service to form and identify particles. While the current reconstruction chain already supports reconstruction of all the sub-system and creation of full events, upgrades to the existing software implementation and algorithms are being studied.