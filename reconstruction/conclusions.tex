\section{Conclusions}

We have presented the software framework and event reconstruction algorithms that are currently being utilized
for the processing of data collected by the CLAS12 experiment in Hall~B at Jefferson Lab. The framework was
developed to allow processing of CLAS12 data for reconstruction and analysis based on a service-oriented
architecture. The specific software applications leverage an extensive set of common libraries for handling I/O,
geometry, databases, and magnetic field that are designed to support data monitoring, calibration, reconstruction,
and analysis.

Full event reconstruction is implemented in the framework as a chain of micro-services that perform reconstruction
of the individual CLAS12 subsystems and whose output information is collected by the Event Builder service to form
and identify particles. While the current reconstruction chain already supports reconstruction of all subsystems
and the creation of full events with performance consistent with expectations, upgrades to the existing software
implementation and algorithms are under study. However, the current status of event reconstruction based on
data collected during the first production data runs with CLAS12 with the electron beam are reported and
discussed in detail in Ref.~\cite{clas12-nim} that show the efficacy of the developed reconstruction framework,
common tools and detector calibration applications, and the associated algorithms required for event reconstruction.
