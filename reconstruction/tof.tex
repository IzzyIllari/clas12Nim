\subsection{Time-of-Flight Systems}

For the CLAS12 Forward Time-of-Flight system (FTOF)~\cite{ftof-nim} and Central Time-of-Flight system
(CTOF)~\cite{ctof-nim}, the raw data from the detector elements readout during data acquisition that are
associated with a charged or neutral hit, include an ADC charge and hit time from a fitted flash ADC waveform
and a TDC time. The ADC and TDC information is recorded only for hits that are above the hardware readout
threshold. For hits in the FTOF and CTOF, the hardware thresholds for the flash ADC and discriminators are set
to be $\sim$1~MeV. During event reconstruction the ADC and TDC information are converted into deposited energy
and timing information after completion of the detector calibration procedures (see Refs.~\cite{ftof-nim,ctof-nim}
for calibration details). The measured hit times in each counter are related to the event start time defined with
respect to the appropriate RF beam bucket to determine the charged or neutral particle flight time.

\subsubsection{Reconstructed Hit Time}
\label{rec:time}

The reconstructed scintillation bar hit times need to account for the time delays along the readout path that
include the PMT signal transit time, the signal propagation times through the signal cables and the electronics,
and any time-walk effects associated with the readout discriminators. For the FTOF readout, leading-edge
discriminators are employed, while for the CTOF readout constant fraction discriminators are employed and no
external time-walk corrections are required. The hit times reconstructed by the readout through the PMTs at
the left and right ends of each scintillation bar are given by:

\begin{equation}
t_{L/R} = (C_{TDC} \cdot TDC_{L/R}) - t_{L/R}^{walk} - \frac{C_{LR}}{2} + C_{p2p},
\end{equation}

\noindent
where $C_{TDC}$ is the TDC channel-to-time conversion factor (0.024~ns/bin), $TDC$ is the measured TDC value
relative to the trigger signal, $t^{walk}$ is the time walk function to correct for the pulse amplitude dependence of
the crossing times of the discriminator threshold, $C_{LR}$ is a time offset to center the left-right TDC difference
distribution about 0, and $C_{p2p}$ is a time offset to account for the different delay times along the signal path from
the photocathode of the PMT to the readout electronics. The paddle-to-paddle time offsets $C_{p2p}$ mainly account
for the different signal cable lengths.

The FTOF and CTOF particle hit times relative to the trigger signal can be determined separately from the times
$t_L$ and $t_R$ measured by the left and right PMTs of a given scintillation bar using:

\begin{equation}
t_{hit}^{L/R} = t_{L/R} - \frac{d_{L/R}}{v_{eff}},
\end{equation}

\noindent
where $d_{L/R}$ represents the distances along the bar from the hit point to the left and right PMTs given by
$d_{L/R}= L/2 \pm y$ with $y$ the hit coordinate along the bar and $L$ the counter length. The average hit time
is given by:

\begin{equation}
\bar{t}_{hit} = \frac{1}{2} ( t_{hit}^L + t_{hit}^R ) = \frac{1}{2} \left[ t_L + t_R - \frac{L}{v_{eff}} \right],
\end{equation}

\noindent
where $v_{eff}$ is the effective speed of light in the scintillation bar.

The hit coordinate along the bar is defined with respect to the center of the bar using:

\begin{equation}
\label{tof-coor}
y = \frac{v_{eff}}{2} (t_L - t_R - y_{offset}),
\end{equation}

\noindent
where $y_{offset}$ is an additional coordinate shift to center the coordinate distribution about zero.

In addition to the hit time determined from the TDC information associated with the left and right PMTs
from each counter of the FTOF and CTOF systems, a time is also derived from fitting the leading edge of
the flash ADC pulse shape. Due to the choice of fast timing PMTs for the detector readout and the use of
250~MHz FADCs, the number of samples on the leading edge of the PMT pulses is only 3 to 4, hence
the timing resolution is only $\sim$1~ns. In the event reconstruction of FTOF and CTOF hits, however, we
require that the hit times determined from the TDCs and from the FADCs match to within 10~ns to
reduce the probability of a mismatch of the ADC and TDC data for a given scintillation bar hit.

For CLAS12 flight time measurements, the event start time is determined by the FTOF system using the
path length $L$ determined from the forward drift chamber tracking by comparing the FTOF hit time
traced back to the vertex and linking this time to the closest RF pulse from the 499~MHz RF signal from
the accelerator. The timing residuals

\begin{equation}
t_{res} = mod \left [ \left (\bar{t}_{hit} - \frac{L}{\beta c} \right) - \left (t_{RF} + \frac{z_v}{\beta_e c} \right),
T_{RF} \right ]
\end{equation}

\noindent
are centered about zero. The reconstruction algorithm determines the event start time looking first for an
$e^-$ and then an $e^+$ in the ECAL. If none exists, then it looks for a high momentum $\pi^+$ or $\pi^-$. In
each case, if multiple candidates exist, the track with the highest momentum is chosen. In the expression for
$t_{res}$ above, $z_v$ represents the reconstructed event vertex from traceback of the drift chamber track
to the distance of closest approach to the defined beamline and $T_{RF}$ is the RF period. The term
$z_v/\beta_e c$ shifts the RF reference time to the event vertex. The modulus is employed as the RF time is a
pulse train with a period of $T_{RF}$. For the CTOF reconstruction, the flight time is computed as
$\bar{t}_{hit} - t_{ST}$.

\subsubsection{Reconstructed Hit Energy}
\label{rec:energy}

The reconstructed energy from the left and right ADC values of the PMTs for a given scintillator bar is given by:

\begin{equation}
E_{L/R} = (ADC_{L/R} - PED_{L/R}) \left [ \frac{\left( \frac{dE}{dx} \right )_{MIP} \cdot t}{ADC_{MIP}} \right ],
\end{equation}

\noindent
where $(ADC  - PED)$ is the measured pedestal-subtracted ADC integral, $ADC_{MIP}$ is the ADC value for normally
incident minimum-ionizing particles (MIPs) at the center of the scintillation bar, $\left( \frac{dE}{dx} \right)_{MIP}$
is the energy loss for MIPs in the scintillation bars (1.956~MeV/cm), and $t$ is the scintillation bar thickness.
The deposited energy is computed as the geometric mean of the deposited energy as determined from the left and
right PMTs $E_L$ and $E_R$ as $E_{dep} = \sqrt{E_L E_R}$.

\subsubsection{Hit Clustering and Matching}

A charged track incident upon the FTOF or CTOF can cross more than one scintillation bar. If there are multiple
scintillation bar hits associated with a single incident charged particle track, a hit cluster can be defined. These
clusters have associated with them a hit coordinate, energy, and hit time.

The $(x,y,z)$ coordinates assigned to a scintillation bar hit are given at the middle of the bar across its width
and depth, and from eq.(\ref{tof-coor}) along the counter length. Hits are assigned as part of a cluster if they
fall within a matching distance of the charged track projected from the drift chambers to the FTOF counters
using a straight line extrapolation or from the Central Vertex Tracker to the CTOF counters following the
helical trajectory in the solenoid field. This matching distance is $\pm$1 counter about the hit scintillation bar
and $\pm$10~cm along the counter length.

With hit clusters defined, the associated cluster coordinate is defined as the energy-deposited weighted average.
The deposited energy of the cluster hit is the energy sum of the cluster hits and the hit time is the energy-deposited
weighted average. Note that in both the FTOF and CTOF systems, the maximum cluster size is $N=2$.

For the FTOF system in the range of polar angles from 5$^\circ$ to 30$^\circ$ where two parallel scintillation
bar planes are included (referred to as panel-1b - closest to the target, and panel-1a - farthest from the target)
(see Ref~\cite{ftof-nim}), the hit time is assigned as the time resolution weighted average of the two cluster times
after evolving the panel-1a hit to the location of the panel-1b hit using the path length determined from the
extrapolated drift chamber track. The combined plane hit time is given by:

\begin{equation}
t_{comb} = \frac{ t_{1b}^{cluster} {\delta_{1b}}^{-1} + (t_{1a}^{cluster} - \Delta r/\beta) {\delta_{1a}}^{-1}}
{\left( {\delta_{1b}}^{-1} + {\delta_{1a}}^{-1} \right)},
\end{equation}

\noindent
where $\delta_{1b/1a}$ are the measured counter effective time resolutions, $t_{1b/1a}^{cluster}$ are the cluster
hit times, and $\Delta r/\beta$ is the path length from the panel-1a hit location to the panel-1b hit location for
the track of speed $\beta$.

As the effective FTOF time resolutions for the panel-1b counters are in the range from $60-110$~ps and those
for the panel-1a counters are in the range from $90-160$~ps, the combined plane hit time resolution is $\sim$20\%
better than that for the panel-1b hit time resolution. If there is a panel-1b (panel-1a) hit without an associated
panel-1a (panel-1b) hit, the hit time is defined solely by the panel-1b (panel-1a) hit.