\subsection{Time-of-Flight Systems}
\label{tof-sys}

The time-of-flight (TOF) detectors for CLAS12 include the Forward Time-of-Flight system (FTOF)~\cite{ftof-nim}
and Central Time-of-Flight system (CTOF)~\cite{ctof-nim}. The FTOF consists of planes of scintillator counters
located between the RICH/LTCC and the ECAL. Two parallel counter arrays in each Forward Detector sector are
employed to achieve the desired time resolution. The arrays are referred to as panel-1b (closer to the target) and
panel-1a (farther from the target). The different FTOF arrays can be seen in Fig.~\ref{fig:dcTracks}. The CTOF
consists of a barrel of scintillator counters located just outside of the CVT within the solenoid.

The raw data from the detector PMTs read out during data acquisition include an ADC charge and hit time from a
fitted flash ADC waveform and a TDC time. The ADC and TDC information is read out and recorded only for those
channels that are above the $\sim$1~MeV hardware readout threshold for the flash ADCs and discriminators of
both systems. 

As the FTOF and CTOF counters employ double-ended PMT readout, the calibration procedures for these
systems (described in detail in Refs.~\cite{ftof-nim,ctof-nim}) allow the reconstruction to report accurate hit
times and deposited energy associated with both PMT signals above threshold. At this point the event reconstruction
combines the PMT hit times and energies to give a hit time and energy deposition associated with the scintillation
counter. In a second phase, hits in adjacent counters, due to particles that pass through multiple counters in the
FTOF and CTOF systems (so-called ``corner clippers''), are combined into clusters with an associated time,
coordinate, and deposited energy. The algorithms for the hit and cluster definition are detailed in the next sections.

\subsubsection{Raw Counter Hits}

Raw hits for the TOF systems are defined by matching ADC and TDC information reported for each counter. These
are matched based on the comparison of TDC time with the time from the FADC waveform analysis. The latter is
derived from fitting the leading edge of the FADC pulse shape. Due to the choice of fast timing PMTs for the
detector readout and the use of 250~MHz FADCs, the number of samples on the leading edge of the PMT pulses is
only 3 to 4, hence the FADC timing resolution is only $\sim$1~ns. The FADC and TDC times are then required to be
within a selected window. The windows parameters, position and width, as well as all other constants used by the
reconstruction package are loaded at run time from CCDB. Currently the window width used is 10~ns, which was
found to be sufficient to reduce the probability of a mismatch of the ADC and TDC data for a given scintillation bar
hit. This is especially important for the FTOF as the ADC value of the hit is used to compute the time-walk
correction.

\subsubsection{Reconstructed Counter Hits}
\label{rec:hits}

Raw hits are processed to determine reconstructed hits with energy, time, and position information.  The
reconstructed hit times from the individual PMTs need to account for the time delays along the readout path
that include the PMT signal transit time, the signal propagation times through the signal cables and the electronics,
and any time-walk effects associated with the readout discriminators. For the FTOF readout, leading-edge
discriminators are employed, while for the CTOF readout, constant fraction discriminators are employed and no
external time-walk corrections are required. The hit times reconstructed by the TDC readout of the PMTs at the
ends of each scintillation bar (referred to generically here as 1 and 2) are given by:

\begin{multline}
t_{1/2} = (CONV \cdot TDC_{1/2}) - t_{1/2}^{walk} \\ \mp \frac{C_{12}}{2} + C_{RF} + C_{p2p},
\end{multline}

\noindent
where $CONV$ is the TDC channel-to-time conversion factor (0.024~ns/bin), $TDC$ is the measured PMT TDC
value relative to the trigger signal, $t^{walk}$ is the time-walk function to correct for the pulse amplitude
dependence of the crossing times of the discriminator threshold (used only for FTOF), $C_{12}$ is a time offset
to center the PMT TDC difference distribution about 0, and $C_{RF}$ and $C_{p2p}$ are the time offsets to
align all of the counter hit times with respect to the accelerator RF time and to each other, respectively. The
paddle-to-paddle time offsets $C_{p2p}$ mainly account for the signal delays along the cable lengths from the PMT
output to the readout electronics.

The FTOF and CTOF particle hit times relative to the trigger signal can be determined separately from the times
$t_1$ and $t_2$ measured by the PMTs of a given scintillation bar using:

\begin{equation}
\label{thit-1}
t_{hit}^{1/2} = t_{1/2} - \frac{d_{1/2}}{v_{eff}},
\end{equation}

\noindent
where $d_{1/2}$ represents the distances along the bar from the hit point to the PMT given by:

\begin{equation}
d_{1/2}= L/2 \pm y,
\end{equation}

\noindent
with $y$ the hit coordinate along the bar (determined from forward tracking for the FTOF and central tracking
for the CTOF) and $L$ is the counter length. The average counter hit time is given by:

\begin{equation}
\label{thit-2}
\bar{t}_{hit} = \frac{1}{2} ( t_{hit}^1 + t_{hit}^2 ) = \frac{1}{2} \left[ t_1 + t_2 - \frac{L}{v_{eff}} \right],
\end{equation}

\noindent
where $v_{eff}$ is the effective speed of light in the scintillation bar.

Using the timing information from the PMTs at the ends of each bar, the hit coordinate along the bar with respect
to the center of the bar can be defined from the FTOF or CTOF information alone using:

\begin{equation}
\label{tof-coor}
y = \frac{v_{eff}}{2} (t_1 - t_2 - C_{12}).
\end{equation}

\noindent
It is this coordinate determination that is compared against the projected coordinate from tracking to
determine of the time-of-flight hit matches to a projected track.

The algorithm detailed above and currently in use requires good ADC and TDC information for the PMTs at both
ends of the counter to be available. However, if one of the PMTs of a counter is malfunctioning, Eq.(\ref{thit-1})
shows that the hit time recorded from the working PMT alone can be used to reconstruct the particle hit time using
tracking information to correct for the light propagation delay along the counter. The loss of one PMT involves a
$\sqrt{2}$ worse timing resolution. Algorithms to address these cases are already implemented in the reconstruction
service but presently are disabled.

The reconstructed energies from the ADC values of the PMTs (1 and 2) for a given scintillator bar are given by:

\begin{equation}
E_{1/2} = (ADC_{1/2} - PED_{1/2}) \left [ \frac{\left( \frac{dE}{dx} \right )_{MIP} \cdot t}{ADC_{MIP}} \right ],
\end{equation}

\noindent
where $(ADC-PED)$ is the measured pedestal-subtracted ADC integral, $ADC_{MIP}$ is the ADC value for
normally incident minimum-ionizing particles (MIPs) at the center of the scintillation bar,
$\left( \frac{dE}{dx} \right)_{MIP}$ is the energy loss for MIPs in the scintillation bars (1.956~MeV/cm), and
$t$ is the scintillation bar thickness. The deposited energy is computed as the geometric mean of the deposited
energy as determined from the two counter PMTs as:

\begin{equation}
E_{dep} = \sqrt{E_1 E_2}.
\end{equation}

\subsubsection{Hit Clustering and Matching}
\label{sec:tof-cluster}

If there are multiple scintillation bar hits associated with a single incident charged particle track, a hit cluster
can be defined. These clusters have associated with them a hit coordinate, deposited energy, and hit time.
Hits are assigned as part of a cluster in either the FTOF or CTOF if their position and time falls within selected
windows. In each of the sectors of the CLAS12 Forward Detector there are three FTOF counter arrays. In the
polar angle range from 5$^\circ$ to 35$^\circ$ are located the parallel counter arrays called panel-1a and panel-1b.
In the polar angle range from 35$^\circ$ to 45$^\circ$ are located the counter arrays called panel-2. The cluster
algorithm looks to define hit clusters matched to tracks separately in each of these counter arrays.

With hit clusters defined, the associated cluster coordinate along the counter length is defined as the
energy-deposited weighted average of the reconstructed $y$ coordinate from Eq.(\ref{tof-coor}) as:

\begin{equation}
y_{cluster} = \sum_{i=1}^{N} y_i \cdot \Delta E_i.
\end{equation}

\noindent
Note that in both the FTOF and CTOF systems, the maximum cluster size is limited to $N=2$. For the coordinate
transverse to the counter length along the counter width, the coordinate is defined as the average of the
coordinates associated with the middle of the bar.

The assigned cluster energy is the sum of the deposited energies in the counters associated with the defined
cluster,

\begin{equation}
  E_{cluster} = \sum_{i=1}^N E_{dep}^i.
\end{equation}

In the FTOF when there is a defined hit or a defined cluster in both panel-1b and panel-1a, a second cluster matching
algorithm (using timing and hit-coordinate information solely from the FTOF) is applied to determine if the hit
or cluster in panel-1b and the hit or cluster in panel-1a are associated with the same incident track matched to
the panel-1b hit or cluster. If they are associated, a corrected FTOF time based on the panel-1a and panel-1b
cluster times is computed using a time resolution weighting according to the counter in each cluster with the
largest energy deposition using:

\begin{equation}
  t_{corr} = \frac{\displaystyle \frac{\displaystyle t_{1b}^{cluster}}{\displaystyle \delta_{1b}} +
    \frac{\displaystyle (t_{1a}^{cluster} - \Delta r/\beta c)}{\displaystyle \delta_{1a}}}
  {\displaystyle \left( \frac{\displaystyle 1}{\displaystyle \delta_{1b}} +
    \frac{\displaystyle 1}{\displaystyle \delta_{1a}} \right)}.
\end{equation}

\noindent
Here $\delta_{1a,1b}$ are the effective time resolutions measured for the counters determined during the
FTOF calibration procedure and $t_{1a,1b}^{cluster}$ are the cluster hit times in panel-1a and panel-1b. The term
$\Delta r/(\beta c)$ accounts for the path length difference between the panel-1b cluster hit coordinate and
the panel-1a cluster hit coordinate and comes from forward tracking information. As $\beta$ depends on the
FTOF time, it is assumed that it is based on the panel-1b time information (the array with the better timing
resolution).

Given the effective FTOF counter resolutions, the overall FTOF hit time resolution is improved by 15-20\%
when combining the times from panel-1b and panel-1a in this manner. Of course, if the track interacts with only
panel-1a or with only panel-1b due to the slightly different solid angles of coverage of the arrays, then only the
single plane hit time is used in the event reconstruction. 

Note that employing the cluster times has not yet been fully validated in the event reconstruction but is currently
under test using Monte Carlo data samples. While this validation is in progress, the information passed from the
time-of-flight systems to the Event Builder is based on reconstructed hits.
