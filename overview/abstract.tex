\begin{abstract}

The CLAS12 detector is used to study electro-induced nuclear and hadronic reactions by providing efficient detection of charged and neutral particles over a large fraction of the full solid angle. A collaboration of XX institutions has designed fabricated, assembled, and commissioned CLAS12 in Hall B at the Thomas Jefferson  National Accelerator Facility. The CLAS12 detector is based on a combination of of a six coil toroidal magnet and a high field solenoid magnet. The combined magnetic field provides a large azimuthal coverage in both azimuthal and  polar angles. Trajectory reconstruction using drift chambers at forward angles results in a momentum resolution of 0.5\%. At large polar angles vertex resolution is $\approx 200-300\mu\rm{m}$ ({\bf put correct values in} ) with momentum resolution of a few \% ({\bf put correct values here}) .  Cherenkov counters, time-of-flight systems and calorimeters provide good particle identification. Fast triggering and high data acquisition rates allow operation in luminosities of $10^{35}\rm{cm}^{-2}\rm{s}^{-1}$ for extended periods of time. These capabilities are being used in a broad scientific program to study the structure and interactions of baryons, meson and nuclei using polarized and unpolarized targets. This paper is a comprehensive and general description of the design, construction and performance of CLAS12.

\end{abstract}
