
\section{Readout performance}

The summer-fall cosmic ray commissioning phase and the December 2017 engineering run validated the functionality of the 
MVT readout system. The slow control of the front-end electronics was integrated within the overall CLAS12 EPICS 
framework. Continuous monitoring of temperatures, on-board generated voltages and power consumption were possible for 
each of the 48 FEUs as well as for entire readout system. Implemented software interlocks preventing overheating and 
overcurrent of the frontends proved itself efficient.

According to field maps the residual magnetic field of the 5T solenoid may vary from 0.6T to 0.8T in the MVT frontend 
electronics area. It had no apparent influence on the readout system performance. During several ramp-up and ramp-down 
sessions of the magnet no detectable increase of the overall power consumption or of the supply currents of individual 
FEU were detected. The data acquisition was stable as well.

The cosmic ray commissioning permitted finalizing of the MVT acquisition software integration with the CLAS12 CODA 
framework. A suite of utilities was developed to perform automated runs to determine the pedestal equalization 
parameters and zero suppression thresholds for data taking, in addition to a quick analysis of the MVT data quality 
estimation and timing adjustment.

The low (about 20 Hz) trigger rate of the cosmic runs allowed to keep a relatively large number of samples for each 
retained channel with the charge deposition above the programmed zero suppression threshold. 16 samples proved enough 
to encompass completely the detector signals giving the possibility to perform detector performance studies. However, 
the luxury to keep the entire shape of the signals is not possible during the beam operation. Indeed, with the trigger 
rates as high as 10 kHz the CLAS12 event builder throughput will not be enough to absorb the total amount of data 
produced by MVT. The number of samples was reduced to 6 in accordance to the design specifications. The readout timing 
window with the retained samples was centered on the signals' maxima. Two modes of operation were used: consecutive and 
sparse readouts, as shown on Figure 19. The first one implies the readout of consecutive samples. With the sampling rate 
of 25 MHz, the timing window is $6 \times 40 = 240 \text{ns}$ wide. The sparse mode was set to readout every other 
sample (equivalent to a 80 ns period) resulting to the doubled timing window of 480 ns with wider signal shape recorded.


Even with this optimization, the average archived MVT data per event of about 13 Kbyte remained higher than the 
estimated 6 Kbyte used for the design specifications. At 10 kHz trigger rate the so-called EVIO formatted data 
throughput exceeded the bandwidth of the single 1 GB Ethernet link connecting the unique MVT backend to the CLAS12 
event builder network. The raw non-formatted data exchanged between the two SSP/BEU modules and the crate controller 
was saturating the 200 Mbyte/s throughput of the VME backplane. Despite of the bottlenecks, the built-in rate 
regulation capabilities made the MVT system operate stably at about 7.5 kHz max trigger rate imposing dead-time on the 
CLAS12 DAQ. 

Both bottlenecks were removed in January 2018 by splitting the single backend in two backend crates. During the 
following physics run, trigger rates of 10-12 kHz were sustained routinely. Recently further improvement was performed 
by installing 10 GB Ethernet network interface cards on the controllers of the both backend crates. This will allow 
trigger rates of 20 kHz foreseen for the physics run during the fall 2018.

Nevertheless, the difference between the projected MVT event size and the observed one has to be understood. The 
influence of the ambient electro-magnetic noise can be excluded as the vast majority of the collected data contains 
well-shaped detector signals. The data reduction starts to be significant when the zero suppression thresholds are 
above 8$\sigma$ of common-mode corrected pedestal noise, cutting drastically down the number of retained channels with 
otherwise good signal content.

Even if the large data volume turns out to be inherent to operation of the MVT detectors in the high current beam, 
further improvement of sustainable trigger rates is possible. If the studies will indicate an acceptable signal timing 
“walk”, one can a) envisage diminishing the number of samples to 5 reducing both the background and physics generated 
data; and b) diminishing the zero-suppression window from the current 120 ns to 80 ns reducing the contribution from 
the ghost hits generated by the background. An obvious performance improvement would be the distribution of the backend 
functionality from the two to three crates lessening the burden on the VME backplane. About 50 kHz trigger rate can be 
envisaged. The above improvements do not require any additional developments and are already supported by the current 
hardware, firmware and software. Two further modifications are viable but require development efforts of varying 
intensity. Firstly, one can double the bus width within the SSP/BEU firmware increasing the processing and formatting 
of the FEU packets in order to perform the local event building faster. Lastly, one can replace the current signal 
distribution module in the backend by the JLAB VXS-Trigger-Processor and use its 10/40 GB Ethernet port as the 
interface with the event builder thus avoiding completely the VME bus. This ultimate change would allow the use of 
single backend crate. Trigger rates as high as 65 kHz can be expected and even higher (80 kHz if 4-sample readout is 
acceptable).
