% \subsection{Readout system}

% The extremely tight design of the CLAS12 Central Detector leaves a very narrow space between the MVT and its neighbor detectors. Consequently, a readout architecture based on the off-detector front-end electronics has been adopted. Lightweight micro-coaxial cable assemblies with low linear capacitance carry bare unamplified signals to the front-end units (FEU) housed in crates some 1.5--2 m upstream of the detectors. The front-end electronics are responsible for the pre-amplification and shaping of the detector signals, for holding the latter in a pipeline waiting for trigger process to yield, for the digitization and compression of the selected event data and for their delivery to the back-end electronics. The back-end is responsible for data concentration event-wise. It provides an interface with the CLAS12 event building system. It also ensures a fixed latency path between the CLAS12 trigger system and the FEUs. It receives the system clock and trigger from the CLAS12 trigger supervisor and synchronously conveys them to the FEUs over bidirectional optical links.

% To improve MVT readout noise immunity the readout takes advantage of the continuous sampling of detector signals. Pickup noise usually affects groups of neighboring signal lines. It is possible to determine and remove this coherent noise greatly smoothing the induced fluctuations. For each trigger the signal samples are compared to the channel discriminating threshold after the common mode noise subtraction. For channels with charge deposits above the thresholds, a fixed number of consecutive samples are kept for offline analysis. The retained samples describe the signal development in the channel. Fitting their values with a known function allows an accurate estimation of deposited charge and of signal timing.

% The system is dimensioned to read out $6\times10^3$ channels of the forward station and almost $16\times10^3$ channels of the barrel station. The expected 10 to 20 MHz physics background results in strip hit rates of 60 kHz in the forward detectors and of 20 kHz in the barrel detectors. The readout system is compliant with the CLAS12 requirement of a 20 kHz maximum trigger rate and provides a sufficiently deep data pipeline to cope with a trigger latency as long as 16 $\mu$s. The front-end electronics has been designed to withstand the residual 1 T magnetic field of the central detector solenoid.
% The back-end of the data acquisition system of the MVT is based on JLab standard VME/VXS hardware including TI, SD and SSP boards. Dedicated firmware and software was designed to ensure full compatibility with the CODA DAQ software environment deployed in Hall B. MVT data files comply with EVIO requirements. Raw data files can optionally be collected for debugging purposes.