\section{Requirements}

The CLAS12 detector (see Ref.~\cite{overview-ref}) was designed to study the interactions of electrons and photons with nucleons and nuclei at a nominal luminosity of $1\times 10^{35}$~cm$^{-2}$s$^{-1}$ \cite{clas12-nim}. The CLAS12 Trigger System has to provide trigger signals for these processes. Based on the simulation of the physics processes of interest, the required event rate was estimated to be up to 20~kHz. The trigger latency was required to be not less than 8~$\mu$s to provide sufficient time for trigger logic processing.

%(see Fig.~\ref{fig:CLAS12}). 

The following detectors were defined to be part of the trigger system:

\begin{itemize}
	\item High Threshold Cherenkov Counter (HTCC, \cite{htcc-ref})
	\item Drift Chambers (DC, \cite{dc-ref})
	\item Forward Time-of-Flight System (FTOF, \cite{ftof-ref})
	\item Electromagnetic Calorimeter (ECAL, \cite{ec-ref}).
	\item Central Time-of-Flight (CTOF, \cite{ctof-ref})
	\item Central Neutron Detector (CND, \cite{cnd-ref})
	\item Forward Tagger (FT, \cite{ft-ref})
\end{itemize}

Each of these detectors is required to provide information to the Trigger System. To achieve that, the front-end electronics were required to be designed with built-in trigger components. 

 
