\section{Conclusions}

The work on the CLAS12 Trigger System started in 2008. The system was designed and implemented from
2008 until 2017 and has been successfully used during the development, testing, and commissioning phases of
all CLAS12 detectors. In December 2017, the CLAS12 Trigger System was ready for the first beam experiment.
During the first year of operation of CLAS12, the Trigger System was improved to take advantage of its flexible
and powerful design, to account for the performance of the various components, and to add new features to
increase system efficiency and, most of all, purity. By the end of 2018, the system was in full operation mode,
allowing accumulation of data with the portion of ``good'' events on the level of higher than 50\%. The achieved
performance of the CLAS12 Trigger System allows use without significant changes for the entire CLAS12
physics program.

\section{Acknowledgments}

We are grateful to the administrative, engineering, and technical staff of JLab for constant support. The
CLAS12 Trigger System development was conducted in close cooperation with all CLAS12 detector groups, who
contributed to the system design. In particular, we appreciate hard work of the JLab Fast Electronics Group
and the JLab CODA Group personnel. Our special thanks to CLAS12 project leaders Volker Burkert and Latifa
Elouadrhiri for leadership and setting goals. This material is based upon work supported by the U.S.
Department of Energy, Office of Science, Office of Nuclear Physics under contract DE-AC05-06OR23177.

