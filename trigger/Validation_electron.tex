\subsection{Validation of the Electron Trigger}
\label{elctron_trigger_validation}
As a reminder the electron trigger logic uses responces from PCAL, EC, HTCC and DC (see eq. \ref{eq:em_trg_formula}), and as it was described in section \ref{sec:validation_random}, for trigger validation we have used a Random Trigger data. 
The $\mathrm{1^{st}}$ step in the validation of electron trigger is a selection of  events with ``clean electron''. The CLAS12 offline reconstruction software assignes PID to each reconstruction particle \cite{offline-ref} ( for electron PID=11) however in this studies we imposed additional cuts. 
In particular 
\begin{itemize}
 \item DC roads are optimized for tracks originating from the target, that is why in the offline analysis we have put a cut on the vertex ``Z'' coordinate to make sure the track originates from the target.
 \item Selected events where electron hits calorimeters (PCal and EC) in a fiducial region, to make sure the shower is fully reconstructed.
 \item Applied trigger condition cuts on offline cluster energies in the PCAL and EC, also on number of photoelectrons in HTCC.
\end{itemize}
After applying above mentioned cuts, for each of these electrons, we have checked whether the electron trigger bit is set for the corresponding sector. At the end the trigger efficiency is defined as the number of ``Bit  Set'' events over the number of all events with ``clean'' electron.
Both RG-A and RG-B experiments required to have a good (close to $100 \%$) trigger efficiency for electron above $\mathrm{2\;GeV}$. Since both PCAL and EC are sampling calorimeters, $\mathrm{2\;GeV}$ electrons will deposit only part (in average about $25\%$ in our case) of the the total energy. Then because of shower and light fluctuations some $\mathrm{2\;GeV}$ electrons will have less than $\mathrm{25\%}$ energy reconstructed in the calorimeters. Based on this, in the trigger we required the energy to be more than $\mathrm{300 \; MeV}$, which guarantee that more than $\mathrm{99\%}$ of $\mathrm{2\;GeV}$ electrons will deposit energy above the threshold.

\begin{figure}[!htb]
 \centering
 \subfloat[]{\label{fig:em_eff_Allcomponents}\grinp[width=0.25\tw]{img/All_components_65416544.pdf}}
 \subfloat[]{\label{fig:em_nphe_Miss}\grinp[width=0.25\tw]{img/nphe_missed1_65416544.pdf}}
 \caption{(a) Momentum distribution of ``Good electrons''. The brown distribution represents all ``Good electrons'', The blue histograms represents all events where the electron trigger bit was not set, the black histogram represent events, which doesn't have $\mathrm{EC}\times \mathrm{PCal}$ trigger bit, and the red one represent events that missed the electron trigger bit. (b) distribution of the number of photoelectrons for where the electron has more than $\mathrm{2\;GeV}$ energy and missed the HTCC trigger bit.}
 \label{fig:em_missed_events}
\end{figure}

Fig.\ref{fig:em_eff_Allcomponents} shows the momentum distributions of all ``Good'' electrons (in brown), electrons when electron trigger bit was not set  (in blue), when $\mathrm{EC}\times \mathrm{PCAL}$ bit was not set (in black), and red represents events when HTCC bit was not set. As one can see, above $\mathrm{2\; GeV}$ most of events have only HTCC bit missing.  Fig.\ref{fig:em_nphe_Miss} presents the distribution of the number of photoelectrons for events, which have no  HTCC trigger bit. As one can see about $\mathrm{90\%}$ of these events are    at the threshold region (reminder that in the trigger we used 2 photoelectron threshold).  The Trigger System has different from the offline reconstruction  precision of gains and pedestals values. ({\color{Red}probably Ben can comment what was the exact precision}) It could potentially create  such threshold related effects.

\begin{figure}[!htb]
 \centering
 \grinp[width=0.45\tw]{img/em_Efficiency_65416544.pdf}
 \caption{Trigger efficiency as a function of the electron momentum. Efficiency stays above 99\% for entire momentum range.}
 \label{fig:em_eff}
\end{figure}

The final efficiency is shown in the Fig.\ref{fig:em_eff}, where we can see that  the trigger efficiency is above $\mathrm{95\; \%}$
for electrons with momentum above $\mathrm{2\ GeV}$.
