\subsection{Validation of the Electron Trigger}
\label{elctron_trigger_validation}
As a reminder the electron trigger logic uses responses from the PCAL, EC, HTCC, and DC (see eq. \ref{eq:em_trg_formula}), and as was described in Section \ref{sec:validation_random}, for trigger validation we have used Random Trigger data. 
The first step in the validation of an electron trigger is a selection of  events with a ``clean electron''. The CLAS12 offline reconstruction software assigns a PID to each reconstructed particle \cite{offline-ref} (for electrons PID=11), however in these studies, we imposed additional cuts. 
In particular 
\begin{itemize}
 \item DC roads are optimized for tracks originating from the target, which is why in the offline analysis we put a cut on the vertex ``z'' coordinate to make sure the track originates from the target.
 \item Selected events where the electron hits the calorimeters in the fiducial region, to make sure the shower energy is fully reconstructed.
 \item Applied trigger condition cuts on the offline cluster energies in the PCAL and EC, also on number of photoelectrons in the HTCC.
\end{itemize}
After applying the above-mentioned cuts for each of these electrons, we checked whether the electron trigger bit is set for the corresponding sector. At the end the trigger efficiency is defined as the number of ``Bit  Set'' events over the number of all events with a ``clean'' electron.
The CLAS12 experiments required trigger efficiency close to 100\% for electrons above $\mathrm{2\;GeV}$. Since both the PCAL and EC are sampling calorimeters, $\mathrm{2\;GeV}$ electrons will deposit only part (in average about $25\%$ in our case) of their total energy. Because of shower and light fluctuations, some $\mathrm{2\;GeV}$ electrons will have less than $\mathrm{25\%}$ of their energy reconstructed in the calorimeters. Based on this, we required the energy threshold in the trigger to be more than $\mathrm{300 \; MeV}$, which guarantees that more than $\mathrm{99\%}$ of $\mathrm{2\;GeV}$ electrons will deposit energy above the threshold.

\begin{figure}[!htb]
 \centering
 \subfloat[]{\label{fig:em_eff_Allcomponents}\grinp[width=0.25\tw]{img/All_components_65416544.pdf}}
 \subfloat[]{\label{fig:em_nphe_Miss}\grinp[width=0.25\tw]{img/nphe_missed1_65416544.pdf}}
 \caption{(a) Momentum distribution of ``good electrons''. The brown distribution represents all ``good electrons'', the blue histogram represents all events where the electron trigger bit was not set, the black histogram represents events, that do not have a $\mathrm{EC}\times \mathrm{PCAL}$ trigger bit, and the red histogram represents events that missed the electron trigger bit. (b) Distribution of the number of photoelectrons for events where the electron has more than $\mathrm{2\;GeV}$ energy and missed the HTCC trigger bit.}
 \label{fig:em_missed_events}
\end{figure}

Fig.~\ref{fig:em_eff_Allcomponents} shows the momentum distributions of all ``good'' electrons (in brown), electrons when the electron trigger bit was not set  (in blue), when the $\mathrm{EC}\times \mathrm{PCAL}$ bit was not set (in black), and events when the HTCC bit was not set (in red). Above $\mathrm{2\; GeV}$ most events have only the HTCC bit missing.  Fig.~\ref{fig:em_nphe_Miss} shows the distribution of the number of photoelectrons for events that have no  HTCC trigger bit. About $\mathrm{90\%}$ of these events are at the threshold region (a 2 photoelectron threshold was employed).  The Trigger System has a different (from the offline reconstruction) precision of gains and pedestals values, in particular in the trigger the pedesal value is constant for all events in the run, but in the offline reconstruction the pedestal is calculated event by event using first samples of the FADC readout data. This will create  such threshold related effects. The final trigger efficiency is shown in Fig.~\ref{fig:em_eff}, which shows that  the trigger efficiency is above 95\% for electrons with momentum above $\mathrm{2\ GeV}$.

\begin{figure}[!htb]
 \centering
 \grinp[width=0.45\tw]{img/em_Efficiency_65416544.pdf}
 \caption{Trigger efficiency as a function of the electron momentum. The efficiency is above 99\% for the entire momentum range.}
 \label{fig:em_eff}
\end{figure}


