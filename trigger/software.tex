\section{Software}



\subsection{Development Software} Ben

\subsection{Operating Systems} Moffit

Stage 1 and stage 3 of the CLAS12 Trigger System controlled through Arch Linux running in Xilinx arm ...

Stage 2 controlled by CentOS Linux running on Intel controller. As results all 3 stages designed to provide convenient access to ...
using c/c++ programs.


Archlinux with armv7

\begin{itemize}
\item Linux Kernel 4.4.0
  \begin{itemize}
  \item Updates with specific support for Xilinx products
  \item Availaable at https://github.com/Xilinx/linux-xlnx.git
  \item Custom device-tree
  \item Provides FPGA programming interface
  \item Allocates Physical Memory for use with DMA and event buffers.
  \item Standard I$^2c$ and SPI API.
  \end{itemize}

\item Archlinux compiled for armv7
  \begin{itemize}
  \item Available at https://archlinuxarm.org/
  \item Filesystem over NFS.
  \item diskless booting using tftp
  \end{itemize}

\end{itemize}



\subsection{Configuration Software}

\subsubsection{Configuration Files} sergey

CLAS12 Trigger system has large amount of parameters controlling its logic. Those parameters are set by writing values to hadrware registers, and controlled by reading those registers back. System is using ascii files, example is shown on (Fig.). Every line contains key word and the number of corresponding parameter values. Directive 'include' allows to use hierarhical set of configuration files. Normally main configuration file is selected during run startup procedure, and run control software resolves all 'include' directives creating one big configuration file. That file is used to program all trigger harwdare registers, and it is also written to the data stream for bookkeeping purposes. Register contents are read back and results recorded into data stream as well, providing full control of trigger system settings. Normally the same configuration files contains DAQ settings as well, makeing it complete source for entire DAQ/Trigger system settings.


\subsubsection{Timing Setting} ben

One of the important components of the trigger setting process is delay curves measurements. For that purpose software procedure was developed. It includes special trigger configuration file and software tools to change individual subsystem latencies, record set of rate-related scalers, and produce corresponding delay plots (Fig). In a process, most time-precise trigger components latency was set, and all other components latencies were change step by step scanning area around expected latency values. That procedure was repeated every time trigger logic was changed.


\subsubsection{Gain calibration and Threshold settings} ben

One of the important settings in trigger system is FADC ones. As it stated above FADC boards serves as initial stage of the most of the Stage 1 trigger components (except Drift Chamber), and correct pedestal and gain calibration is critical for correct trigger system performance. Pedestal and gain measurements were conducted before run startup, and values were loaded using configuration files. As result, all thresholds in configuration files were set using understandable units such as MeV for calorimeters and the number of photoelectrons for cherenkov counter.
