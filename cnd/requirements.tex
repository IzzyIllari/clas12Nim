\section{Requirements}

Measuring Deeply Virtual Compton Scattering (DVCS) on a neutron target ($en \to e' n' \gamma$) is one of the necessary steps to complete
our understanding of the structure of the nucleon in terms of Generalized Parton Distributions (GPDs) \cite{Mueller:1998fv,Ji:1996ek,Radyushkin:1996nd}.
DVCS on a neutron target allows one to perform a quark-flavor decomposition of the GPDs, combined with the results for DVCS on a proton target.
Moreover, it plays a complementary role to DVCS on a transversely polarized proton target in the determination of the GPD $E$, the least known and least
constrained GPD that enters Ji's sum rule \cite{Ji:1996ek}, which links integrals of GPDs to the total angular momentum of the quarks.
To measure $n$-DVCS on a deuterium target ($ed\to e'n\gamma(p)$) with CLAS12, the electron and the DVCS photon, emitted mainly at small angles, can be detected in the CLAS12 forward calorimeters (ECAL \cite{ec-nim} and FT~\cite{ft-nim}), while the neutron is emitted predominantly (for $\sim 80$\% of the events) at $\theta> 40^o$ in the laboratory frame, with an average momentum around 0.5~GeV.
These kinematic constraint conditions drive the design specifications for the CND. 
With the aid of the CLAS12 fast Monte Carlo tool (FASTMC), the requirements in terms of angular and momentum resolutions on the detected neutrons were determined by studying the missing mass (``$MM$'') of the $e'n'\gamma$ system.
Using realistic resolutions on the electron and photon calculated by FASTMC, it was found that if the neutron momentum resolution is kept below 10\%, its effect on the $MM$ resolution is negligible with respect to the other particles in the reaction \cite{Niccolai:2018qzm}. 

Therefore, considering that the detection capabilities of CLAS12 for electrons and high-energy photons are fixed, the requirements of the CND are:
\begin{itemize}
\item{good neutron identification capabilities for the kinematic range of interest ($0.2<p_n<1.2$ GeV, $40^o<\theta_n<80^o$);}
\item{neutron momentum resolution $\sigma_P/P$ within 10\%.}
\end{itemize}

\subsection{Constraints}
\label{sect_constraints}
The available radial space in the CLAS12 Central Detector is limited by the presence of the Central Time-Of-Flight system (CTOF) \cite{ctof-nim} and of the solenoid magnet \cite{magnets-nim}, which left about 10 cm free. 
However, the CTOF can also be used to detect neutrons, adding an additional 2-3\% of detection efficiency. 
The Central Vertex Tracker (CVT) \cite{svt-nim,mm-nim} can be used as a veto for charged particles.
Finally, the strong fringe field of the 5-T magnetic field required careful consideration for the positioning and the type of the CND PMTs.

After extensive Geant4 simulations and R\&D studies devoted to examine the various options for the CND and its possible photodetectors \cite{Niccolai:2018qzm}, the final design choice was a barrel of standard plastic scintillator bars of trapezoidal cross section, all with their long sides parallel to the beam direction. This geometry is similar to that of the CTOF \cite{ctof-nim}. 

As previously stated, one of the two requirements of the CND is good neutron identification capabilities. If the charged particles are vetoed by the central tracker, the only particles remaining from the target that can be misidentified for neutrons are photons. Using plastic scintillators, the most straightforward way to distinguish neutrons from photons is by measuring their time-of-flight (TOF) and comparing the values of $\beta$:
\begin{eqnarray}
\beta= \frac{l}{TOF\cdot c},
\end{eqnarray}
where $c$ is the speed of light and $l$ is flight path of the particle from the target to the scintillator bar. This can be obtained, in our geometry, as
\begin{eqnarray}
l= \sqrt{z^2+r^2},
\end{eqnarray}
where $z$ and $r$ are the hit position along the beam $z$ axis and in the radial direction, respectively. To obtain $z$ one must measure the time of the hit at both ends of the scintillator bar:
\begin{eqnarray}
z=\frac{1}{2} \cdot v_{eff}\cdot(t_{left}-t_{right}),
\end{eqnarray}
where $v_{eff}$ is the effective velocity of light propagation in the scintillator material. To know $r$ it is necessary to have reasonably small radial segmentation: $r$ will be given by the distance between the target and the middle of the hit paddle.
   
Geant4-based simulations show that to ensure a good photon/neutron separation for the neutron momentum range of the $n$-DVCS reaction, the CND has to be equipped with photodetectors ensuring a time resolution of about 150 ps. 


