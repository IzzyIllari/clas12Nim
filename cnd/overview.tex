\section{Overview}
The Central Neutron Detector (CND) is the outermost of the subsystems composing the Central Detector of CLAS12. It consists of a barrel of three layers of scintillators coupled at their downstream ends with U-turn light guides and read out at their upstream ends by photomultiplier tubes (PMTs) connected to the bars via 1-m-long bent light guides to position them in a fringe-field region of the CLAS12 5-T superconducting solenoid. The CND was installed in the CLAS12 solenoid, and subsequently started its data taking, in the fall of 2017. 
Geant4-based simulations, calibrated with measurements carried out with the CND using cosmic rays muons, showed that the efficiencies obtainable with this detector and its photon-rejection capabilities are sufficient to collect good statistics on the beam-spin asymmetry for the neutron-DVCS reaction over a wide phase space, using the allocated beam time for CLAS12 with a deuterium target \cite{Jlab12_CLAS_n}. The first beam data collected by CLAS12 on a proton target confirmed the design performance. This detector will also be used in other $n$-DVCS experiments \cite{Jlab12_CLAS_n_pol}, and whenever the detection of the recoil neutron may be required ($N^*$ program, for instance, or all the deeply-virtual meson production reactions on the neutron).



