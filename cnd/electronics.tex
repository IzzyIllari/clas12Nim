\section{Electronics and Readout}
The completely resistive high-voltage dividers of the CND were designed following the voltage distribution ratio suggested by Hamamatsu. The tube-base assembly was developed at IPN Orsay with the aim to mechanically match the mild-steel PMT shielding, for a compact and robust design. 
In order to operate the PMTs, high voltages (typically in the range of 1500 V) are provided by multi-channel CAEN SY527 power supplies. 
%(Fig.~\ref{hv_ps_figure}, left). 
The HV boards adopted for the CND are CAEN A734N (16 channels, 3 kV max voltage, 3 mA max current). %(Fig.~\ref{hv_ps_figure}, right). 
The signal of each PMT is sent to an active splitter.
The three splitter modules used for the CND were originally developed by IPN Orsay for the G0 experiment (Hall C, JLab) \cite{Androic:2011rha}. Each module is an active 64-channel splitter with unity gain, so that there is no loss of amplitude. The 64 SMA inputs are placed in the back panel. In the front panel there are 8 8-channel output connectors (DMCH) for the time signals and 4 16-channel output connectors (FASTBUS) for the charge signals. 
The charge signal from the splitter is sent to JLab-designed 16-channel 250~MHz VXS-based flash-ADCs.  
The time signal from the splitter is sent to a constant fraction discriminator (CFD) GANELEC FCC8, originally developed for the TAPS detector in Mainz. Each CFD module is an 8-channel CAMAC unit with LEMO 00 input connectors and 2x8-pin output connectors in differential ECL. The threshold can be set for each channel individually using a manual switch or remote control, and no time-walk adjustement is required for the module.  
The discriminated time signal then goes to the TDC (CAEN VX1290A, 32 channels/board, 25 ps/channel resolution). 
In total, the read-out system includes 3 splitter modules, 19 CFD modules, 5 TDC boards, and 8 ADC boards. 



