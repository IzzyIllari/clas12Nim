\section{Conclusions}

This article presents the requirements, the design, the calibration and reconstruction procedures, and the performance of the Central Neutron Detector for CLAS12. It consists of a barrel of three layers of scintillators coupled at their downstream ends with U-turn light guides and read out at their upstream ends by conventional PMTs connected to the bars via 1-m-long bent light guides and placed in the fringe field region of the CLAS12 solenoid. 
The performance measured with beam data, which agree with the results of our Geant4-based simulations, show that the efficiencies obtainable with this detector and its photon-rejection capabilities allow for the collection of good statistics on the beam-spin asymmetry for the $n$-DVCS reaction over a wide phase space, using the allocated beam time for CLAS12 with a deuterium target~\cite{Jlab12_CLAS_n}. This detector will also be used in other $n$-DVCS experiments \cite{Jlab12_CLAS_n_pol}, and whenever the detection of the recoil neutron may be required (e.g. the excited nucleon $N^*$ program
or all of the deeply virtual meson production reactions on the neutron). 

\section{Acknowledgments}

This work was supported by IN2P3-CNRS (France) and European (Sixth Framework Program I3-HP) funds. This material is also based upon work supported by the U.S. Department of Energy, Office of Science, Office of Nuclear Physics under contract DE-AC05-06OR23177.




