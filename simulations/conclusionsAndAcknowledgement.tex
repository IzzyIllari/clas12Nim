\section{Conclusion}
In this paper the GEMC Geant4 implementation of the CLAS12 detector was presented.
Thanks to the flexibility of C++ and Geant4, and the power of object-oriented programming,
a detailed Geant4 simulation of the CLAS12 detectors was developed in which the geometry,
digitization, and simulation parameters were decoupled from the code.
This allowed the geometry to be implemented through several sources: native Geant4 volumes, imports from CAD engineering models,
and CLAS12 JAVA geometry services. Realistic detector responses that
make use of the data calibration constants to provide output distributions comparable to the data are included
using plugin-like algorithms. The data acquisition response is reproduced by electronic time window algorithms.

The framework has been shown to perform very well when comparing simulation rates to data
and it was an essential component to measure and optimize the CLAS12 detectors design, calibration procedures, performance,
and to improve the significance and accuracy of physics analyses and science goals.

\section{Acknowledgements}

We would like to gratefully acknowledge the Geant4 team
for their school workshops and accurate answers to our questions and problems.

This material is based upon work supported by the U.S. Department of Energy,
Office of Science, Office of Nuclear Physics under contract DE-AC05- 06OR23177.

