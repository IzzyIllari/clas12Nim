\section{Barrel Micromegas Tracker (BMT)}

\subsection{Geometry}

The Micromegas geometry is implemented through the native gemc geometry API.


The strip identification is performed in the Process ID routine.

\subsubsection{Geometry Git Location}
The github location of the gemc perl api script is \url{https://github.com/gemc/detectors/tree/master/clas12/micromegas}.


\subsection{Process ID}
At each geant4 step, the local coordinate in the sensor volume are used to calculate the strip id.
The algorithm includes the lorentz angle based on the magnetic field strength, the angula, pitch between the strips,
the dead zones of the sensitive parts. A virtual electron avalanche is simulated based the energy deposited. The avalanche
is deposited onto one strip or distributed among several to account for the energy sharing.



\subsection{Digitization}

\subsubsection{ADC}
The Micromegas digitization provides the ADC value calculated using the total energy deposited (after hit sharing).


\subsubsection{TDC}
There is no timing information in the output.

\subsubsection{Summary of CCDB Table used}

\begin{itemize}
	\item /geometry/cvt/mvt/bmt\_layer\_noshim
	\item /geometry/cvt/mvt/bmt\_strip\_L1
	\item /geometry/cvt/mvt/bmt\_strip\_L2
	\item /geometry/cvt/mvt/bmt\_strip\_L3
	\item /geometry/cvt/mvt/bmt\_strip\_L4
	\item /geometry/cvt/mvt/bmt\_strip\_L5
	\item /geometry/cvt/mvt/bmt\_strip\_L6
\end{itemize}


\subsection{Digitized Bank}
The digitized output bank has $ID=500$, and the variables are summarized in Table \ref{tab:mmBank}

\begin{table}[h]
	\begin{center}
		\begin{tabular}{| c | c | c |}
			\hline \hline
			Variable         & Description  & Tag  \\
			\hline
              layer  &                                      layer number  &    1   \\
             sector  &                                     sector number  &    2   \\
              strip  &                                      strip number  &    3   \\
               Edep  &                                  energy deposited  &    4   \\
                ADC  &                                               ADC  &    5   \\
			\hline \hline
		\end{tabular}
	\end{center}
	\caption{The digitized micromegas bank}\label{tab:mmBank}
\end{table}

\subsubsection{Time Window}
The timewindow of the Micromegas is set to to 132 ns.

\subsubsection{Background merging algorithm}

\subsubsection{Process Routine Git Repository Location}
The BST hit process routines are located in the repository: \url{https://github.com/gemc/source/tree/master/hitprocess/clas12/micromegas}

