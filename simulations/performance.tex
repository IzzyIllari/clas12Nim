\section{Performance}

The performance of the CLAS12 simulations is measured by comparing the predicted background rates from beam
interactions in the target with the actual experimental rates. The benchmarks are also quantified for each detector
geometry and digitization routine.

\subsection{Comparison of Rates with Data}

On December 17, 2017, the nominal luminosity of 1$\times$10$^{35}$~cm$^{-2}$s$^{-1}$ (75~nA on a 5-cm long
liquid-hydrogen target) was achieved in CLAS12 for the first time. The rates in each of the CLAS12 detectors were
measured.

The drift chamber hit occupancy was compared for both the FT-ON and FT-OFF configurations. The integrated occupancy
in Regions 1, 2, and 3 are summarized in Tables \ref{tab:ftOnComparison} and \ref{tab:ftOffComparison}. The DC
readout time windows of the experiment at the time were different than what was in the simulation, so the data has
been scaled accordingly. The predicted rates agree quite well with the measured ones.

\begin{table}[h]
	\begin{center}
		\begin{tabular}{| c | c | c |}
			\hline \hline
			Region & Data (re-scaled) &  GEMC \\
			\hline
			1 &  2.8\%  & 2.7\% \\
			2 &  0.6\%  & 0.8\% \\
			3 &  1.5\%  & 1.2\% \\
		\hline \hline
		\end{tabular}
	\end{center}
	\caption{Drift chamber hit occupancy comparison between simulation and data for the FT-ON configuration at full
          luminosity.}
        \label{tab:ftOnComparison}
\end{table}

\begin{table}[h]
	\begin{center}
		\begin{tabular}{| c | c | c |}
			\hline \hline
			Region & Data (re-scaled) &  GEMC \\
			\hline
			1 &  1.7\%  & 1.1\% \\
			2 &  0.3\%  & 0.4\% \\
			3 &  0.9\%  & 0.7\% \\
		\hline \hline
		\end{tabular}
	\end{center}
	\caption{Drift chambers hit occupancy comparison between simulation and data for the FT-OFF configuration at full
          luminosity.}
        \label{tab:ftOffComparison}
\end{table}

Similar agreements were found with the rates in the other CLAS12 detectors. In particular:

\begin{itemize}
	\item FTOF: good agreement with data for the PMT currents~\cite{ftof-nim};
	\item CTOF good agreement with data for the upstream PMT counter rates, while the downstream counter rates
	  are about a factor of three lower in the simulation than they are in the data, probably due to the simulation
          not taking into account the Cherenkov light produced in the light guides~\cite{ctof-nim};
	\item FT: good agreement with data for PMT currents and radiation doses~\cite{ft-nim}.
\end{itemize}


\subsection{Benchmarks}

The GEMC event simulation rate has been measured for single and multiple tracks. The numbers reported here
refer to averages over several 2017 laptops, desktops, and computing farm nodes, and refer to running GEMC in
single-threaded mode. The full CLAS12 geometry has been used, with a Runge-Kutta field integration algorithm
and linear interpolation for both the solenoid and the torus fields. Single meson tracks in the forward region are
simulated with an event rate of about 10~Hz. The electron simulation takes about twice as long due to the shower
simulations in the EC and PCAL calorimeters and Cherenkov photon production in both the HTCC and the LTCC, for
an average rate of about 5~Hz.

A quantitative study of the event rate for 3 particles (2 in the Forward Detector, one in the Central Detector)
details the time spent in each detector geometry, digitization, and magnetic field. The particles generated are:

\begin{itemize}
	\item one 7 GeV electron between polar angles 15\mdeg \ and 25\mdeg;
	\item one 2 GeV photon between polar angles 15\mdeg \ and 25\mdeg;
	\item one 2 GeV proton at $\theta=$90\mdeg.
\end{itemize}

The results are shown in Table~\ref{tab:benchmarks}. The final rate for the 3 particles in the complete CLAS12 setup
is 1.7~Hz.

\begin{table}[h]
	\begin{center}
		\begin{tabular}{ |c | c | c | c |} \hline \hline
System	 & event rate (ms)   &  \% of total\\
			\hline
Target   &  1.2   & 0.20  \\
SVT      &  0.9   & 0.16  \\
CTOF     &  0.4   & 0.06  \\
CND      &  0.2   & 0.04  \\
Solenoid &  33.9  & 5.77  \\
MM       &  26.1  & 4.44  \\
HTCC     &  20.5  & 3.49  \\
Torus    &  45.1  & 7.66  \\
FT       &  10.5  & 1.78  \\
DC       &  48.3  & 8.21  \\
RICH     &  23.6  & 4.01  \\
LTCC     &  27.6  & 4.69  \\
FTOF     &  11.3  & 1.02  \\
PCAL     &  186.2 & 31.67 \\
EC       &  152.4 & 25.91 \\ \hline
CLAS12   &  588.1 & 100 \\ \hline \hline
	\end{tabular}
	\end{center}
	\caption{GEMC simulation benchmarks in each CLAS12 detector and magnet. The time spent (by swimming and
          digitization) within each system by 3 particles (2 in the Forward Detector, one in the Central Detector) is
          tabulated. The calorimeters are responsible for more than half the computing time, due to shower simulations.
          Swimming in the magnetic fields accounts for about 13.5\% of the total CPU time. The overall rate for the full
          CLAS12 detector is 1.7~Hz.}
\label{tab:benchmarks}
\end{table}

\noindent Simulations that include the complete beam-target interactions using the nominal luminosity of \cLuminosity\
are made using 124,000 electrons per event. In this case the time to complete one event varies between one and two
minutes depending on the CPU type and available memory.

































