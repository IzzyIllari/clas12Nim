\section{Distribution and Documentation}

The gemc framework is documented on the gemc website \cite{gemc}. This includes the latest news and releases,
examples, procedure details and all options documentation.

The software is distributed in two ways: with docker, or by downloading the source from the its public repository.

\subsubsection{Docker}

A docker container with the necessary libraries to run GEMC and the reconstruction software
is created in the JeffersonLab hub repository \cite{jlabDocker}


The container is tagged, and every tag contains a set version of these libraries:

\begin{itemize}
	\item event generators such as generate-dis, dvcsgen generator executables
	\item gemc with the clas12 geometry
	\item CLARA
	\item Coatjava
\end{itemize}

Collaborators access these containers and all the software inside by using the command:

\begin{lstlisting}[language=Python]
docker run -it --rm jeffersonlab/clas12simulations:iprod bash
\end{lstlisting}

The only requirement is the docker app, avaialble in Windows, Mac and Linux OS flavors.


\subsubsection{Source Code Dowload}

The code repository is \url{https://github.com/gemc/source}. To compile GEMC several libraries are needed:

\begin{itemize}
	\item clhep: Class Library for High Energy Physics \cite{clhep}
	\item xercesc: validating XML parser \cite{xercesc}
	\item geant4: the libraries to simulate the passage of particles through matter \cite{geant4}
	\item qt: a C++ graphic library \cite{qt}
	\item evio: the CLAS12 data format \cite{evio}
	\item CCDB: the calibration database based on mysql \cite{ccdb}
\end{itemize}






repository code handling (fork and pull)


\subsubsection{Clas12tags repository}


\subsubsection{Contribution to code and geometry}



