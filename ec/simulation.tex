\section{Conceptual design}

GEANT simulations showed that the existing electromagnetic calorimeter (EC) of CLAS~\cite{clas6nsim} will not be able to absorb the full energy of the electromagnetic showers produced by electrons and photons with momentum above 5 GeV/c. The leakage from the back of the calorimeter will diminish the energy resolution, see Fig. 1.a. Simple kinematics of π0 decay show that above a momentum of 5.5 GeV/c, the opening angle of the decay photons becomes too small to be resolved with the existing EC, at the distance about 6 m from the target. The readout segmentation of EC is only ∼ 10 cm. 


The design parameters of the PCAL were established using a full GEANT simulations
of the PCAL-EC system. PCAL was positioned in front of the current EC, see Fig. 2. These
studies are described in detail in~\cite{2007001} and are summarized below. The mechanical design
depends on the number of scintillator-lead layers, on the angular coverage of the PCAL, and
on the size of the readout segmentation. These parameters were determined by the physics
requirements for the detection and identification of high energy electrons, photons, and pizeros
via $2\gamma$ decay.



