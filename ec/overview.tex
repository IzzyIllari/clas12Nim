\section{Overview}

A primary goal of the CLAS12 physics program~\cite{Burkert:2018nvj} is to provide a systematic study of
the internal dynamics of the nucleon by accessing the nucleon's Generalized Parton Distributions (GPDs). This
is accomplished through measurements of deeply virtual Compton scattering (DVCS), deeply virtual meson
production (DVMP), and single spin asymmetries (SSA). These experiments require accurate kinematical
analysis of neutral and charged particles at high momentum. In particular, all CLAS12 electroproduction
experiments require the efficient detection and reliable identification of energetic electrons, photons, and
neutrons using the forward electromagnetic calorimeter (ECAL).

For CLAS12~\cite{nim:overview} it was necessary to augment the existing CLAS calorimeter (EC)~\cite{clas6nim}
with a separate pre-shower calorimeter (PCAL) installed in front of the EC on the Forward Detector. Simulations
using the Geant3 package showed that the thickness of the EC alone (16 radiation lengths) would not be sufficient
to absorb the full energy of electromagnetic showers produced by electrons and photons with momenta above
5~GeV. Since the ECAL is used in the trigger this would ultimately reduce the efficiency for detection of the
highest energy electrons. High energy neutral mesons ($\pi^0$,$\eta$) present a challenge as well. Neutral
mesons decay immediately into two photons with an opening angle that decreases as the momentum increases. This
calls for an improved calorimeter position resolution in order to resolve the two decay photons in the full momentum
range of interest up to few GeV.  This is particularly critical for the reconstruction of the DVCS final state, since
a single high energy photon is produced in the reaction $ep \to e'p'\gamma$ and the largest background to this
process is from single $\pi^0$ production, $ep \to e'p'\pi^0$. In addition, direct $\pi^0$ production complements
the DVCS measurements by accessing GPDs at low and high momentum transfer $|t|$. Clearly, accurate and efficient
$\pi^0$ reconstruction is crucial to separate these two processes. Finally, the capability to detect and identify
neutral mesons from the two-photons decay is of particular importance for the CLAS12 meson spectroscopy program,
since these states are part of many of the reactions of interest~\cite{MesonSpec}.

The performance requirements of the PCAL were therefore driven by two factors: 1) electromagnetic shower
containment within the ECAL volume for the highest CLAS12 electron energies to preserve the desired energy
resolution and efficiency and 2) improved spatial resolution to cleanly separate multiple shower clusters.
