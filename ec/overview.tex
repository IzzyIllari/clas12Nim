\section{Overview}

One of the main goals of the CLAS12 physics program is to study the internal nucleon dynamics by accessing the nucleon's Generalized Parton Distributions (GPDs), which can be accomplished through measurements of cross sections and spin asymmetries in Deeply Virtual Compton Scattering (DVCS) and Deeply Virtual Meson Production (DVMP) processes. These experiments depend on detection of neutral and charged particles at high momentum. In addition, all CLAS12 electroproduction experiment require the reliable detection and identification of high energy electrons, using the shower energy from the electromagnetic calorimeter.

High-energy particles will be produced in the interaction of up to 11 GeV electron beams with a variety of targets for the proposed CLAS12 experiments. The electron identification will use shower energy in the electromagnetic calorimeter and, hence, requires good energy resolution for the full range of energies. The separation of a single high energy photon from the photons from $\pi^{0}$ decay is very important for the DVCS experiment. A single high energy photon is produced in the reaction $ep \to ep\gamma$ and the largest background to this process is from single $\pi^{0}$ production, $ep \to ep\pi^{0}$. In addition, direct $\pi^{0}$ production complements the DVCS measurements by accessing GPDs at low and high momentum transfer $|t|$. Clearly, accurate and efficient $\pi^{0}$ reconstruction is crucial to separate these two processes. 

High energy electron and neutral pion detection presents a challenge to the CLAS Electromagnetic Calorimeters (EC) that will be reused for CLAS12. Due to limited radiation length thickness, $15$ r.l., a significant part of the electromagnetic shower from high energy electrons ($>7$ GeV) will leak out from the back of the EC and will not be accounted for. This will result in a worsening of the energy resolution for high energy electrons. Neutral pions decay immediately into two photons with an opening angle that decreases as the $\pi^{0}$ momentum increases. The two photons from $\pi^{0}$ decay could not be distinguished from a single high energy photon by the EC for pion energies above $5$ GeV. For full reconstruction of the high-energy showers and to separate high energy $\pi^{0}$'s and photons, a Pre-shower Calorimeter detector (PCAL) was built and installed in front of the EC for the CLAS12 detector \cite{asryan}. In the following sections, we describe the general geometrical properties of the PCAL.



