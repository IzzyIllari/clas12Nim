\section{Overview}

A primary goal of the CLAS12 physics program is to make a systematic study of internal nucleon dynamics by accessing the nucleon's Generalized Parton Distributions (GPDs). This is accomplished through measurements of Deeply Virtual Compton Scattering (DVCS) and Deeply Virtual Meson Production (DVMP) processes and single spin asymmetries (SSA). 
These experiments depend on detection of neutral and charged particles at high momentum. In particular, all CLAS12 electroproduction experiments require the reliable detection and identification of energetic electrons, photons and neutrons using the forward electromagnetic calorimeter (FEC).

For CLAS12 it was necessary to augment the existing CLAS calorimeter (EC) \cite{clas6nim} with a separate pre-shower calorimeter (PCAL) installed in front of EC. Simulations using GEANT showed that the thickness (15 r.l.) of the EC alone would not be sufficent to absorb the full energy of electromagnetic showers produced by electrons and photons with momentum above 5 GeV/c. Since the FEC is used in the trigger this would ultimately reduce the efficiency for detection of the highest energy electrons.  High energy neutral mesons ($\pi^{0}$,$\eta^{0}$) present a challenge as well. Neutral mesons decay immediately into two photons with an opening angle that decreases as the momentum increases. Unless there is sufficient position resolution in the FEC, the two photons from a decay could be seen as a single high-energy photon.  This would complicate the reconstruction of the DVCS final state, since a single high energy photon is produced in the reaction $ep \to ep\gamma$ and the largest background to this process is from single $\pi^{0}$ production, $ep \to ep\pi^{0}$.  In addition, direct $\pi^{0}$ production complements the DVCS measurements by accessing GPDs at low and high momentum transfer $|t|$. Clearly, accurate and efficient $\pi^{0}$ reconstruction is crucial to separate these two processes. 

The performance requirements of the PCAL were therefore driven by two factors: 1) electromagnetic shower containment within the FEC volume at the full electron energies to preserve the desired energy resolution and efficiency and 2) improved spatial resolution to cleanly separate multiple shower clusters.  




