\section{Summary}
We described the design, assembly, installation and calibration of six new Pre-Shower Calorimeter (PCAL) modules for CLAS12.  The PCAL were built to extend the performance of the CLAS Electromagnetic Calorimeters (EC) for operation using beam energies up to 12 GeV.  Both PCAL and EC are sampling calorimeters utilizing a novel triangular hodoscope geometry with a combination of stereo transverse and threefold longitudinal readout of electromagnetic showers.  The PCAL design extends the performance of the EC by increasing the total radiation length need to full absorb electron and photon showers.  In addition PCAL uses wave-shifting fiber readout to improve both absolute scintillator light yield and light attenuation, while increasing the transverse segmentation improves the spatial resolution by at least a factor of 2.    Preliminary analysis of physics runs with 7.5 and 10.6 GeV electron beams indicate the combined system of PCAL+EC are meeting design goals for position, energy and timing resolution as well as electron trigger efficiency.  The detection efficiency for neutrons shows the expected momentum dependence, although the absolute efficiency does not yet reach the 80$\%$ expected from simulations.  At present we are still in the early stages of studying the luminosity and resolution dependence of the CLAS12 reconstruction efficiency.

\section*{Acknowledgments}

Thanks to F.-X. Girod for the use of his preliminary analysis of pizero energy corrections.  Acknowledgements go to T. Chetry and N. Compton of Ohio University for programming contributions towards defining the PCAL and EC geometry and development of  calibration algorithms and to the Hall B engineering and technical staff for their careful attention to the design, transportation and installation of the PCAL modules.
This material is based upon work supported by the U.S. Department of Energy, Office of Nuclear Physics Division, under contract No. DE-AC05-06OR23177 and was also supported by the National Science Foundation Major Research Instrumentation Project (MRI) grant PHY-0821173.





