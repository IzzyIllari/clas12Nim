\section{Event Reconstruction}

Reconstruction of the FT sub-detector information and the matching between the detectors to determine the type and
three-momentum of the incident particles is implemented in the CLAS12 Java reconstruction framework. Details on
the algorithms and implementation are provided in Ref.~\cite{reconstruction}. In the following we briefly summarize
the main steps and final outputs.

FT-Cal hits are reconstructed from the analysis of the recorded FADC information to extract energy and time;
hits are then associated based on position and time to form clusters whose energy and centroid position are used
as an initial seed to define the three-momentum of the incident particles. Similarly, FT-Hodo hits are reconstructed
from the FADC raw information and matched based on position and timing to form clusters of matching tiles in the
two layers of the detector. These are matched to clusters in the calorimeter based on position and time to distinguish
charged particles from neutrals. Finally, FT-Trk hits are also reconstructed from the raw data and geometrically
grouped to form clusters in each of the detector layers separately. Combinations of clusters in the $x-y$ layers of
each of the two sub-detectors are used to define crosses that are finally matched to calorimeter clusters to improve
the determination of the impact point of the particle.

