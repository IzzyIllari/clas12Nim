\section{Conclusions}

This paper describes the layout and performance of the CLAS12 Forward Tagger. This system was designed to detect
electrons scattered at very small angles, 2.5$^\circ$ to 4.5$^\circ$, and to perform measurements of hadronic reactions
in the kinematics of quasi-real photoproduction. In this regime, the virtual photon exchanged by the electron
interaction with the target has very low four-momentum transfer $Q^2$ and can be considered as a real photon. These
kinematics are ideally suited for the study of hadron production and spectroscopy,  extending the physics reach of the
CLAS12 experiment beyond its original scope.

The Forward Tagger, composed of an electromagnetic calorimeter for electron detection and energy measurements,
a hodoscope to distinguish electrons from photons, and a tracker to precisely measure the electron scattering plane,
was designed to be permanently installed in CLAS12 as an integral part of the beamline. After extensive simulation
and detector prototyping studies, the three Forward Tagger detectors were assembled and tested separately prior
to integration and installation in CLAS12. Upon installation, the full system was commissioned first with cosmic ray
data taking and then with beam during the CLAS12 engineering run. These studies enabled us to optimize the detector
configuration and to consolidate the calibration procedures for all system components before the start of physics
experiments with CLAS12.

The system response has been studied based on different physics reactions to determine acceptance, energy and
timing resolution, and trigger performance. While further improvements are expected based on refinements of the
calibration procedures and reconstruction algorithms, the Forward Tagger performance is qualitatively in agreement
with the system design specifications, enabling the physics program for which this detector system was designed.
 
