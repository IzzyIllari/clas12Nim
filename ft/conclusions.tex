\section{Conclusions}
This paper describes the layout and performance of the CLAS12 Forward Tagger. This system was design to detect electrons scattered at very small angles, 2.5$^\circ$ to 4.5$^\circ$, and perform measurements of hadronic reaction in the kinematics of quasi-real photo-production. In this regime, the virtual photon exchanged by the electron interaction with the target has very low four-momentum transfer $Q^2$ and can be considered as a real photon. This kinematics is ideally suited for the study of hadron production and spectroscopy,  extending the physics reach of the CLAS12 experiment beyong the original scope.
The Forward Tagger, composed by an electro-magnetic calorimeter for the electron detection and energy measurement, an hodoscope to distinguish electrons from photons and a tracker to measure precisely the electron scattering plane, was designed to be permanently installed in CLAS12 as integral part of the beamline. After extensive simulation studies with the CLAS12 GEANT4 framework and prototyping, the three FT detectors were assembled and tested separately prior integration and installation in CLAS12. Upon installation, the whole FT was commissioned first with cosmic ray data taking and specialized run and, in a second time, with beam during the CLAS12 engineering run. These studies enabled us to optimize the detector configuration and consolidate the calibration procedures for the whole FT, to start the physics production. The system response has been studied based on different physics reaction to determine acceptance, energy and timing resolution and trigger performance. While further improvements are expected based on refinements of the calibration procedures and reconstruction algorithms, the FT performance are qualitatively inline with the design specifications, enabling the physics program that FT was designed for.