\section{Signals and Readout}

\subsection{Front-end electronics, FSSR2}

There are 512 channels per module read out by FSSR2 chips, mounted on a hybrid. The FSSR2 ASIC has been developed at Fermilab for the BTeV experiment \cite{FSSR}. The chip features a data-driven architecture (self-triggered, time-stamped). Each of the 128 input channels of the FSSR2 ASIC has a preamplifier, a shaper that can adjust the shaping time (50--125~ns), a baseline restorer (BLR), and a 3-bit ADC. The period of the clock called beam crossing oscillator (BCO) sets the data acquisition time. If a hit is detected in one of the channels, the core logic transmits pulse amplitude, channel number, and time stamp information to the data output interface. The data output interface accepts data transmitted by the core, serializes it, and transmits it to the data acquisition system. To send the 24-bit readout words one, two, four, or six Low Voltage Differential Signal (LVDS) serial data lines can be used. Both edges of the 70~MHz readout clock are used to clock data, resulting in a maximum output data rate of 840 Mb/s. The readout clock is independent of the acquisition clock. Power consumption is $\le$ 4~mW per channel. The FSSR2 is radiation hard up to 5~Mrad. 

\subsection{VSCM}
Each of the four FSSR2 ASICs reads out 128 channels of analog signals, digitizes and transmits them to a VXS-Segment-Collector-Module (VSCM) card developed at Jefferson Lab. The event builder of the VSCM uses the BCO clock timestamp from the data word of each FSSR2 ASIC and matches it to the timestamp of the global system clock, given by the CLAS trigger. The event builder buffers data received from all FSSR2 ASICs for a programmable latency time up to $\sim$16~$\mu$s. The VSCM is set up to extract event data within a programmable lookback window of $\sim$16~$\mu$s relative to the received trigger. The trigger latency is expected to be $\sim$8~$\mu$s.

\subsection{DAQ}
TI, SD, timing.
