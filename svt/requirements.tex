\section{Physics Requirements and Technical Specs}
Essential parts of the CLAS12 physics program, such as the measurement of patron distribution functions, require tracking of low-momentum  particles with a few percent momentum resolution and about one degree angle resolution at large angles \cite{TDR12}. Stable operation of the tracker at instantaneous luminosities up to 10$^{35}~$cm$^{-2}$s$^{-1}$ is required over periods of several years. This is achieved by the central tracker, installed inside the CLAS12 5 T superconducting solenoid magnet,  providing a highly uniform field in the tracking volume and acting as a M\"oller electron shield. Silicon detector technology provides an excellent match to the central tracking system in the CLAS12 configuration, where small space and high luminosity operation is needed for accurate measurements of exclusive processes at high momentum transfer. The silicon energy band gap (1.12 eV at room temperature) is large enough to have a low leakage current due to electron-hole pair generation, while it is small enough to allow production of a large number of charge carriers per unit energy loss of the ionizing particles. The large energy loss per traversed length of the ionizing particle (3.8 MeV/cm for a minimum ionizing particle) due to the high material density (2.33 g/cm$^3$) leads to production of  measurable signals in thin detectors. Because of the high mobility of electrons and holes, silicon detectors can be used in high-rate environments, with charge collection times on the order of ns. The expected integrated luminosity per year in CLAS12 is 500 fb$^{-1}$. The radiation dose for the CLAS12 Silicon Vertex Tracker (SVT) sensors (carbon target) is 2.5~Mrads over the period of 15 years.

The SVT provides tracking capabilities in the central detector region by measuring recoil baryons, large angle pions,  kaons, and protons with tracking efficiency $\ge$ 90$\%$, transverse momentum resolution $\delta$p$_{T}$/p$_{T}$ $\le$ 5$\%$, and angular resolution for polar angles $\delta$$\theta$ $\le$ 10--20 mrad (within 35$^\circ$--125$^\circ$) and azimuthal angles $\delta$$\phi$ $\le5$ mrad (within $\ge$ 90$\%$ of 2$\pi$). The CLAS12 central detector consists of the SVT as the inner detector, surrounded by the Barrel Micromegas Tracker (BMT), the Central Time-Of-Flight system (CTOF), and the Central Neutron Detector (CND). The required momentum resolution is provided by the SVT while the BMT improves polar angle resolution due to the strips crossing at 90$\degree$. Tracks match up with hits in the CTOF system for $\beta$ vs. $p$ measurements (particle identification). The SVT allows reconstructing detached vertices, e.g. K$_{s}\!\to\!\pi^{+}\pi^{-},~\Lambda\!\to\!\pi^{-}p,~\Xi\!\to\!\Lambda\pi$, for an efficient experimental program in strangeness physics.  

%SVT provides tracking capabilities in the central detector region: 
 %\begin{itemize} \itemsep1pt \parskip0pt \parsep0pt
%\item Measure recoil baryons and large angle pions, kaons  
%\item Polar angle ($\theta$) coverage: 35$^\circ$--125$^\circ$
%\item Azimuthal angle ($\phi$) coverage: $\ge$ 90$\%$ of 2$\pi$
%\item Momentum resolution: $\delta$p$_{T}$/p$_{T}$ $\le$ 5$\%$
%\item Angle resolution: $\delta$$\theta$ $\le$ 10--20 mrad, $\delta$$\phi$ $\le5$ mrad
%\item Tracking efficiency: $\ge$ 90$\%$
%\item Match up tracks with hits in the CTOF for $\beta$ vs. p measurement (particle ID)
%\item Reconstructing~detached~vertices,~e.g.~K$_{s}\!\to\!\pi^{+}\pi^{-},~\Lambda\!\to\!\pi^{-}p,~\Xi\!\to\!\Lambda\pi$ for efficient experimental program in strangeness physics
%\item Stable operation in 5 Tesla magnetic field at instantaneous luminosities L=10$^{35} cm^{-2}s^{-1}$
 %\end{itemize}

To satisfy the physics requirements on track momentum resolution, the SVT must have low mass inside the acceptance region. The SVT module position tolerances should be within 20, 500, and 100 $\mu$m across the module, along the module, and along the beam, respectively.
