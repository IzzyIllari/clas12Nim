\section{Conclusions}

The SVT is installed in the CLAS12 spectrometer in Hall B of Jefferson Lab, and the performance of the modules measured during detector integration has been confirmed. No channels were lost during the installation. The SVT barrel has been electrically  tested with the number of defective channels of 0.1$\%$, well within the specification. The chip average ENC noise is uniform, $\sim$1600 e, on par with the leading silicon strip trackers (see Table~\ref{tab:enc-table}). There is no evidence of coherent noise between the modules and other components. The tracker has been commissioned with cosmic rays and integrated as part of the CLAS12 Central Detector. Experience in operating and commissioning the tracker has been gained during the first year of operation. The tracking performance was studied with beam data and matches the physics requirements. 

\begin{table}[hbt]
\begin{tabular}{llll}
\hline
Detector/module      & Strip length, cm & ASIC         & ENC, e\\ \hline
ATLAS barrel           & 13                      & ABCD3A    & 1500  \\
CMS TOB OB1        & 18                      & APV25       & 1100  \\
CDF Run 2b L0        & 24                     & SVX4         & 1600  \\
CLAS12 SVT            & 33                     & FSSR2       & 1600 \\ \hline
\end{tabular}
\caption{ENC of the silicon strip modules from the silicon strip trackers at CERN, Fermilab, and JLab.}
\label{tab:enc-table}
\end{table}

\section{Acknowledgements}

We appreciate the contribution of J.  Andresen, C. Britton, S. Chappa, A. Dyer, J. Hoff, V. Re, and T. Zimmerman to reviewing the design of the HFCB. We are grateful to the administrative, engineering, and technical staff of the Fermilab Silicon Detector Facility and the Carbon Fiber Lab for excellent work on the module assembly. We thank Fermilab's administrative and technical staff for their contribution to the project. We would like to thank Hall B staff, the Fast Electronics Group, the Detector Support Group, and the Micromegas team for their generous support of the SVT project. We  also  thank the CLAS12 collaboration for manning shifts and taking high quality data. This material is based upon work supported by the U.S. Department of Energy, Office of Science, Office of Nuclear Physics under contract DE-AC05-06OR23177.


