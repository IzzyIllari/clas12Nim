\section{Local reconstruction}

The extraction of signals is done with a threshold set based on the signal-to-noise ratio. A hit is created when a pulse height on a channel exceeds a certain signal-to-noise ratio. To account for particle hits with signals shared by adjacent channels due to capacitive coupling, when the signals of neighboring strips exceed a threshold, they are added to the cluster. The number of strips in a cluster is called cluster size or cluster strip multiplicity. In a binary readout system, where the position information is derived from the strip with the highest pulse height, the root mean square of the spatial resolution is given by the readout pitch divided by square root of 12. FSSR2 is a binary chip and the 3-bit ADC is provided for the calibration purposes. Although the precision of the digitized pulse height is poor, it is still possible to use this information in the reconstruction to improve the spatial resolution compare to binary signal processing. The cluster position is determined from the centroid of the signal amplitudes by a center-of-gravity method using charge sharing between neighboring strips due to capacitive coupling. 
