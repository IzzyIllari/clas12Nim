\section{Overview}

slowcontrol overview description \cite{einstein}

The slow controls system provides the necesssary tools for controlling and monitoring all the hardware in the CLAS detector and beamline systems.  User interfaces are unified across diverse types of hardware and detectors, streamlining procedures for novice and expert operators.

Including an alarm system with visible and audible notifications in the control room and email notifications to on-call experts.

The system is based on Experimental Physics Industrial Control System (EPICS), currently version 3.14.12.5.  We run about 90 EPICS input-output controllers (IOCs) interfacing with approximately 50 different types of hardware via various protocols.

Most of the IOCs run on Red Hat Enterprise 7 (RHEL7) rack-mounted Dell servers in the control room.  A few IOCs run in VxWorks 5.X real-time operating systems on Motorolla VME controllers, primarily for high-rate, synchronous beam monitoring, and legacy magnet power supplies.

The system utilizes many hardware and software interlocks.

We utilize two internal channel access gateways, one read-only for a web interface, and the other for avoiding many direct connections to superconducting magnet controls systems.  Wherever appropriate, the standard autosave feature of EPICS is utilized to automatically preserve settings across IOC reboots, and the standard burt save/restore mechanisms.

The user side of the controls system runs on desktop PCs running RHEL7, where all standard linux system software is maintained via Red Hat's standard package managemement system.

Installation and configuration of the desktop and server linux machines, including custom service daemons, cron jobs, IOCs, network disk mounting, and deployment of upgrades and software changes, is managed via puppet and a central server maintained by Jefferson Lab.  Resulting maintainence is almost hands-free, and recovery from hardware failures or power outages is largely automated.  Monitoring of the servers is performed with Nagios, also provided by Jefferson Lab, which provides email notifications on system errors.

Fully integrated legacy, standalone systems into EPICS.

Resiliency

