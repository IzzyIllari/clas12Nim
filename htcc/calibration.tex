\section{Calibration and Event Reconstruction}
\subsection{Gain Calibration of PMTs} We use ET 9823QKB photomultipliers that have a 5" in diameter quartz face plate. These PMTs have a dark current up to 1000 nA, as specified by manufacturer. At a nominal high voltage of around $\sim$ 2,400 V, it was impossible to observe a single photoelectron peak due to a high dark noise rate. Therefore we developed a special procedure for the gain calibration of the PMTs using a fast stable external light source---the core component of the HTCC Light Monitoring System (LMS). The LED light is distributed relatively evenly via an integrating sphere and fiber optics. The PMT signals illuminated by a light source (an array of LED lights) have timing parameters very close to the timing of the dark noise signals. In fact, the signal from an LED light mimics that of the signal from the detector. Since LMS allowed us to generate light pulses of variable intensity and frequency it became possible to observe the PMT response (which was well separated from the exponential dark noise background). \\
\indent Via analysis of FADC histograms, that we obtained at constant intensity of an LED light at a fixed frequency, we were able to extract the  position of single photoelectron peak in cases when the average intensity of light (which is fluctuating) impinging the PMT photocathode is as low as on the order of equivalently a few photoelectrons. Of course this required that we use the capable LMS to generate light pulses at stable average intensity. The results of a single photoelectron peak position can be used for gain matching. This is in the case for if we see that measurements of all PMTs in parallel (at the same LMS settings) need to be calibrated. This can be done by adjusting the high voltage (HV) applied to individual channels. \\
\indent Such a calibration procedure requires us to use a fast and stable light source with an adjustable light intensity and light pulse frequency. The distinguishing feature of the HTCC LMS is that the observed repeatability of results is within 5-10\% of that obtained in runs at different but stable light intensities and frequencies. Consequently there is no need to keep the light source intensity uniform, i.e. stay the same or close to the same in different calibration runs that are taken whenever necessary. Runs were taken at constant HV (Q1) with changing LED light intensity ($\mu$). These runs were fit in batch mode with the production fitter and all runs have good $\chi{^2}$. (to be modified when pictures available) 

\subsection{Response Equalization} Different factors (including imperfections of mirror working surface, dust deposition, condensation of fumes, overall mirror shape distortions and gain instability of the individual PMTs) can lead to a variation in the signal strength from the individual channels---even after comprehensive single photoelectron calibration is complete. These variations should be corrected independently of their physical origin, as the trigger efficiency is heavily dependent on the uniformity of the HTCC response. In the beginning of every physics run period we analyze the first data in order to estimate the signal position in each of the forty-eight channels. We then develop corresponding correction factors, which align the signals between individual channels. These correction factors are then propagated to both the offline reconstruction (CLAS Calibration Data Base, or ccdb) and online trigger gain files.

\subsection{Timing Calibration} Since the HTCC is the part of the trigger, it is required that the timing of the individual channels is aligned on the level of the FADC to aid the online cluster reconstruction. As a result, the timing calibration of the HTCC is done in two steps: the first step is performed on the level of the FADC, and the second step (finer step) is done in the offline calibration. \\
\indent Online calibration is done using the independent trigger from the Forward Tagger. Timing of all forty-eight HTCC channels is aligned in the FADC configuration files by setting the appropriate delays with a precision of 4 ns (the best available using the FADC). Since the timing resolution of individual channels is on the order of about 1 ns we can achieve better resolution of the detector than the 4 ns available from the FADC. To do so, we calculate time at the vertex for each of forty-eight channels and estimate the time shift between individual channels. These time shifts are added to the ccdb for all forty-eight channels and are applied at the reconstruction stage.

\subsection{Event Reconstruction} The goal of the cluster reconstruction algorithm is to reconstruct the real signal strength, time, and position from the raw ADC signal from the FADC board. It is done in two steps:

\begin{enumerate}
    \item In the decoding stage: the signal is converted from the hardware notation (crate, slot, channel) into the CLAS12 notation (sector, layer, component). For each the signal, strength (in the ADC channels) and timing is determined from the threshold crossing and the pedestal is subtracted.
    \item In the reconstruction stage: the signal in terms of the ADC signals is converted into the number of the photoelectrons using the gain constants in the CLAS Calibration Data Base (ccdb). The physical design of the HTCC allows the Cherenkov radiation from one electron to split into up to four channels. In order to reconstruct the real signal strength, we need to combine such split signals into the single cluster. We start by selecting the strongest hit and use it as a starting point of the cluster. Then we look for the adjacent hits within a certain time window. If such hits are found, they will be added to the growing cluster. On the final stage the signal strength is determined as the sum of the individual signals, and the signal time is determined as the average between the individual signals, weighted by the corresponding number of the photoelectrons. The hit angular coordinate is determined as the average between the individual hits forming the cluster. Hits, attributed to the established clusters, are removed from the further consideration, and algorithm starts to look for the next cluster until list of hits is exhausted.
\end{enumerate}