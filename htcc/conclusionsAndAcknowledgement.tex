\section{Conclusions}

The High Threshold Cherenkov Counter has been designed and built to meet all requirements that were defined
mostly by its location and available space in front of the forward drift chambers of the CLAS12 spectrometer. A
new technology of building light-weight multifocal ellipsoidal mirrors was developed and successfully used. The
detector introduces a small amount of material in the CLAS12 acceptance that is only the radiator gas and the
mirror itself that has thickness less than the total radiation length of the CO$_2$ radiator. There are no
elements within the acceptance that support the HTCC mirror. The detector provides full azimuthal coverage and
very efficient light collection: the Cherenkov light is detected after one reflection in 80\% of events and after two
reflections in the remaining 20\% of events. Experiments with an electron beam have confirmed all design
parameters of the detector. The performance of the HTCC is adequate, reliable, and meets all expectations for
the CLAS12 experiments.

\section{Acknowledgements}

We appreciate the contribution and valuable help of D. McKay, C. Apeldoorn, and M. Powers in the development and
implementation of a new technology for the HTCC construction. Only due to their direct involvement in the project
and tight control of results was it possible to solve successfully so many difficult problems, and consequently
completely avoid any compromises that would otherwise lead to reduced overall performance of the detector.
This material is based upon work supported by the U.S. Department of Energy, Office of Science, Office of Nuclear
Physics under contract DE-AC05-06OR23177.



