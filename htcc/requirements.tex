\section{Requirements}

The core requirements for the HTCC are summarized in the Table~\ref{tab:1}. Based on these necessary general
conditions we derived the more specific and essential demands that had to be taken into consideration. As a result,
it was necessary to spend time on the R$\&$D of scaled prototype mirror facets. In order to provide minimal losses
in light collection, i.e. to provide maximal signal strength, an ellipsoidal multifocal mirror design was chosen. This
necessitated a significant upgrade of available machines for manufacturing the parts. Thus the formulated R$\&$D
goals covered both the properties of the mirrors and the equally important choice of construction technology. It must
be mentioned that any polishing of working surfaces was excluded from consideration in the first place due to very
high cost and time-consuming procedures that would have been involved otherwise. \\
\indent With regard to the combined mirror installation, it was critical to avoid the use of any mirror support
structure within the acceptance of the HTCC. One of the problems that was addressed was finding a way to assemble
the mirror to make it a self-supporting, light-weight structure. In this case the mirror design did not allow for
adjustment of individual mirror facets after the combined mirror was fully assembled. Thus the construction and
assembly procedures had to be precise enough to guarantee the geometrical specifications of the multifocal mirror,
which could then be adjusted only as a whole unit. \\
\indent Directly assembling the detector in the experimental hall was impractical for several reasons. It would require
having a controlled clean environment in the area. Additionally, since the HTCC is a single unit detector that covers all
six sectors of the CLAS12 Forward Detector, its size was larger than the space available for assembly, and the final
transportation was heavily regulated beforehand to avoid crucial difficulties. 
 
\begin{table*}[t]
	\centering
	\caption{Core requirements for the HTCC design.}
	\begin{tabular}{ | r | l | }
		\hline
		PARAMETER & DESIGN VALUE \\ 
		\hline
		Working Gas & CO$_2$ @ 1~atm, 25$^\circ$C  \\ 
		\hline
		Angular Coverage & $\theta = 5^\circ - 35^\circ$; $\phi = 0^\circ - 360^\circ$ \\
		\hline
		Threshold & 15 MeV (electrons) \\
		\hline
		Threshold & 4.9 GeV (charged pions) \\
		\hline
		Rejection of charged pions & $0.5 \times 10^3$  ($\sim99\%$ electron detection efficiency)  \\
		\hline
		Overall Dimensions & $\geq 15$~ft and $L = 6$~ft along beam direction  \\
		\hline
		Mirror Type & Combined, self-supporting  \\
		\hline
		Mirror Substrate Structure & Composite  \\
		\hline
		Mirror Thickness & 200~mg/cm$^2$  \\
		\hline
		Number of Channels & $(12 \times 4) = 48$  \\
		\hline
		Photomultiplier Tubes & Photocathode of $\sim$5~in in diameter  \\
		\hline
		Number of Reflections & 1 (in most cases)   \\
		\hline
		Environment & Magnetic field of 35 G (along PMT axis)   \\
		\hline
	\end{tabular}
	\label{tab:1}
\end{table*}

Environmental concerns---such as what gases would be in use, the inside/outside temperature of the detector,
the humidity of the air, as well as the quality of pavement along the transportation route---were also addressed.
Additional requirements with regard to the detector maintenance and year-round controls were applied
to the HTCC by using experience acquired with the Low Threshold Cherenkov Counter built for CLAS
\cite{Adams:2001kk}.
