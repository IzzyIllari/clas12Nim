\section{Requirements}

The core requirements that needed to be met are summarized in the Table \ref{tab:1}.

\begin{table*}[t]
	\centering
	\caption{Core Requirements}
	\begin{tabular}{ | r | l | }
		\hline
		PARAMETER & DESIGN VALUE \\ 
		\hline
		Working Gas & CO$_2@$1 atm, 25$\degree$C  \\ 
		\hline
		Angular Coverage & $\vartheta = 5\degree - 35\degree$; $\varphi = 0\degree - 360\degree$ \\
		\hline
		Threshold & 15 MeV/c (electrons) \\
		\hline
		Threshold & 4.9 GeV/c (charged pions) \\
		\hline
		Rejection of charged pions & $0.5 \times 10^3$  ($\sim99\%$ electron detection efficiency)  \\
		\hline
		Overall Dimensions & $\geq 15$ ft and $L = 6$ ft along beam direction  \\
		\hline
		Mirror Type & Combined, self-supporting  \\
		\hline
		Mirror Substrate Structure & Composite  \\
		\hline
		Mirror Thickness & 200 mg/cm$^2$  \\
		\hline
		Number of Channels & $(12 \times 4) = 48$  \\
		\hline
		Photomultiplier Tubes & Photocathode of $\sim$5 inch in diameter  \\
		\hline
		Number of Reflections & 1 (in most cases)   \\
		\hline
		Environment & Magnetic Field of 35 Gauss (along PMT axis)   \\
		\hline
	\end{tabular}
	\label{tab:1}
\end{table*}

Based on these necessary general conditions we derived the corresponding more specific and essential compulsory demands that had to be taken into consideration. As a result it was necessary to spend time on the R$\&$D of scaled prototype mirror facets. In order to provide minimal loses in light collection, i.e. to provide maximal signal strength, the construction of an ellipsoidal multifocal mirror was chosen. This necessitated a significant upgrade of available machines for manufacturing parts. Thus the formulated R$\&$D goals covered both properties of mirrors and the equally important choice of construction technology. It must be mentioned that any polishing of working surfaces were excluded from consideration in the first place due to very high cost and time consuming procedures that would have been involved otherwise. \\
\indent With regards to the combined mirror installation it was critical that we do not use any mirror support structure within the working acceptance of the HTCC. One of the problems that was addressed was finding a way to assemble the mirror to make it a self-supporting and having light weight structure. In this case the mirror design was therefore such that one cannot individually adjust the mirror facets after the combined mirror is fully assembled. Thus the construction and assembly procedures had to be precise enough to guarantee the geometrical specifications of the multifocal mirror. The multifocal mirror can then be adjusted only as a whole unit. \\
\indent Directly assembling the detector in the experimental hall was impossible for several reasons. It would require having a controlled clean environment in the area. Since the HTCC is a single unit detector that covers all six sectors of CLAS12 its size was larger than the space available for assembly, and the final transportation was heavily regulated beforehand to avoid crucial difficulties. Environmental concerns---such as what gases would be in use, the inside/outside temperature of the detector, the humidity of the air, as well as quality of pavement along transportation routes---have also been addressed. Additional requirements with regards to the detector maintenance and year-round controls were applied to the HTCC by using experience acquired with the Low Threshold Cerenkov Counter built for CLAS6, \cite{second}.

