\section{Simulation} Photomultiplier tubes are available with a face plate (entry window) that can be in various shapes. In the HTCC Cerenkov light is generated along the entire length of a scattered electron's trajectory in the working volume of the detector. The light collection geometry provided by the fully ellipsoidal mirror with point-to-point focusing is valid only in the case when one focal point is in the target position and when the second focal point is at the face of the light detector (PMT). Consequently one must expect considerable changes in the size of image in the focal plane due to the continuous slipping of the light emission point along the electron trajectory. Moreover there is no light emitted by a scattered electron moving from the target until it crosses the entry window of the HTCC and is them detected by the detector. A light collection pattern has been simulated for the exact HTCC geometry to answer following basic questions with regards to the light detector:

\begin{itemize}
\item Which shape (convex or flat) has the most efficient light collection?
\item What window material has the highest possible signal strength?
\item What are the actual image dimensions in the focal planes?
\item What would be the basic dimensions of a Winston Light Concentrator, if we had to use them?
\end{itemize}

In Fig.VII-1 we show results for two different PMTs, comparing those with convex windows and those with flat windows. Clearly the flat entry window provides better results. As far as window material is concerned, as one can expect, the quartz window provides the highest signal as opposed to UV-transmitting glass (Fig.VII-2). Since PMTs with quartz entry windows are significantly more expensive and are also fragile we have run comparative tests of 5" ET 9823 PMTs with both quartz and UV transmitting glass window. These tests have proven the advantages of a PMT with a quartz window, as predicted in MC simulations. The 5” quartz phototube ET-9823QKB used in the HTCC has a photocathode that is actually 110 mm in diameter ($\sim$4.3"). In the following pictures we show the results of the MC simulations of the light collection pattern on the face of PMT1, which is detecting light reflected by a mirror facet that covers a polar angle range of 5$\degree$ to 12.5$\degree$. \\
\indent Data are obtained for 2 GeV electrons for a point-like target with and without 5T solenoidal field generated by the Central Superconductive Solenoid (CSS), and for a 10 cm long target with a 5T field. On all three pictures the white circle (Ø 110 mm) represents the boundary of the PMT light sensitive area. There are very small changes in light collection pattern. Data are presented in logarithmic scale so the most of the light is impinging the photocathode directly. The next three pictures show the distributions of light on the face of the Winston Light Concentrators obtained in the same conditions mentioned above. For illustration there are shown circles of diameter of 181.4 mm. The last picture shows the same results for an extended target and a 5T field with a circle of smaller diameter of 161.4 mm. Based on these results the Winston Light Concentrators used in the HTCC have a fully circular opening of radius R=7.4 cm and length of 190 mm, leaving the PMT photocathode (diameter 110mm) completely open. Estimates for signal strength for 2 GeV electrons have been obtained also for point-like and extended targets with and without 5T field. \\
\indent The following pictures present angular dependencies on the polar angle and the azimuthal angle. One can see that the signal strength is increasing with the polar angle. This is because the electrons scattered at a smaller angle travel a shorter distance in the radiator gas as compared to the electrons moving under larger angles. The minimum signal strength is estimated to be about 14–15 photoelectrons (phe). For electrons scattered in range of polar angles from 5\degree to 35\degree we have a complete and uniform coverage of entire 2$\pi$ acceptance, as demonstrated by the azimuthal dependence. The average signal strength is about 17 phe. This estimate have been obtained by taking into consideration possible reduction of mirror reflectivity due to the unavoidable degradation of the reflective surface during the construction stage (limited stability of reflectivity, dust and fume deposition, mechanical defects etc.) that has first been observed after the construction of the Low Threshold Cerenkov Counter for the CLAS spectrometer. \\
\indent The major source of background events in the HTCC are the results of secondary interactions of charged pions in the working volume of the HTCC, as well as with components outside the detector---especially in the region between the target and the entry window, including the window itself. Charged pions with energies well below the detection threshold can knock out $\delta$ -electrons that can generate Cerenkov light moving in the working volume. In Fig.VII-7 we show the estimates of background rates in four different HTCC channels (one half-sector) at CLAS12, designed with a luminosity of L $\approx$ 10$^{35}$ cm$^{-2}$ sec$^{-1}$. The total background rate is about 20 kHz. \\
\indent The most important parameters for the Cerenkov counters are the electron detection efficiency and the charged pion rejection power. In Fig.VII-8 the rejection of charged pions is shown for the HTCC channels at three different thresholds for electrons at 2 GeV, along with the corresponding estimates for the electron detection efficiency. Similar results for electrons at 4 GeV are shown in Fig.VII-9. One can conclude that at the threshold of 3 phe the average rejection factor is greater than 1000 at 2 GeV and at least 500 at 4 GeV. At the same time electron detection efficiency is close to 100\%.